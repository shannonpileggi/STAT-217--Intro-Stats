

\PassOptionsToPackage{subsection=false}{beamerouterthememiniframes}
\PassOptionsToPackage{dvipsnames,table}{xcolor}
\documentclass[fleqn]{beamer}
\usepackage{graphicx}
\usepackage{multirow}
\usepackage{multicol}
\usepackage{amsmath,amsfonts,amsthm,amsopn}
\usepackage{color, colortbl}
\usepackage{subfig}
\usepackage{wrapfig}
\usepackage{fancybox}
\usepackage{tikz}
\usepackage{fancyhdr}
\usepackage{setspace}
\usepackage{xcolor}
\usepackage{movie15}
\usepackage{pifont}
\usepackage{soul}
\usepackage{booktabs}
\usepackage{fancyvrb,newverbs}
\fvset{fontsize=\footnotesize}
\RecustomVerbatimEnvironment{verbatim}{Verbatim}{}

%\usepackage{fancybox}

\usetheme{Szeged}
\usecolortheme{default}

%\definecolor{links}{HTML}{2A1B81}
%\definecolor{links}{blue!20}
\hypersetup{colorlinks,linkcolor=,urlcolor=blue!80}

\setbeamertemplate{blocks}[rounded]
\setbeamercolor{block title}{bg=blue!40,fg=black}
\setbeamercolor{block body}{bg=blue!10}

%\definecolor{myblue1}{blue!10}

%\colorlet{breaks}{myblue1}

\newenvironment<>{clicker}[1]{%
  \begin{actionenv}#2%
      \def\insertblocktitle{#1}%
      \par%
      \mode<presentation>{%
        \setbeamercolor{block title}{fg=white,bg=magenta}
       \setbeamercolor{block body}{fg=black,bg=magenta!10}
       \setbeamercolor{itemize item}{fg=magenta}
       \setbeamertemplate{itemize item}[triangle]
       \setbeamercolor{enumerate item}{fg=magenta}
     }%
      \usebeamertemplate{block begin}}
    {\par\usebeamertemplate{block end}\end{actionenv}}




\defbeamertemplate*{footline}{infolines theme}
{
  \leavevmode%
  \hbox{%
  \begin{beamercolorbox}[wd=.333333\paperwidth,ht=2.25ex,dp=1ex,left]{author in head/foot}%
    \usebeamerfont{author in head/foot}~~\insertshortinstitute: \insertshorttitle
  \end{beamercolorbox}%
  \begin{beamercolorbox}[wd=.67\paperwidth,ht=2.25ex,dp=1ex,right]{date in head/foot}%
    \usebeamerfont{date in head/foot}%\insertshortdate{}\hspace*{2em}
    \insertframenumber{} / \inserttotalframenumber\hspace*{2ex}
  \end{beamercolorbox}
  }%
  \vskip0pt%
}

\newcommand{\cmark}{\ding{51}}%
\newcommand{\xmark}{\ding{55}}%
\newcommand{\grp}{\textcolor{magenta}{Group Exercise}}
\newcommand{\bsans}[1]{\underline{\hspace{0.2in}\color{blue!80}{#1}\hspace{0.2in}}}
\newcommand{\bs}{\underline{\hspace{0.3in}}}


\definecolor{cverbbg}{gray}{0.93}
\newenvironment{cverbatim}
 {\SaveVerbatim{cverb}}
 {\endSaveVerbatim
  \flushleft\fboxrule=0pt\fboxsep=.5em
  \colorbox{cverbbg}{\BUseVerbatim{cverb}}%
  \endflushleft
}
\newenvironment{lcverbatim}
 {\SaveVerbatim{cverb}}
 {\endSaveVerbatim
  \flushleft\fboxrule=0pt\fboxsep=.5em
  \colorbox{cverbbg}{%
    \makebox[\dimexpr\linewidth-2\fboxsep][l]{\BUseVerbatim{cverb}}%
  }
  \endflushleft
}




\title[Unit 3 Deck 4]{Final Review and Practice Problems}
\author[Pileggi]{Shannon Pileggi}

\institute[STAT 217]{STAT 217}

\date{}


\begin{document}

\begin{frame}
\titlepage
\end{frame}

\begin{frame}
\frametitle{OUTLINE\qquad\qquad\qquad} \tableofcontents[hideallsubsections]
\end{frame}


%===========================================================================================================================
\section[Overview]{Overview}
%===========================================================================================================================

%\subsection{}

\begin{frame}
\frametitle{Final exam}
The final exam \emph{is} cumulative
    \begin{itemize}
    \item
    $\sim$ 20\% Unit 1
    \item
    $\sim$ 20\% Unit 2
    \item
    $\sim$ 60\% Unit 3
    \end{itemize}
    \vskip20pt
    Please bring your calculator! Questions?
\end{frame}

%\begin{frame}
%\Large{Before we get started...  Thanks for being a great class!  I have really enjoyed teaching you this quarter!}
%\end{frame}



\begin{frame}[label=topics]
\frametitle{Topics covered }
\begin{itemize}
    \item
    Descriptive Statistics \hyperlink{descriptive}{\beamerreturnbutton{More}}
    \item
    Study Design \hyperlink{studydesign}{\beamerreturnbutton{More}}
    \item
    Associations \hyperlink{associations}{\beamerreturnbutton{More}}
    \item
    Distributions and Probability \hyperlink{probability}{\beamerreturnbutton{More}}
    \item
    Sampling Distributions \hyperlink{sampling}{\beamerreturnbutton{More}}
    \item
    Confidence Intervals \hyperlink{CIs}{\beamerreturnbutton{confidence intervals}}
    \item
    Hypothesis Tests \hyperlink{tests}{\beamerreturnbutton{steps}} \hyperlink{errors}{\beamerreturnbutton{errors}}
    \item
    Different Methods \hyperlink{overview}{\beamerreturnbutton{overview}} \hyperlink{methods}{\beamerreturnbutton{tests}} \hyperlink{pvals}{\beamerreturnbutton{calculations}} \end{itemize}
\end{frame}


\begin{frame}
\begin{clicker}{Which of the following are \underline{true} statements about $p$-values?  Mark \underline{all} that apply.}
\begin{enumerate}
    \item
    A nonsignificant difference (eg, $p$-value$>$0.05) means that the null hypothesis is true.
    \item
    Sample size can affect your $p$-value.
    \item
    A scientific conclusion should be based solely on whether or not the $p$-value is significant.
\end{enumerate}
\end{clicker}
\end{frame}


\begin{frame}
\frametitle{Final Remarks}
\begin{itemize}
    \item
    quantitative methods/statistics are applied in \emph{all} disciplines, regardless of whether you are in the humanities, social sciences, or natural sciences
    \item
    a $p$-value isn't the end of the story
    \begin{itemize}
        \item
        association does not mean causation
        \item
        a statistically significant result isn't always meaningful
    \end{itemize}
    \item
    sometimes statistical analysis is the end result of the research, but sometimes it is just the beginning....
\end{itemize}
\end{frame}




%===========================================================================================================================
\section[Data]{Data}
%===========================================================================================================================
\begin{frame}
\tableofcontents[currentsection, hideallsubsections]
\end{frame}

%\subsection{}
\begin{frame}
\begin{columns}
\column{0.6\textwidth}
\includegraphics[width=1.0\textwidth]{Figures/ufo_conf.png}\\
\column{0.4\textwidth}
\begin{itemize}
    \item
   established in 1991
   \item
   largest \href{http://ufocongress.com/}{conference} on UFOs in US
   \item
   registration $\sim$\$200
\end{itemize}
\end{columns}
\begin{itemize}
\item[]
    \item
   $>$20 speakers discussing topics related to the UFO phenomenon including technology, government cover-ups, exopolitics, black projects, crop circles, alien visitation and more
   \item
   speakers include astrophysicists, nuclear physicists, abductees, and former top-secret-clearance military personnel
\end{itemize}
\end{frame}

\begin{frame}
\frametitle{The data}
\begin{columns}
\column{0.8\textwidth}
\begin{itemize}
    \item
    Feb 21-27, 2010 in Laughlin, Nevada
    \item
    anonymous survey of conference attenders
    \item
    collected by the Dept of Sociology at Baylor University (Texas)
    \item
    400 surveys distributed, 156 returned, 104 used
    \item
    97 variables
    \begin{itemize}
        \item
        UFO beliefs and theories
        \item
        UFO experiences and beliefs about government conspiracies related to UFOs
        \item
        non-UFO paranormal beliefs and experiences
        \item
        religion
        \item
        demographics
    \end{itemize}
\end{itemize}
\column{0.3\textwidth}
%\pause
\includegraphics[width=1.0\textwidth]{Figures/book.png}
\end{columns}
\end{frame}


\begin{frame}
\frametitle{The data, $n=104$}
\includegraphics[width=0.8\textwidth]{Figures/summaryufo.png}\\
\vskip10pt
%\pause
\includegraphics[width=0.25\textwidth]{Figures/hist_age.pdf}
\includegraphics[width=0.25\textwidth]{Figures/hist_kids.pdf}
\includegraphics[width=0.25\textwidth]{Figures/hist_salary.pdf}
\includegraphics[width=0.25\textwidth]{Figures/hist_edu.pdf}
\end{frame}

%===========================================================================================================================
\section[Practice]{Practice}
%===========================================================================================================================
\begin{frame}
\tableofcontents[currentsection, hideallsubsections]
\end{frame}

%\subsection{}
\begin{frame}
\frametitle{Research question 1}
\framesubtitle{Which null hypothesis?}
\begin{clicker}{The average years of education of the general population is 13, whereas the average years of education among the 104 respondents is 16.  Is average years of education for those interested in UFOs the same as the general population?}
\begin{enumerate}
    \item
    $H_0: \mu_1=\mu_2$
    \item
    $H_0: \bar{x}_1=\bar{x}_2$
    \item
    $H_0: \mu_0=13$
    \item
    $H_0: \mu_0=16$
    \item
    $H_0: \mu=16$
    \item
    $H_0: \mu=13$
      \item
    $H_0: \bar{x}=13$
    \item
    $H_0: \bar{x}=16$
\end{enumerate}
\end{clicker}
\end{frame}

\begin{frame}
\frametitle{Research question 2}
\framesubtitle{Which method?}
\begin{clicker}{Is whether or not an individual had a UFO experience associated with years of education?}
\begin{enumerate}
    \item
    one sample z-test
    \item
    two sample z-test
    \item
    chi-squared test
    \item
    one sample t-test
    \item
    two sample t-test
    \item
    paired t-test
    \item
    ANOVA
    \item
    Linear regression
\end{enumerate}
\end{clicker}
\end{frame}

\begin{frame}
\frametitle{Research question 3}
\framesubtitle{Which method?}
\begin{clicker}{Is there an association between gender and belief in big foot?}
\begin{enumerate}
    \item
    one sample z-test
    \item
    two sample z-test
    \item
    one sample t-test
    \item
    two sample t-test
    \item
    paired t-test
    \item
    ANOVA
    \item
    Linear regression
\end{enumerate}
\end{clicker}
\end{frame}



\begin{frame}
\frametitle{Research question 4}
\framesubtitle{Which method?}
\begin{clicker}{Is salary associated with years of education?}
\begin{enumerate}
    \item
    one sample z-test
    \item
    two sample z-test
    \item
    chi-squared test
    \item
    one sample t-test
    \item
    two sample t-test
    \item
    paired t-test
    \item
    ANOVA
    \item
    Linear regression
\end{enumerate}
\end{clicker}
\end{frame}

\begin{frame}
\frametitle{Research question 5}
\framesubtitle{Which method?}
\begin{clicker}{Suppose we categorize years of education as $<$high school, high school, and $>$high school, and we want to determine if there is an association between salary and level of education.}
\begin{enumerate}
    \item
    one sample z-test
    \item
    two sample z-test
    \item
    chi-squared test
    \item
    one sample t-test
    \item
    two sample t-test
    \item
    paired t-test
    \item
    ANOVA
    \item
    Linear regression
\end{enumerate}
\end{clicker}
\end{frame}


%\begin{frame}
%\frametitle{Research question 6}
%\framesubtitle{Which method?}
%\begin{clicker}{Adjusting for years of education, is salary associated with whether or not the individual believes in big foot?}
%\begin{enumerate}[A.]
%    \item
%    one sample z-test
%    \item
%    two sample z-test
%    \item
%    chi-squared test
%    \item
%    one sample t-test
%    \item
%    two sample t-test
%    \item
%    paired t-test
%    \item
%    ANOVA
%    \item
%    Linear regression
%    \item
%    Multiple linear regression
%\end{enumerate}
%\end{clicker}
%\end{frame}


\begin{frame}
\frametitle{Contingency tables}
\begin{center}
\begin{tabular}{|l|cc|r|}
      \hline
               &\multicolumn{2}{|c|}{Beliefs} & \\
               &   Cosmic Force & God & Total\\
      \hline
      Bigfoot Yes     &   51   & 36 & 87 \\
      Bigfoot No      &   9    & 8 & 17 \\
       \hline
       Total  &   60  & 44 & 104 \\
       \hline
\end{tabular}
\end{center}
\begin{clicker}{Which of the following is \underline{false}?}
\begin{enumerate}
    \item
    Among those who believe in God, 36/44=81.8\% believe in bigfoot.
    \item
    Among those who believe in cosmic force, 51/104=49.0\% believe in bigfoot.
    \item
    Overall, the proportion of individuals who believe in bigfoot is higher than the proportion of individuals who believe in God.
    \item
    All of the above are true.
\end{enumerate}
\end{clicker}
\end{frame}

\begin{frame}
\frametitle{Correlation}
\begin{clicker}{The correlation between years of education and salary is 0.22.  This means that}
\begin{enumerate}
    \item
    As annual salary increases by \$1, education increases by 0.22 years.
    \item
    As education increases by one year, annual salary increases by \$0.22.
    \item
    Since the correlation is not 0, we can predict a salary perfectly from years of education.
    \item
    The relationship between salary and years of education follows a curve rather than a straight line.
    \item
    As one of these variables increases, there is a tendency for the other variable to increase also.
\end{enumerate}
\end{clicker}
\end{frame}

\begin{frame}
%\frametitle{SLR}
\includegraphics[width=0.6\textwidth]{Figures/rq3.png}\\
\begin{clicker}{\small{What is the estimated regression equation to predict salary  by years of education ?}}
\begin{enumerate}\small{
    \item
    $\hat{y}= 5024 + 4053 \times years\textunderscore edu$
    \item
    $\hat{y}= 5024 + 4053 \times salary$
    \item
    $\hat{y}= 4053 + 5024 \times salary$
    \item
    $\hat{y}= 4053 + 5024 \times years\textunderscore edu$
    \item
    $\hat{y}= 27855 + 1720 \times years\textunderscore edu$
    \item
    $\hat{y}= 1720 + 27855 \times salary$}
\end{enumerate}
\end{clicker}
\end{frame}


\begin{frame}
\frametitle{SLR}
\begin{columns}
\column{0.6\textwidth}
\includegraphics[width=1.0\textwidth]{Figures/rq3.png}
\column{0.4\textwidth}
\begin{clicker}{What is the null hypothesis tested on the line where the p-value is \texttt{0.0204}?}
\begin{enumerate}
    \item
    $H_0$: $\mu_d=0$
    \item
    $H_0$: $\beta_0=0$
    \item
    $H_0$: $\beta_1=0$
    \item
    $H_0$: $\mu_1=\mu_2$
\end{enumerate}
\end{clicker}
\end{columns}
\end{frame}

\begin{frame}[fragile]
\begin{lcverbatim}
> cor(ufo$years_edu,ufo$salary)
[1] 0.2272182
\end{lcverbatim}
\includegraphics[trim = 0mm 0mm 0mm 20mm, clip,width=0.6\textwidth]{Figures/rq3.png}\\
\begin{clicker}{\small{Which of the following statements is \underline{true}?}}
\begin{enumerate}
    \small{
    \item
    We have evidence of an association between salary and years of education, and the association is strong.
    \item
    Although we have evidence of an association between salary and years of education, the association is weak.
    \item
    We do not have evidence between salary and years of education.
}
\end{enumerate}
\end{clicker}
\end{frame}

%\begin{frame}
%\frametitle{Clicker}
%\begin{columns}
%\column{0.6\textwidth}
%\includegraphics[width=1.0\textwidth]{Figures/rq3.png}
%\column{0.4\textwidth}
%\begin{clicker}{Which of the following is \underline{true} regarding the residual standard error?}
%\begin{enumerate}[A.]
%    \small{
%    \item
%    If the residual standard error were to increase, the $R^2$ would also increase.
%    \item
%    All observed data points are \$47,010 above the predicted value of salary.
%    \item
%    As education increases by one year, annual salary increases by \$47,010.
%    \item
%    The standard deviation of salary for a fixed years of education is \$47,010.}
%\end{enumerate}
%\end{clicker}
%\end{columns}
%\end{frame}


\begin{frame}
\begin{columns}
\column{0.4\textwidth}
\includegraphics[width=1.0\textwidth]{Figures/edu_salary.pdf}
\column{0.6\textwidth}
\begin{clicker}{Which conditions of linear regression are not satisfied?}
\begin{enumerate}
    \item
    independence of observations
    \item
    linear relationship between $x$ and $y$
    \item
    constant variability in $y$ about the regression line
    \item
    no conditions are violated
\end{enumerate}
\end{clicker}
\end{columns}
\end{frame}

\begin{frame}
\framesubtitle{ANOVA to determine if mean salary differ by level of education ($<$HS, =HS,$>$HS)}
\begin{tabular}{lrrrrr}
            & Df & Sum Sq & Mean Sq & F value & Pr($>$F) \\
    source  &      2  &   267  &  133.3 &   0.477 &   0.623 \\
    Residuals &   101&   15932  &  279.5   \\
\end{tabular}
\begin{clicker}{How can we interpret these ANOVA results?}
\begin{enumerate}
    \small{
    \item
    We have evidence that the mean salary is the same for all three levels of education.
    \item
    We have evidence that the mean salary is different for all three levels of education.
    \item
    We have evidence that the mean salary differs for at least two levels of education.
    \item
    We do not have evidence that the mean salary differs by level of education.}
\end{enumerate}
\end{clicker}
\end{frame}


\begin{frame}
\framesubtitle{Is whether or not an individual had a UFO experience associated with years of education?}
\includegraphics[width=0.90\textwidth]{Figures/rq1_noCI.png}\\
\vskip10pt
\begin{clicker}{What can we say about 95\% CI for $\mu_1-\mu_2$?}
\begin{enumerate}
    \item
    it would contain $\bar{x}_1=15.7$
    \item
    it would contain $p=0.046$
    \item
    it would contain zero
    \item
    it would not contain zero
\end{enumerate}
\end{clicker}
\end{frame}

\begin{frame}
\frametitle{Sampling Distributions}
\begin{clicker}
{For which of the following scenarios is the sampling distribution of the sample mean approximately normally distributed?}
\begin{enumerate}
    \item
    Population is right skewed and $n = 10$
    \item
    Population is left skewed and $n = 40$
    \item
    Population is normal and $n = 10$
    \item
    1, 2 and 3
    \item
    2 and 3 only
\end{enumerate}
\end{clicker}
\end{frame}

%
%%===========================================================================================================================
%\section[HW]{HW}
%%===========================================================================================================================
%\begin{frame}
%\tableofcontents[currentsection, hideallsubsections]
%\end{frame}
%
%
%\subsection{}
%\begin{frame}
%\frametitle{Homework}
%For \textbf{Friday}:
%\begin{itemize}
%   \item
%   Don't forget to submit Lab 10 Practice and Lab 10 Peer Evaluation by 9 am on Friday
%   \item
%   Come prepared to Lab 11 by reading the lab manual and watching the lab video
%   \item
%   Lecture HW 9 is due on Blackboard by 5pm Friday
%   \item
%   Complete the QTM 100 survey
%    \item[]
%\end{itemize}
%Looking ahead:
%\begin{itemize}
%   \item
%   Final exam Tuesday, May 7 12:30-3:00
%   \item
%   And have a \textbf{great} summer!
%\end{itemize}
%\end{frame}

%===========================================================================================================================
% Extra
%===========================================================================================================================

%\appendix
%\newcounter{finalframe}
%\setcounter{finalframe}{\value{framenumber}}


\section[Review Slides]{Review Slides}

%\subsection{}

\begin{frame}[label=descriptive]
\frametitle{Descriptive statistics}
\begin{itemize}
    \item
    summarizing and visualizing categorical variables
    \item
    summarizing and visualizing quantitative variables
    \item
    statistics that are/are not robust to outliers
\end{itemize}
\begin{flushright}
\hyperlink{topics}{\beamerreturnbutton{Back}}
\end{flushright}
\end{frame}


\begin{frame}[label=studydesign]
\frametitle{Study design}
\begin{itemize}
    \item
    recognizing observational vs experimental studies
    \item
    potential sources of bias in a study (sampling bias, nonresponse bias, response bias)
      \item
    recognizing types of observational study design (simple, stratified, cluster)
    \item
    recognizing response vs explanatory variables
    \item
    identifying potential confounding variables
    \item
    association does not imply causation
\end{itemize}
\begin{flushright}
\hyperlink{topics}{\beamerreturnbutton{Back}}
\end{flushright}
\end{frame}

\begin{frame}[label=associations]
\frametitle{Associations}
\begin{columns}
\column{0.20\textwidth}
\vskip20pt
Association? \\
\vskip40pt
No association?
\column{0.25\textwidth}
\centering{\footnotesize{quantitative-quantitative}}\\
\centering{$r$, $b_1$}\\
\includegraphics[width=1.0\textwidth]{Figures/scatter_weight.pdf}\\
\includegraphics[width=1.0\textwidth]{Figures/scatter_height.pdf}
\column{0.25\textwidth}
\centering{\footnotesize{categorical-quantitative}}\\
\centering{$\bar{x}_1$, $\bar{x}_2$}\\
\includegraphics[width=1.0\textwidth]{Figures/bp_height_sex.pdf}\\
\includegraphics[width=1.0\textwidth]{Figures/bp_bmi_sex.pdf}
\column{0.25\textwidth}
\centering{\footnotesize{categorical-categorical}}\\
\centering{$\hat{p}_1$, $\hat{p}_2$}\\
\includegraphics[width=1.0\textwidth]{Figures/bar_weapon_gender.pdf}\\
\includegraphics[width=1.0\textwidth]{Figures/bar_weapon_bullied.pdf}
\end{columns}
\begin{center}
\textcolor{OrangeRed}{\small{We need formal statistical tests to determine the direction, magnitude, and significance of the association!}}
\end{center}
\begin{flushright}
\hyperlink{topics}{\beamerreturnbutton{Back}}
\end{flushright}
\end{frame}



\begin{frame}[label=probability]
\frametitle{Probability and Distributions}
Distributions:
    \begin{itemize}
        \item
        normal/$z$, $t$%, chi-squared
        \item
        68 - 95 - 99.7 rule for normal
        \item[]
    \end{itemize}
  Interpreting a probability:
    \begin{itemize}
        \item
        is it large or small?
        \item
        would the event be likely to occur by random chance alone?
    \end{itemize}
\begin{flushright}
\hyperlink{topics}{\beamerreturnbutton{Back}}
\end{flushright}
\end{frame}


\begin{frame}[label=sampling]
\frametitle{Sampling Distributions}
    \begin{itemize}
        \item
        For a random sample of size $n$ from a population with proportion $p$, then when $np\geq10$ and $n(1-p)\geq10$ the \textbf{sampling distribution} of the \textbf{sample proportion} is normally distributed with $\mbox{mean}(\hat{p})=p$ and $\mbox{sd}(\hat{p})=\sqrt{\frac{p(1-p)}{n}}$.
        \item[]
        \item
        When underlying population distribution is \underline{normally distributed} or sample size large enough such that CLT applies ($n>30$), then when sampling from a population with mean $\mu$ and standard deviation $\sigma$ the \textbf{sampling distribution} of the \textbf{sample mean} is normally distributed with $\mbox{mean}(\bar{x})=\mu$ and $\mbox{sd}(\bar{x})=\frac{\sigma}{\sqrt{n}}$.
    \end{itemize}
    \begin{flushright}
    \hyperlink{topics}{\beamerreturnbutton{Back}}
    \end{flushright}
\end{frame}

\begin{frame}[label=CIs]
\frametitle{Confidence intervals}
\begin{itemize}
    \item
    used to \emph{estimate} plausible values for a parameter of interest (like a mean or a proportion)
    \item
    CIs take the form
    \begin{center}
    $estimate \pm (z^*\mbox{or } t^*) \times se$
    \end{center}
    \item
    true meaning: in the long run, if we took many samples and calculated many confidence intervals, 95\% of 95\% CIs would actually capture the true parameter value
    \item
    know how to interpret
    \item
    understand how a confidence interval changes when $n$, $s$, or the confidence level changes
\end{itemize}
\begin{flushright}
\hyperlink{topics}{\beamerreturnbutton{Back}}
\end{flushright}
\end{frame}


\begin{frame}[label=tests]
\frametitle{Hypothesis tests}
\begin{enumerate}
    \item
    Define the parameter of interest
    \item
    State the null and alternative hypotheses
    \item
    Identify the appropriate test
    \item
    State assumptions
    \item
    Calculate the test statistic
    \begin{center}
    $\displaystyle \mbox{test statistic} = \frac{\mbox{sample statistic}-\mbox{null hypothesis value}}{\mbox{standard error of the sample statistic}}$
    \end{center}
    \item
    Calculate the $p$-value (for STAT 217, provided by R!)
    \item
    State your conclusion
\end{enumerate}
\begin{flushright}
\hyperlink{topics}{\beamerreturnbutton{Back}}
\end{flushright}
\end{frame}


\begin{frame}[label=errors]
\frametitle{Types of Errors}
\begin{table}[H]
\caption{Possible outcomes of a hypothesis test}
\begin{center}
%\resizebox{0.7\textwidth}{!}{
 \begin{tabular}{l|cc|}
    \cline{2-3}
    \textbf{Decision based} & \multicolumn{2}{|c|}{\textbf{Unknown Truth}}\\
    \textbf{on observed data}                                                         & $H_0$ true                   & $H_0$ false \\
    \cline{2-3}
     Fail to reject $H_0$ &\textcolor{blue}{Correct Decision}  & \textcolor{red}{Type II Error}\\
                                            Reject $H_0$ & \textcolor{red}{Type I Error} & \textcolor{blue}{Correct Decision}\\
  \cline{2-3}
\end{tabular}%}
\end{center}
\end{table}
\vskip10pt
Confidence intervals and hypothesis tests results should agree:
\begin{itemize}
        \item
        When you reject $H_0$, the corresponding CI \textbf{should not} include the null value tested in $H_0$.
        \item
        When you fail to reject $H_0$, the corresponding CI \textbf{should} include the null value tested in $H_0$.
\end{itemize}
\begin{flushright}
\hyperlink{topics}{\beamerreturnbutton{Back}}
\end{flushright}
\end{frame}


\begin{frame}[label=overview]
\frametitle{Overview of Statistical Methods}
\begin{columns}
\column{0.45\textwidth}
\underline{Quantitative variable - means}
\begin{itemize}
    \item
    one sample t-test
    \item
    paired t-test
    \item
    two sample t-test
    \item
    anova
\end{itemize}
\column{0.55\textwidth}
\underline{Categorical variable - proportions}
\begin{itemize}
    \item
    one sample z-test
    \item
    \textcolor{gray}{N/A for STAT 217}
    \item
    two sample z-test
    \item
    chi-squared test
\end{itemize}
\end{columns}
\vskip20pt
Neither a mean or a proportion: simple linear regression.
\begin{flushright}
\hyperlink{topics}{\beamerreturnbutton{Back}}
\end{flushright}
\end{frame}


\begin{frame}[label=methods]
\frametitle{Different methods}
\resizebox{1.0\textwidth}{!}{
\begin{tabular}{|lllll|}
    \hline
    \textbf{Method} & \textbf{Use} & \textbf{Variables} & \textbf{Estimation} & \textbf{Testing}\\
    \hline
    Single proportion & categorical response  & one categorical variable & CI for $p$ & $H_0$: $p=p_0$\\
    \emph{(one-sample $z$-test)}                  & in single group & & &\\
    \hline
    $^*$Two proportions &  categorical response  & two categorical variables &  CI for $p_1-p_2$ & $H_0$: $p_1=p_2$\\
    \emph{(two-sample $z$-test)}                   &  in two groups & & &\\
    \hline
    $^*$Chi-squared test & categorical response & two categorical variables & N/A & $H_0$: no association/\\
                       & in $\geq 2$ groups & & &\hspace*{0.2in} vars independent\\
    \hline
    Single mean &  quantitative response & one quantitative variable  & CI for $\mu$ & $H_0$: $\mu=\mu_0$\\
    \emph{(one-sample $t$-test)}                  & in single group & & &\\
    \hline
    $^*$Two means & quantitative response &  one quantitative variable and & CI for $\mu_1-\mu_2$ & $H_0$: $\mu_1=\mu_2$\\
    \emph{(two-sample $t$-test)}                   &  in two groups & one categorical variable & &\\
    \hline
    $^*$Dependent means & quantitative response & two paired & CI for $\mu_d$ & $H_0$: $\mu_d=0$\\
    \emph{(paired $t$-test)} & measured on same observation & quantitative variables & & \\
    \hline
    $^*$ANOVA &  quantitative response & one quantitative variable and & Tukey pairwise & $H_0$: $\mu_1=\mu_2=\cdots=\mu_g$\\
                       & in $>2$ groups & one categorical variable & intervals &\\
    \hline
    $^*$Linear regression &  quantitative response and &  2 quantitative variables & CI for $\beta_1$ & $H_0$: $\beta_1=0$ \\
                       &  a quantitative explanatory & & &\\
    \hline
\end{tabular}}
\vskip10pt
\footnotesize{$^*$The starred methods can answer the question ``Is there an association?''  If we reject $H_0$, then we conclude that some sort of association is present in the two variables.}
\begin{flushright}
\hyperlink{topics}{\beamerreturnbutton{Back}}
\end{flushright}
\end{frame}

\begin{frame}[label=pvals]
\frametitle{Test statistics and Confidence Intervals}
\resizebox{1.0\textwidth}{!}{
{\renewcommand{\arraystretch}{2.2}
\begin{tabular}{|llll|}
    \hline
    \textbf{Method} & $\mathbf{H_0}$ & \textbf{Test Statistic} &\textbf{Confidence Interval} \\
    \hline
    Single proportion & $p = p_0$ & $\displaystyle z=\frac{\hat{p}-p_0}{\sqrt{\frac{p_0(1-p_0)}{n}}}$ & $\displaystyle \hat{p} \pm z^*\sqrt{\frac{\hat{p}(1-\hat{p})}{n}}$\\
    \hline
    %& & & \\
    Two proportions &  $p_1 = p_2$ & $\displaystyle z=\frac{(\hat{p}_1-\hat{p}_2)-0}{se}$ & $\displaystyle (\hat{p}_1-\hat{p}_2) \pm z^*\times se$\\
    \hline
     % & & & \\
    Chi-squared test & vars independent & $\displaystyle \chi^2=\sum\frac{(\mbox{observed}-\mbox{expected})^2}{\mbox{expected}}$   & N/A\\
    \hline
     %& & & \\
    Single mean &  $\mu = \mu_0$ & $\displaystyle t=\frac{\bar{x}-\mu_0}{s/\sqrt{n}}$ & $\displaystyle \bar{x} \pm t^*_{df=n-1} \times s/\sqrt{n}$ \\
    \hline
    % & & & \\
    Two means & $\mu_1 =\mu_2$ & $\displaystyle t=\frac{(\bar{x}_1 - \bar{x}_2)-0}{\sqrt{\frac{s_1^2}{n_1}+\frac{s_2^2}{n_2}}}$ &  $\displaystyle (\bar{x}_1-\bar{x}_2) \pm t^*_{df=given} \times \sqrt{\frac{s_1^2}{n_1}+\frac{s_2^2}{n_2}}$  \\
    \hline
    % & & & \\
    Dependent means & $\mu_d = 0$ & $\displaystyle t=\frac{\bar{x}_d-0}{s_d/\sqrt{n}}$  & $\displaystyle \bar{x_d} \pm t^*_{df=n-1} \times s_d/\sqrt{n}$  \\
    \hline
    % & & & \\
     ANOVA & $\mu_1=\mu_2=\cdots=\mu_g$ & N/A & Tukey \\
     \hline
     %& & & \\
     Linear regression  & $\beta_1 = 0$ & $\displaystyle t=\frac{\hat{\beta}_1-0}{se_{\hat{\beta}_1}}$ & $\displaystyle \hat{\beta}_1 \pm t^*_{df=n-2} \times se_{\hat{\beta}_1}$ \\
    \hline
\end{tabular}}}
\begin{itemize}
    \item
    \scriptsize{
    When performing a statistical analysis with data in R, R by default assumes the two-sided alternative hypotheses as presented above, and all $p$-values presented represent the final $p$-values (ie, no need to multiply by two).}   \hyperlink{topics}{\beamerreturnbutton{Back}}
\end{itemize}
\end{frame}




\end{document}
