

\PassOptionsToPackage{subsection=false}{beamerouterthememiniframes}
\PassOptionsToPackage{dvipsnames,table}{xcolor}
\documentclass[fleqn]{beamer}
\usepackage{graphicx}
\usepackage{multirow}
\usepackage{multicol}
\usepackage{amsmath,amsfonts,amsthm,amsopn}
\usepackage{color, colortbl}
\usepackage{subfig}
\usepackage{wrapfig}
\usepackage{fancybox}
\usepackage{tikz}
\usepackage{fancyhdr}
\usepackage{setspace}
\usepackage{xcolor}
\usepackage{movie15}
\usepackage{pifont}
\usepackage{soul}
\usepackage{booktabs}
\usepackage{fancyvrb,newverbs}
\fvset{fontsize=\footnotesize}
\RecustomVerbatimEnvironment{verbatim}{Verbatim}{}

%\usepackage{fancybox}

\usetheme{Szeged}
\usecolortheme{default}

%\definecolor{links}{HTML}{2A1B81}
%\definecolor{links}{blue!20}
\hypersetup{colorlinks,linkcolor=,urlcolor=blue!80}

\setbeamertemplate{blocks}[rounded]
\setbeamercolor{block title}{bg=blue!40,fg=black}
\setbeamercolor{block body}{bg=blue!10}

%\definecolor{myblue1}{blue!10}

%\colorlet{breaks}{myblue1}

\newenvironment<>{clicker}[1]{%
  \begin{actionenv}#2%
      \def\insertblocktitle{#1}%
      \par%
      \mode<presentation>{%
        \setbeamercolor{block title}{fg=white,bg=magenta}
       \setbeamercolor{block body}{fg=black,bg=magenta!10}
       \setbeamercolor{itemize item}{fg=magenta}
       \setbeamertemplate{itemize item}[triangle]
       \setbeamercolor{enumerate item}{fg=magenta}
     }%
      \usebeamertemplate{block begin}}
    {\par\usebeamertemplate{block end}\end{actionenv}}




\defbeamertemplate*{footline}{infolines theme}
{
  \leavevmode%
  \hbox{%
  \begin{beamercolorbox}[wd=.333333\paperwidth,ht=2.25ex,dp=1ex,left]{author in head/foot}%
    \usebeamerfont{author in head/foot}~~\insertshortinstitute: \insertshorttitle
  \end{beamercolorbox}%
  \begin{beamercolorbox}[wd=.67\paperwidth,ht=2.25ex,dp=1ex,right]{date in head/foot}%
    \usebeamerfont{date in head/foot}%\insertshortdate{}\hspace*{2em}
    \insertframenumber{} / \inserttotalframenumber\hspace*{2ex}
  \end{beamercolorbox}
  }%
  \vskip0pt%
}

\newcommand{\cmark}{\ding{51}}%
\newcommand{\xmark}{\ding{55}}%
\newcommand{\grp}{\textcolor{magenta}{Group Exercise}}
\newcommand{\bsans}[1]{\underline{\hspace{0.2in}\color{blue!80}{#1}\hspace{0.2in}}}
\newcommand{\bs}{\underline{\hspace{0.3in}}}


\definecolor{cverbbg}{gray}{0.93}
\newenvironment{cverbatim}
 {\SaveVerbatim{cverb}}
 {\endSaveVerbatim
  \flushleft\fboxrule=0pt\fboxsep=.5em
  \colorbox{cverbbg}{\BUseVerbatim{cverb}}%
  \endflushleft
}
\newenvironment{lcverbatim}
 {\SaveVerbatim{cverb}}
 {\endSaveVerbatim
  \flushleft\fboxrule=0pt\fboxsep=.5em
  \colorbox{cverbbg}{%
    \makebox[\dimexpr\linewidth-2\fboxsep][l]{\BUseVerbatim{cverb}}%
  }
  \endflushleft
}




\title[Unit 3 Deck 3]{Correlation and Simple Linear Regression}
\author[Pileggi]{Shannon Pileggi}

\institute[STAT 217]{STAT 217}

\date{}


\begin{document}

\begin{frame}
\titlepage
\end{frame}

\begin{frame}
\frametitle{OUTLINE\qquad\qquad\qquad} \tableofcontents[hideallsubsections]
\end{frame}


%===========================================================================================================================
\section[The Data]{The Data}
%===========================================================================================================================
%\subsection{}

\begin{frame}
\frametitle{Chocolate Consumption, Cognitive Function,
and Nobel Laureates by Franz H. Messerli, M.D.}
\framesubtitle{\emph{New England Journal of Medicine} 367(\textbf{16}), Oct 18, 2012}
\begin{itemize}
    \item
    chocolate contains flavanols, which are loosely linked to improved cognitive function
    \item
    Is there a relationship between country's level of chocolate consumption and its population's cognitive function?
    \item
    explanatory variable: per capita yearly chocolate consumption (in kg)
    \item
    response variable: total number of Nobel laureates per 10 million person (through 2011)
\end{itemize}
\end{frame}

\begin{frame}
\frametitle{Chocolate Consumption, Cognitive Function,
and Nobel Laureates by Franz H. Messerli, M.D.}
\framesubtitle{\emph{New England Journal of Medicine} 367(\textbf{16}), Oct 18, 2012}
\begin{center}
\includegraphics[width=0.7\textwidth]{Figures/chocolate.png}
\end{center}
\end{frame}


\begin{frame}
\frametitle{Discussion}
\begin{itemize}
    \item
    Is this study observational or experimental?
    \item[]
    \item
    What is the unit of observation?
    \item[]
    \item
    Describe the sample:
    \item[]
    \item
    Describe the population:
    \item[]
    \item
    What are other variables may affect the relationship between chocolate consumption and nobel laureates?
    \item[]
\end{itemize}
\end{frame}

\begin{frame}
\frametitle{Discussion, continued}
 Other oddities:
    \begin{itemize}
    \item
    Does it make sense to use nobel laureates as a measure of a country's cognitive function?
    \item
    Do we even know if the nobel laureates themselves ate chocolate?
    \item
    Were the variables measured on the same temporal scale?
    \item
    How are nobel prize winners identified with a specific country?
    \item[]
    \item[]
    \end{itemize}
  \href{http://www.stats.org/stories/2012/Cacao_or_Caca_oct16_12.html}{Stats.org}: Cacao or Caca? How the media bit into chocolate Nobel prize link
\end{frame}

\begin{frame}
\frametitle{In the media}
\begin{itemize}
    \item
    \href{http://healthland.time.com/2012/10/12/can-eating-chocolate-help-you-win-a-nobel-prize/}{Time}: Secret to Winning a Nobel Prize?  Eat more Chocolate
    \item
    \href{http://now.msn.com/chocolate-consumption-per-capita-linked-to-number-of-nobel-prize-wins}{MSN}: Want to win a Nobel Prize? Eat more chocolate
    \item
    \href{http://www.forbes.com/sites/larryhusten/2012/10/10/chocolate-and-nobel-prizes-linked-in-study/}{Forbes}: Chocolate and Nobel Prizes Linked in Study
    \item
    \href{http://www.usatoday.com/story/news/nation/2012/10/10/nobel-prizes-chocolate/1625403/}{USA Today}: Study links eating chocolate to winning Nobels
    \item
    \href{http://www.reuters.com/article/2012/10/10/us-eat-chocolate-win-the-nobel-prize-idUSBRE8991MS20121010}{Reuters}: Eat chocolate, win the Nobel Prize?
    \item
    \href{http://www.npr.org/blogs/thesalt/2012/10/12/162733830/the-secret-to-genius-it-might-be-more-chocolate}{NPR}: The Secret to Genius?  It Might Be More Chocolate
\end{itemize}
\end{frame}


\begin{frame}
\frametitle{From the author (Messerli)}
Reuters:
\footnotesize{
\begin{quote}
    ``I started plotting this in a hotel room in Kathmandu, because I had nothing else to do, and I could not believe my eyes,'' he told Reuters Health. All the countries lined up neatly on a graph, with higher chocolate intake tied to more laureates.  The link was so strong it made a joke out of a statistic that virtually all studies in medical journals hinge on - the so-called p-value.
\end{quote}}
NPR:
\footnotesize{
\begin{quote}
    ``I have published about 800 papers in peer-reviewed journals,'' he says, ``and every single one of them stands and falls with the p-value. And now here I find a p-value of 0.0001, and this is, to my way of thinking, a completely nonsensical relation. Unless you — or anybody else — can come up with an explanation. I've presented it to a few of my colleagues, and nobody has any thoughts.''
\end{quote}}
\end{frame}




\begin{frame}[fragile]
\frametitle{The Data}
First 10 observations:
\vskip10pt
\begin{lcverbatim}
     country nobel_rate chocolate   GDP_cap   totalpop literate
1  Australia      5.451       4.5 35052.512   22323900     99.9
2    Austria     24.332      10.2 36119.406    8423635     99.9
3    Belgium      8.622       4.4 33020.438   11047744     99.9
4     Brazil      0.050       2.9 10264.006  196935134     90.5
5     Canada      6.122       3.9 35738.703   34483975     99.8
6      China      0.060       0.7  7417.888 1344130000     95.4
7    Denmark     25.255       8.5 32601.660    5570572     99.9
8    Finland      7.600       7.3 32030.703    5388272     99.9
9     France      8.990       6.3 29963.223   65371613    100.0
10   Germany     12.668      11.6 34572.938   81797673     99.9
\end{lcverbatim}
\vskip5pt
\begin{clicker}{Do \texttt{nobel\textunderscore rate} and \texttt{chocolate} represent paired data?}
\begin{enumerate}
    \item
    yes
    \item
    no
\end{enumerate}
\end{clicker}
\end{frame}

%===========================================================================================================================
\section[Correlation]{Correlation}
%===========================================================================================================================
\begin{frame}
\tableofcontents[currentsection, hideallsubsections]
\end{frame}
%\subsection{}

\begin{frame}
\frametitle{Association between two quantitative variables}
\begin{columns}
\column{0.5\textwidth}
\includegraphics[width=0.9\textwidth]{Figures/literacy_lifeexp.pdf}\\
\begin{itemize}
    \item
    Positive association: as literacy rate tends to increase, life expectancy of females also tends to increase
\end{itemize}
\column{0.5\textwidth}
\includegraphics[width=0.9\textwidth]{Figures/fertility_lifeexp.pdf}\\
\begin{itemize}
    \item
    Negative association: as fertility rate tends to increase, life expectancy tends to decrease
    \item[]
\end{itemize}
\end{columns}
\end{frame}

\begin{frame}
\frametitle{Correlation}
The \textbf{correlation} \emph{r} measures the \textbf{strength} and \textbf{direction} of the \textbf{linear association} between two quantitative variables.  Correlation is always between between $-1$ and $+1$.
\begin{columns}
\column{0.5\textwidth}
\begin{center}
\includegraphics[width=0.9\textwidth]{Figures/literacy_lifeexp.pdf}\\
    $r=0.87$
\end{center}
\column{0.5\textwidth}
\begin{center}
\includegraphics[width=0.9\textwidth]{Figures/fertility_lifeexp.pdf}\\
    $r=-0.84$
\end{center}
\end{columns}
\end{frame}



\begin{frame}
\frametitle{Correlation examples}
\includegraphics[width=0.19\textwidth]{Figures/corr1.pdf}
\includegraphics[width=0.19\textwidth]{Figures/corr2.pdf}
\includegraphics[width=0.19\textwidth]{Figures/corr3.pdf}
\includegraphics[width=0.19\textwidth]{Figures/corr4.pdf}
\includegraphics[width=0.19\textwidth]{Figures/corr5.pdf}\\
\vskip10pt
Rules of thumb:
\begin{itemize}
    \item
    $0<|r|<0.3$ - weak correlation
    \item
    $0.3<|r|<0.7$ - moderate correlation
    \item
    $0.7<|r|<1.0$ - strong correlation
\end{itemize}
\end{frame}

\begin{frame}
\begin{clicker}{Which scenario illustrates the strongest linear association?}
\begin{enumerate}
    \item
    College GPA and high school GPA, $r=0.46$
    \item
    price and quality rating of bike helmets, $r=0.30$
    \item
    years of education and years in jail, $r=-0.53$
    \item
    New York City marathon, age and finish time, $r=0.04$
    \item
    college GPA and a measure of tendency to procrastinate, $r=-0.36$
\end{enumerate}
\end{clicker}
\end{frame}

\begin{frame}[fragile]
\frametitle{Correlation estimate}
\begin{lcverbatim}
> plot(NEJM$chocolate,NEJM$nobel_rate)
> cor(NEJM$chocolate,NEJM$nobel_rate)
[1] 0.8010949
\end{lcverbatim}
\includegraphics[width=0.5\textwidth]{Figures/scatter_choc_nobel.pdf}
\end{frame}

\begin{frame}
\frametitle{\grp}
\begin{columns}
\column{0.5\textwidth}
\includegraphics[width=1.0\textwidth]{Figures/nonlinear.pdf}\\
\column{0.5\textwidth}
\begin{clicker}{What do you think the correlation between $x$ and $y$ is?}
\begin{enumerate}
    \item
    $r=1$
    \item
    $r=0.5$
    \item
    $r=0$
    \item
    $r=-1$
\end{enumerate}
\end{clicker}
\end{columns}
\end{frame}


\begin{frame}
\frametitle{\grp}
\begin{columns}
\column{0.5\textwidth}
\includegraphics[width=1.0\textwidth]{Figures/scatter_totalpop_phone.pdf}\\
\column{0.6\textwidth}
\begin{clicker}{Which of the following best describes the relationship between total population and telephone lines?}
\begin{enumerate}
    \item
    there is a strong negative linear relationship
    \item
    there is a moderate negative linear relationship
    \item
    there is very little, if any, linear relationship
    \item
    there is a moderate positive linear relationship
    \item
    there is a strong positive linear relationship
\end{enumerate}
\end{clicker}
\end{columns}
\end{frame}



\begin{frame}
\frametitle{Chocolate Nobel Data Set Example}
\begin{columns}
\column{0.5\textwidth}
\includegraphics[width=1.0\textwidth]{Figures/scatter_totalpop_phone.pdf}\\
\begin{itemize}
    \item
    Whole data set: $r = -0.35$
\end{itemize}
\column{0.5\textwidth}
\includegraphics[width=1.0\textwidth]{Figures/scatter_totalpop_phone_new.pdf}\\
\begin{itemize}
    \item
    Excluding China: $r = -0.02$
\end{itemize}
\end{columns}
\end{frame}




\begin{frame}
\frametitle{\grp}
\begin{columns}
\column{0.5\textwidth}
\includegraphics[width=1.0\textwidth]{Figures/click_corr1.pdf}
\column{0.5\textwidth}
\begin{clicker}{What is your best guess of the correlation between $x$ and $y$?}
\begin{enumerate}
    \item
    $-1$
    \item
    $-0.7$
    \item
    $-0.2$
    \item
    0
    \item
    $0.2$
    \item
    $0.7$
    \item
    $1$
\end{enumerate}
\end{clicker}
\end{columns}
\end{frame}

%===========================================================================================================================
\section[SLR]{Simple Linear Regression}
%===========================================================================================================================
\begin{frame}
\tableofcontents[currentsection, hideallsubsections]
\end{frame}
%\subsection{}


\begin{frame}
\frametitle{The idea}
\begin{columns}
\column{0.30\textwidth}
\includegraphics[width=1.0\textwidth]{Figures/choc_m1.pdf}\\
\vskip10pt
Estimate the line of best fit:\\
$\displaystyle \hat{y}= b_0 +b_1x$
\vskip30pt
\column{0.70\textwidth}
Correlation allows us to:
\begin{enumerate}
    \item
    identify the direction of the association between $x$ and $y$
    \item
    quantify the strength of the association between $x$ and $y$
    \item[]
\end{enumerate}
A line of best fit additionally allows us to:
\begin{enumerate}
    \item
    make predictions of $y$ based on $x$
    \item
    further describe the relationship between $x$ and $y$ with the slope $b_1$
    %\item
    %use statistical inference about the slope to determine if there is an association between $x$ and $y$
\end{enumerate}
\end{columns}
\end{frame}

\begin{frame}[fragile]
\frametitle{Identifying the line}
\begin{lcverbatim}
> lm(NEJM$nobel_rate~NEJM$chocolate)

Call:
lm(formula = NEJM$nobel_rate ~ NEJM$chocolate)

Coefficients:
   (Intercept)  NEJM$chocolate
        -3.400           2.496
\end{lcverbatim}
\vskip10pt
\begin{tabular}{ll}
General form: & $\displaystyle \hat{y}= b_0 +b_1x$ \\
Here, this is:& $\displaystyle \hat{y}= b_0 +b_1\times chocolate$ \\
& \\
Estimated line: & $\displaystyle \hat{y}= -3.4 +2.5x$ \\
Here, this is: &  $\displaystyle \hat{y}= -3.4 + 2.5\times chocolate$
\end{tabular}
\end{frame}


\begin{frame}
\frametitle{The slope ($b_1$)}
\begin{columns}
\column{0.35\textwidth}
\includegraphics[width=1.0\textwidth]{Figures/choc_slope2.pdf}
\column{0.60\textwidth}
$\displaystyle \hat{y}= -3.4 + 2.5x$
\begin{itemize}
    \item
    the slope of this line is 2.5
    \item
    the slope is the amount that $\hat{y}$ changes when $x$ increases by one unit
    \item
    for each additional 1kg/yr/capita of chocolate that a country consumes, the rate of nobel laureates per 10 million persons increases by 2.5
    \item
    the sign of the slope (positive or negative) indicates the direction of the association
\end{itemize}
\end{columns}
\end{frame}


\begin{frame}
\frametitle{The $y$-intercept ($b_0$)}
\begin{columns}
\column{0.35\textwidth}
\includegraphics[width=1.2\textwidth]{Figures/choc_int.pdf}\\
\column{0.60\textwidth}
$\displaystyle \hat{y}= -3.4 + 2.5x$
\begin{itemize}
    \item
    the intercept of this line is -3.4
    \item
    the intercept is the predicted value of $y$ when $x=0$
    \item
    when a country consumes no chocolate, its predicted number of Nobel laureates is -3.4
    \item
    intercepts aren't always interpretable
\end{itemize}
\end{columns}
\end{frame}

\begin{frame}
Suppose we use literacy rate of a country (measured as a percent) to predict the life expectancy of that country (measured in years), and we estimate the regression line to be $\hat{y}=11.8+0.70x$.
\begin{clicker}{What is the interpretation of the slope?}
\begin{enumerate}
    \item
    For each one percent increase in literacy rate, life expectancy increases by 0.7 years.
    \item
    For each one year increase in life expectancy, literacy rate increases by 0.7 percentage points.
    \item
    For each one percent increase in literacy rate, life expectancy increases by 11.8 years.
    \item
    For each one year increase in life expectancy, literacy rate increases by 11.8 percentage points.
    \item
    The predicted life expectancy of a country with a literacy rate of 0 is 11.8 years.
    \item
    The predicted literacy rate of a country with a life expectancy of 0 is 11.8\%.
\end{enumerate}
\end{clicker}
\end{frame}

\begin{frame}
\frametitle{Prediction ($\hat{y}$)}
\begin{columns}
\column{0.35\textwidth}
\includegraphics[width=1.2\textwidth]{Figures/choc_pred.pdf}\\
\column{0.60\textwidth}
$\displaystyle \hat{y}= -3.4 + 2.5x$
\begin{itemize}
    \item
    when $x=7$, $\hat{y}= -3.4 + 2.5\times7=14.1$
    \item
    the predicted number of Nobel Laureates for a country that consumes 7 kg/yr/capita of chocolate is 14.1
\end{itemize}
\end{columns}
\end{frame}

\begin{frame}[fragile]
Using length of pregnancy in days (the gestation period) to predict babies' birth weight in pounds in the babies data set:
%\includegraphics[width=1.0\textwidth]{Figures/scatter_bwtlbs.pdf}
\begin{lcverbatim}
> cor(babies$bwt_lbs,babies$gestation)
0.42
> lm(bwt_lbs~gestation, data=babies)
Call:
lm(formula = bwt_lbs ~ gestation, data=babies)
Coefficients:
   (Intercept)        gestation
      -1.30920          0.03143
\end{lcverbatim}
\begin{clicker}{\small{What is the correct formula to predict the birth weight of a baby with a gestation period of 300 days?}}
\begin{enumerate}
    \small{
    \item
    $\hat{y} = 0.03 - 1.31 \times 300$
    \item
    $\hat{y} = 0.03 + 0.42 \times 300$
    \item
    $\hat{y} = -1.31 + 0.03 \times 300$
     \item
     $\hat{y} = -1.31 + 0.42 \times 300$}
\end{enumerate}
\end{clicker}
\end{frame}


\begin{frame}
\frametitle{Don't extrapolate!}
\begin{columns}
\column{0.35\textwidth}
\includegraphics[width=1.2\textwidth]{Figures/choc_extra.pdf}\\
\column{0.60\textwidth}
\begin{itemize}
    \item
    Extrapolation is using a regression line to predict $y$-values for $x$-values outside the observed range of the data.
    \item
    Extrapolation gets riskier the farther we move from the range of the given $x$-values.
    \item
    There is no guarantee that the relationship given by the regression equation holds outside the range of sampled $x$-values.
\end{itemize}
\end{columns}
\end{frame}

\begin{frame}
\frametitle{Cautions in Regression}
\begin{itemize}
    \item
    Don't extrapolate
    \item
    The analysis is not robust to outliers (can affect estimates of correlation, slope, and intercept)
    \item
    Correlation/association does not imply causation
    \item
    Other variables can influence the analysis through confounding and interaction - you can control for this in \emph{multivariable regression} where you use more than one variable in your model to predict $y$
\end{itemize}
\end{frame}

\begin{frame}
\begin{clicker}{When using a simple linear regression model, which of the following allows us to assess the \emph{association} between $x$ and $y$?}
\begin{enumerate}
    \item
    the intercept ($b_0$)
    \item
     the slope ($b_1$)
     \item
     both
     \item
     neither
\end{enumerate}
\end{clicker}
\end{frame}

\begin{frame}
Using MMSE score to predict whole brain volume in the ADNI data set: $r=0.19$, $\hat{y}=796239+8302\times MMSE$\\
\begin{columns}
\column{0.4\textwidth}
\includegraphics[width=1.0\textwidth]{Figures/scatter_ADNI.pdf}
\column{0.6\textwidth}
\begin{clicker}{What can we conclude about the strength of the association between MMSE and whole brain volume?}
\begin{enumerate}
    \item
    there is a strong association because the slope is large
    \item
    there is a strong association because the correlation is large
    \item
    there is a weak association because the slope  is small
    \item
    there is a weak association because the correlation is small
\end{enumerate}
\end{clicker}
\end{columns}
\end{frame}

%===========================================================================================================================
\section[Inference for SLR]{Inference for SLR}
%===========================================================================================================================
\begin{frame}
\tableofcontents[currentsection, hideallsubsections]
\end{frame}
%\subsection{}

\begin{frame}
\frametitle{The idea}
\begin{itemize}
\item
the \emph{slope} of the line allows us to assess whether or not there is an association between $x$ and $y$
\item
a slope of 0 indicates no association between $x$ and $y$; a non-zero slope indicates some sort of association between $x$ and $y$
\item
we are estimating a slope $b_1$ based on sample data, but we want to draw conclusions about the slope in the population ($\beta_1$)
\item
We can do this with:
\begin{enumerate}
    \item
    a hypothesis test of $H_0$: $\beta_1=0$ vs $H_a$: $\beta_1 \neq 0$
    \item
    a confidence interval for $\beta_1$
\end{enumerate}
\end{itemize}
\end{frame}

\begin{frame}
\frametitle{Conditions for inference in SLR}
\begin{enumerate}
    \item
    observations are independent
    \item
    linear relationship between $x$ and $y$
    \item
    \textcolor{gray}{``normality'' - not addressed in STAT 217 (requires residuals)}
    \item
    constant variability in $y$ about the regression line
\end{enumerate}
\end{frame}


\begin{frame}
\frametitle{Example Figures}
\begin{columns}
\column{0.35\textwidth}
\includegraphics[width=1.0\textwidth,clip,trim=0mm 0mm 90mm 0mm]{Figures/scattHeadLTotalLTube.pdf}\\
\vskip5pt
\includegraphics[width=1.0\textwidth,clip,trim=90mm 0mm 0mm 0mm]{Figures/scattHeadLTotalLTube.pdf}
\column{0.6\textwidth}
\begin{itemize}
    \item
    possum data
    \item
    the spread in $y$ (head length) is about the same for all values of $x$ (total length)
    \item
    this satisfies ``constant variability in $y$ about the regression line''
    \item[]
    \item
    car data
    \item
    the relationship between $x$ (weight) and $y$ (miles per gallon) follows a curve rather than a straight line
    \item
    this violates ``linear relationship between $x$ and $y$''
    \item[]
\end{itemize}
\end{columns}
\end{frame}

\begin{frame}
\begin{center}
Figure A \hspace{0.4in} Figure B  \hspace{0.4in} Figure C \hspace{0.4in} Figure D
\end{center}
\includegraphics[width=1.0\textwidth,clip,trim=0mm 25mm 0mm 0mm]{Figures/whatCanGoWrongWithLinearModel.pdf}\\
\vskip20pt
\begin{clicker}{Figure (A/B/C/D) violates the independence condition.\\
 Figure (A/B/C/D) violates the linear condition.\\
  Figure (A/B/C/D) violates the constant variability condition.}
\end{clicker}
\end{frame}




\begin{frame}
\frametitle{Assessing conditions}
\begin{columns}
\column{0.5\textwidth}
\begin{center}
\includegraphics[width=1.0\textwidth]{Figures/choc_m1.pdf}\\
\end{center}
\column{0.5\textwidth}
\begin{clicker}{Which condition may be violated?  Select the \emph{best} answer.}
\begin{enumerate}
    \item
    independence of observations
    \item
    linear relationship between $x$ and $y$
    \item
    constant variability in $y$ about the regression line
    \item
    no conditions are violated
\end{enumerate}
\end{clicker}
\end{columns}
\end{frame}

\begin{frame}
\frametitle{Assessing conditions, continued}
\begin{columns}
\column{0.5\textwidth}
\begin{center}
\includegraphics[width=1.0\textwidth]{Figures/choc_m2a.pdf}\\
\end{center}
\column{0.5\textwidth}
\begin{clicker}{Which condition may be violated?  Select the \emph{best} answer.}
\begin{enumerate}
    \item
    independence of observations
    \item
    linear relationship between $x$ and $y$
    \item
    constant variability in $y$ about the regression line
    \item
    no conditions are violated
\end{enumerate}
\end{clicker}
\end{columns}
\end{frame}

\begin{frame}[fragile]
\frametitle{R results}
\begin{lcverbatim}
> m1<-lm(NEJM$nobel_rate~NEJM$chocolate)
> summary(m1)

Call:
lm(formula = NEJM$nobel_rate ~ NEJM$chocolate)

Residuals:
    Min      1Q  Median      3Q     Max
-12.888  -2.953  -0.213   1.992  19.279

Coefficients:
               Estimate Std. Error t value Pr(>|t|)
(Intercept)      -3.400      2.699  -1.260    0.222
NEJM$chocolate    2.496      0.407   6.133 4.37e-06 ***
---
Signif. codes:  0 ‘***’ 0.001 ‘**’ 0.01 ‘*’ 0.05 ‘.’ 0.1 ‘ ’ 1

Residual standard error: 6.26 on 21 degrees of freedom
Multiple R-squared:  0.6418,	Adjusted R-squared:  0.6247
F-statistic: 37.62 on 1 and 21 DF,  p-value: 4.374e-06
\end{lcverbatim}
\end{frame}


\begin{frame}[fragile]
\frametitle{Focus on the coefficients box}
\begin{lcverbatim}
Coefficients:
               Estimate Std. Error t value Pr(>|t|)
(Intercept)      -3.400      2.699  -1.260    0.222
NEJM$chocolate    2.496      0.407   6.133 4.37e-06 ***
\end{lcverbatim}
On the \texttt{NEJM\$chocolate} line:
\begin{columns}
\column{0.1\textwidth}
\column{0.9\textwidth}
\begin{itemize}
    \item[Hypotheses:]
     $H_0$: $\beta_1=0$ vs $H_a$: $\beta_1 \neq 0$
    \item[Test statistic:]
     $\displaystyle t=\frac{\hat{\beta}_1-0}{se_{\hat{\beta}_1}}=\frac{2.496-0}{0.407}=6.133$
    \item[$p$-value:]
     two-tailed area from $t$ distribution with $df=n-2$; $p$-value$ = 0.00000437$
    \item[Conclusion:]
      At $\alpha=0.05$ reject $H_0$; we have evidence of a positive association between chocolate consumption and rate of nobel laureates.
\end{itemize}
\end{columns}
\end{frame}


\begin{frame}[fragile]
\frametitle{Confidence interval for the slope}
\begin{columns}
\column{0.6\textwidth}
\begin{lcverbatim}
> confint(m1)
                   2.5 %   97.5 %
(Intercept)    -9.013147 2.212415
NEJM$chocolate  1.649869 3.342646
\end{lcverbatim}
\column{0.5\textwidth}
\begin{eqnarray*}
\hat{\beta}_1 & \pm &  t^*_{df=n-2} \times se_{\hat{\beta}_1} \\
2.496  & \pm &  2.080 \times 0.407
 \end{eqnarray*}
\end{columns}
\begin{itemize}
    \item[]
    \item
    The 95\% CI for $\beta_1$ is (1.65, 3.34)
    \item
    Because this CI is entirely positive, there is a \emph{positive} association between rate of nobel laureates and chocolate consumption, such that as chocolate consumption increases so does the rate of nobel laureates.
    \item
    At the 95\% confidence level, for each 1kg/yr/capita increase in chocolate consumption the rate of nobel laureates per 10 million persons increases by as few as 1.65 or as many as 3.34
\end{itemize}
\end{frame}


\begin{frame}[fragile]
%\frametitle{\grp}
\begin{lcverbatim}
> cor(babies$bwt_lbs,babies$gestation)
0.42

Coefficients:
                  Estimate Std. Error t value Pr(>|t|)
(Intercept)      -1.309203   0.540712  -2.421   0.1560
babies$gestation  0.031433   0.001931  16.274   <2e-16 ***
\end{lcverbatim}
\begin{columns}
\column{0.3\textwidth}
\begin{center}
\includegraphics[width=1.0\textwidth]{Figures/scatter_bwtlbs.pdf}
\end{center}
\column{0.7\textwidth}
\begin{clicker}{Identify a number that represents \underline{\hspace{0.5in}} between birth weight and gestation days:}
\begin{enumerate}
    \item the magnitude of the association
    \item the strength of the evidence 
    \item the strength of the association
\end{enumerate}
\end{clicker}
\end{columns}
\end{frame}

\begin{frame}[fragile]
\begin{lcverbatim}
> cor(babies$bwt_lbs,babies$gestation)
0.42

Coefficients:
                  Estimate Std. Error t value Pr(>|t|)
(Intercept)      -1.309203   0.540712  -2.421   0.1560
babies$gestation  0.031433   0.001931  16.274   <2e-16 ***
\end{lcverbatim}
\begin{columns}
\column{0.3\textwidth}
\begin{center}
\includegraphics[width=1.0\textwidth]{Figures/scatter_bwtlbs.pdf}
\end{center}
\column{0.7\textwidth}
\begin{clicker}{\small{At $\alpha=0.05$, what can we conclude?  We \underline{(do/do not)} have evidence that $\beta_1$ differs from zero; so  we \underline{(do/do not)} have evidence of an association between gestation days and birth weight.}}
\begin{enumerate}
    \small{
    \item
    do; do
    \item
    do not; do not
    \item
    do; do not
    \item
    do not; do}
\end{enumerate}
\end{clicker}
\end{columns}
\end{frame}

\begin{frame}[fragile]
\begin{lcverbatim}
> cor(babies$bwt_lbs,babies$gestation)
0.42

Coefficients:
                  Estimate Std. Error t value Pr(>|t|)
(Intercept)      -1.309203   0.540712  -2.421   0.1560
babies$gestation  0.031433   0.001931  16.274   <2e-16 ***
\end{lcverbatim}
\begin{columns}
\column{0.3\textwidth}
\begin{center}
\includegraphics[width=1.0\textwidth]{Figures/scatter_bwtlbs.pdf}
\end{center}
\column{0.7\textwidth}
\begin{clicker}{\small{Is there a strong linear association between gestation days and birth weight?}}
\begin{enumerate}
    \small{
    \item
    Yes, because the $p$-value for $H_0$: $\beta_1=0$ is less than $0.05$.
    \item
    No. Although the $p$-value for $H_0$: $\beta_1=0$ is less than $0.05$, the estimated slope is small.
    \item
     No, because the $p$-value for $H_0$: $\beta_1=0$ is greater than $0.05$.
    \item
    Yes, the correlation is positive.
    \item
    No, the correlation isn't that big.}
\end{enumerate}
\end{clicker}
\end{columns}
\end{frame}


\end{document} 