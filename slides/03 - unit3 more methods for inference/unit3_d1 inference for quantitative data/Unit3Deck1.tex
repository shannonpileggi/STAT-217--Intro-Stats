

\PassOptionsToPackage{subsection=false}{beamerouterthememiniframes}
\PassOptionsToPackage{dvipsnames,table}{xcolor}
\documentclass[fleqn]{beamer}
\usepackage{graphicx}
\usepackage{multirow}
\usepackage{multicol}
\usepackage{amsmath,amsfonts,amsthm,amsopn}
\usepackage{color, colortbl}
\usepackage{subfig}
\usepackage{wrapfig}
\usepackage{fancybox}
\usepackage{tikz}
\usepackage{fancyhdr}
\usepackage{setspace}
\usepackage{xcolor}
\usepackage{movie15}
\usepackage{pifont}
\usepackage{soul}
\usepackage{booktabs}
\usepackage{fancyvrb,newverbs}
\fvset{fontsize=\footnotesize}
\RecustomVerbatimEnvironment{verbatim}{Verbatim}{}

%\usepackage{fancybox}

\usetheme{Szeged}
\usecolortheme{default}

%\definecolor{links}{HTML}{2A1B81}
%\definecolor{links}{blue!20}
\hypersetup{colorlinks,linkcolor=,urlcolor=blue!80}

\setbeamertemplate{blocks}[rounded]
\setbeamercolor{block title}{bg=blue!40,fg=black}
\setbeamercolor{block body}{bg=blue!10}

%\definecolor{myblue1}{blue!10}

%\colorlet{breaks}{myblue1}

\newenvironment<>{clicker}[1]{%
  \begin{actionenv}#2%
      \def\insertblocktitle{#1}%
      \par%
      \mode<presentation>{%
        \setbeamercolor{block title}{fg=white,bg=magenta}
       \setbeamercolor{block body}{fg=black,bg=magenta!10}
       \setbeamercolor{itemize item}{fg=magenta}
       \setbeamertemplate{itemize item}[triangle]
       \setbeamercolor{enumerate item}{fg=magenta}
     }%
      \usebeamertemplate{block begin}}
    {\par\usebeamertemplate{block end}\end{actionenv}}




\defbeamertemplate*{footline}{infolines theme}
{
  \leavevmode%
  \hbox{%
  \begin{beamercolorbox}[wd=.333333\paperwidth,ht=2.25ex,dp=1ex,left]{author in head/foot}%
    \usebeamerfont{author in head/foot}~~\insertshortinstitute: \insertshorttitle
  \end{beamercolorbox}%
  \begin{beamercolorbox}[wd=.67\paperwidth,ht=2.25ex,dp=1ex,right]{date in head/foot}%
    \usebeamerfont{date in head/foot}%\insertshortdate{}\hspace*{2em}
    \insertframenumber{} / \inserttotalframenumber\hspace*{2ex}
  \end{beamercolorbox}
  }%
  \vskip0pt%
}

\newcommand{\cmark}{\ding{51}}%
\newcommand{\xmark}{\ding{55}}%
\newcommand{\grp}{\textcolor{magenta}{Group Exercise}}
\newcommand{\bsans}[1]{\underline{\hspace{0.2in}\color{blue!80}{#1}\hspace{0.2in}}}
\newcommand{\bs}{\underline{\hspace{0.3in}}}


\definecolor{cverbbg}{gray}{0.93}
\newenvironment{cverbatim}
 {\SaveVerbatim{cverb}}
 {\endSaveVerbatim
  \flushleft\fboxrule=0pt\fboxsep=.5em
  \colorbox{cverbbg}{\BUseVerbatim{cverb}}%
  \endflushleft
}
\newenvironment{lcverbatim}
 {\SaveVerbatim{cverb}}
 {\endSaveVerbatim
  \flushleft\fboxrule=0pt\fboxsep=.5em
  \colorbox{cverbbg}{%
    \makebox[\dimexpr\linewidth-2\fboxsep][l]{\BUseVerbatim{cverb}}%
  }
  \endflushleft
}




\title[Unit 3 Deck 1]{Inference for Quantitative Data}
\author[Pileggi]{Shannon Pileggi}

\institute[STAT 217]{STAT 217}

\date{}


\begin{document}

\begin{frame}
\titlepage
\end{frame}

\begin{frame}
\frametitle{OUTLINE\qquad\qquad\qquad} \tableofcontents[hideallsubsections]
\end{frame}


%===========================================================================================================================
\section[The Data]{The Data}
%===========================================================================================================================
%\subsection{}
\begin{frame}
\frametitle{Beat the Blues}
\begin{itemize}
    \item
    enrolled patients with depression/anxiety
    \item
    randomly assigned them to Treatment as Usual (TAU) or BtheB, a new treatment delivery therapy via computers
    \item
    measured depression via Beck Depression Inventory (BDI) at baseline (pre-treatment), and 2, 4, 6, and 8 month follow up
    \item
    BDI scores range from 0 to 63 with higher scores indicating more severe depression
\end{itemize}
\end{frame}

\begin{frame}
\begin{clicker}{Is this study observational or experimental?}
\begin{enumerate}
    \item
    observational
    \item
    experimental
\end{enumerate}
\end{clicker}
\begin{enumerate}
    \item[]
    \item[]
    \item
    Describe the \underline{sample}:
    \item[]
    \item[]
    \item
    Describe the \underline{population}:
    \item[]
    \item[]
\end{enumerate}
\end{frame}

\begin{frame}[fragile]
\frametitle{The first 6 observations in the data set}
\begin{lcverbatim}
> head(BtheB)
  drug length treatment bdi.pre bdi.2m bdi.4m bdi.6m bdi.8m
1   No    >6m       TAU      29      2      2     NA     NA
2  Yes    >6m     BtheB      32     16     24     17     20
3  Yes    <6m       TAU      25     20     NA     NA     NA
4   No    >6m     BtheB      21     17     16     10      9
5  Yes    >6m     BtheB      26     23     NA     NA     NA
6  Yes    <6m     BtheB       7      0      0      0      0
\end{lcverbatim}
\end{frame}


%===========================================================================================================================
\section[Paired t-test]{Paired t-test}
%===========================================================================================================================
\begin{frame}
\tableofcontents[currentsection, hideallsubsections]
\end{frame}
%\subsection{}

\begin{frame}[fragile]
\underline{The research question}: Do depression levels change from baseline (pre-treatment) to 2 months follow-up post treatment? Ie, regardless of treatment group, does participating in a clinical trial have an effect on depression levels?
\vskip20pt
\begin{columns}
\column{0.3\textwidth}
\underline{First six observations:}
\begin{lcverbatim}
  bdi.pre bdi.2m
1      29      2
2      32     16
3      25     20
4      21     17
5      26     23
6       7      0
\end{lcverbatim}
\column{0.7\textwidth}
\begin{enumerate}
\item
What types of variables do we have?
\item[]
\item
Are they ``paired''?
\item[]
\item
How many groups are we studying?
\item[]
\item
How can we approach the problem?
\item[]
\end{enumerate}
\end{columns}
\end{frame}

\begin{frame}
%\frametitle{Clicker}
\begin{clicker}{Which scenarios represent paired measurements (or \emph{dependent} samples)?  Mark \underline{all} that apply.}
\begin{enumerate}
    \item
    In order to study the efficacy of a new sunscreen, 50 volunteers put a standard sunscreen on their left arm and the new sunscreen on their right arm.  After 3 hours, degree of redness was assessed for each arm.
    \item
    Newer baseball stadiums are thought to attract more fans.  Attendance records in the old stadium was compared to attendance records in the new stadium for 10 teams.
    \item
    The General Social Survey (GSS) compared the number of hours worked per week among college graduates and non-college graduates.
\end{enumerate}
\end{clicker}
\end{frame}

\begin{frame}[fragile]
\underline{Parameter of interest}: $\mu_d=\mu_{pre}-\mu_{2m}$ = population mean difference in depression scores between baseline and two month follow-up
\vskip20pt
\begin{columns}
\column{0.40\textwidth}
\begin{lcverbatim}
> BtheB$diff2m<-
  BtheB$bdi.pre-BtheB$bdi.2m

  bdi.pre bdi.2m diff2m
1      29      2     27
2      32     16     16
3      25     20      5
4      21     17      4
5      26     23      3
6       7      0      7
\end{lcverbatim}
\column{0.60\textwidth}
Three possible scenarios:
\begin{itemize}
\item
No change:
\item[]
\item[]
\item
Improvement:
\item[]
\item[]
\item
Worsening:
\item[]
\item[]
\end{itemize}
\end{columns}
\end{frame}

\begin{frame}
\frametitle{Answering the research question}
We can answer the research question of interest with:
\begin{itemize}
    \item
    A confidence interval for $\mu_d$
    \item
    A hypothesis test of $H_0$: $\mu_d=0$
    \item[]
    (this is the paired t-test)
    \item
    This is essentially the same inference about a mean that we have already learned
\end{itemize}
\vskip10pt
Conditions required for \emph{both} the CI and HT:
\begin{enumerate}
    \item
    independent observations
    \item
    normal underlying population distribution OR $n \geq 30$
\end{enumerate}
\end{frame}

\begin{frame}
\frametitle{Checking condition 1: independent observations}
\vskip5pt
\includegraphics[width=0.80\textwidth]{Figures/pairedconditions.png}\\
\vskip10pt
The difference in depression scores between baseline and two month follow-up for person 1 is not related to the difference in depression scores between baseline and two month follow-up for person 2 (and for all other subjects in the study).  This condition regarding independent observations is satisfied.
\end{frame}

\begin{frame}[fragile]
\frametitle{Checking condition 2}
The differences in depression scores has a slight right skew; however, the sample size is $n=97>30$ so this condition is satisfied.
\vskip10pt
\begin{lcverbatim}
> favstats(BtheB$diff2m)
 min Q1 median Q3 max     mean       sd  n missing
 -17  0      5 11  41 6.237113 9.474498 97       3
\end{lcverbatim}
\vskip10pt
\begin{columns}
\column{0.7\textwidth}
Summary data needed for both CI and HT:
\begin{itemize}
    \item[]
    $n=$
    \item[]
    $\bar{x}_d=$
    \item[]
    $s_d=$
\end{itemize}
\column{0.3\textwidth}
\includegraphics[width=1.0\textwidth]{Figures/histdiff2m.pdf}
\end{columns}
\end{frame}



\begin{frame}
\frametitle{95\% confidence interval for $\mu_d$}
\begin{center}
$\bar{x}_d\pm t^* \times \frac{s_d}{\sqrt{n}}$
\end{center}
\begin{columns}
\column{0.6\textwidth}
based on a 95\% conf. level\\
and $df=n-1=97-1=96$\\
$t^*=1.98$ (from R) \\
%\pause
\vskip10pt
%\begin{center}
$se=s_d/\sqrt{n}=9.47/\sqrt{97}=0.96$
%\end{center}
%\pause
\vskip10pt
%\begin{center}
$6.23 \pm 1.98 \times 0.96$\\
%\end{center}
\vskip10pt
%\pause
%\begin{center}
$6.23 \pm 1.90$\\
%\end{center}
%\pause
\vskip10pt
%\begin{center}
95\% CI for $\mu_d$: (4.33,8.13)
%\end{center}
\column{0.4\textwidth}
%\pause
\begin{clicker}{What is the margin of error for this confidence interval?}
\begin{enumerate}
    \item
    0.05
    \item
    0.96
    \item
    1.90
    \item
    1.98
    \item
    3.80
\end{enumerate}
\end{clicker}
\end{columns}
\end{frame}

\begin{frame}
\frametitle{Interpretation}
\begin{itemize}
    \item
    Literal interpretation of the interval:
    \item[]
    \item[]
    \item
    Does it include zero?
    \item[]
    \item[]
    \item
    What does that mean?
    \item[]
    \item[]
    \item
    Answer the research question:
    \item[]
    \item[]
\end{itemize}
\end{frame}

\begin{frame}
\begin{clicker}{The 95\% CI for $\mu_d$ is (4.33,8.33).  Which of the following would be a \emph{plausible} $p$-value when testing the hypotheses $H_0$: $\mu_d=0$ vs $H_a$: $\mu_d\neq 0$?}
\begin{enumerate}
    \item
    0.03
    \item
    0.23
    \item
    0.53
    \item
    0.73
    \item
    not enough information to determine
\end{enumerate}
\end{clicker}
\end{frame}



\begin{frame}
\frametitle{Paired t-test}
$H_0$: $\mu_d=0$ vs $H_a$: $\mu_d \neq 0$\\
\vskip20pt
$\displaystyle t=\frac{\bar{x}_d-\mu_0}{s_d/\sqrt{n}} =\frac{6.23-0}{9.47/\sqrt{97}}=6.47$
\vskip20pt
This $t$ test statistic follows a $t$ distribution with 96 ($n-1$) degrees of freedom.  The $p$-value is found by a two-tailed area under the $t$ distribution.
\end{frame}

\begin{frame}[fragile]
\frametitle{Results in R}
\begin{lcverbatim}
#These two commands give equivalent results.  Why?

> t.test(BtheB$diff2m,mu=0)

> t.test(BtheB$bdi.pre,BtheB$bdi.2m,paired=TRUE)


	One Sample t-test

data:  BtheB$diff2m
t = 6.4836, df = 96, p-value = 3.869e-09
alternative hypothesis: true mean is not equal to 0
95 percent confidence interval:
 4.327579 8.146647
sample estimates:
mean of x
 6.237113
\end{lcverbatim}
\end{frame}


\begin{frame}
\frametitle{Interpret the p-value}
\includegraphics[width=0.4\textwidth]{Figures/t96mod.png}\\
If there really was no difference in average depression scores between baseline and 2 months (ie, if $H_0$ true), then the probability that we would observe a test statistic less than -6.48 or greater than 6.48 is really small ($3.869 \times 10^{-9}$).  This presents evidence against $H_0$.
\end{frame}

\begin{frame}
\frametitle{Conclusion in context}
\begin{enumerate}
    \item \textbf{Decision about $H_0$:}
    \item[]
    \item[]
    \item \textbf{Statement about the parameter tested in context of the research question:}
    \item[]
    \item[]
    \item[]
    \item \textbf{Provide a deeper connection of how this relates to the research question:}
    \item[]
    \item[]
    \item[]
\end{enumerate}
\end{frame}

\begin{frame}
\begin{clicker}
{What type of error could we have committed here?}
\begin{enumerate}
   \item
   a Type I error
   \item
   a Type II error
   \item
   either
   \item
   neither
\end{enumerate}
\end{clicker}
\end{frame}

\begin{frame}[fragile]
\frametitle{Meaningful?}
\begin{lcverbatim}
	One Sample t-test

data:  BtheB$diff2m
t = 6.4836, df = 96, p-value = 3.869e-09
alternative hypothesis: true mean is not equal to 0
95 percent confidence interval:
 4.327579 8.146647
sample estimates:
mean of x
 6.237113
\end{lcverbatim}
The results are statistically signficant.  But is it meaningful?
\vskip200pt
\end{frame}

\begin{frame}
Suppose I am dealing with paired data from a randomized clinical trial regarding weight before and after a treatment intending to assist in weight loss.  $\mu_d$ represents the average of post-treatment weight minus the pre-treatment weight.  A 95\% CI for $\mu_d$ is (0.5, 6.7).
\begin{clicker}{What can we infer from this CI?}
\begin{enumerate}
    \footnotesize{
    \item
    the treatment \underline{is not effective} because the CI includes 1, indicating that there is no difference in weight loss before and after treatment
    \item
    the treatment \underline{is effective} for weight loss because the post-treatment weights are on average less than the pre-treatment weights
    \item
    the treatment \underline{is not effective} for weight loss because treatment is causing patients to gain weight, rather than lose weight, on average
    \item
    there is not enough information to determine whether or not the treatment is effective}
\end{enumerate}
\end{clicker}
\end{frame}





%===========================================================================================================================
\section[Two sample t-test]{Two sample t-test}
%===========================================================================================================================
\begin{frame}
\tableofcontents[currentsection, hideallsubsections]
\end{frame}
%\subsection{}

\begin{frame}[fragile]
\underline{The research question}: Do depression levels differ at baseline (\texttt{bdi.pre}) between patients who were and were not already on antidepressant drugs (\texttt{drugs})?  OR: Is there an association between baseline depression scores and whether or not patients take antidepressant drugs?
\vskip10pt
\begin{columns}
\column{0.3\textwidth}
\underline{First six observations:}
\begin{lcverbatim}
  drug bdi.pre
1   No      29
2  Yes      32
3  Yes      25
4   No      21
5  Yes      26
6  Yes       7
\end{lcverbatim}
\column{0.7\textwidth}
\begin{enumerate}
\item
What types of variables do we have?
\item[]
\item
Are they ``paired''?
\item[]
\item
How many groups are we studying?
\item[]
\item
How can we approach the problem?
\item[]
\end{enumerate}
\end{columns}
\end{frame}




\begin{frame}
\begin{clicker}{Which figure would be appropriate to begin to visually assess if there is an association between depression scores and whether or not patients take antidepressant drugs?}
\begin{enumerate}
    \item
    dot plot
    \item
    histogram
    \item
    single boxplot
    \item
    side by side boxplot
    \item
    barplot
\end{enumerate}
\end{clicker}
\end{frame}

\begin{frame}
\frametitle{Is there an association?}
\begin{center}
\includegraphics[width=0.7\textwidth]{Figures/bpbdipredrug.pdf}
\end{center}
\end{frame}


\begin{frame}
\begin{clicker}{Which of these scenarios represent a two sample comparison?}
\begin{enumerate}
    \item
    We take a random sample of 100 Cal Poly students and test if the average GPA differs from 3.3.
    \item
    We take a random sample of 100 Cal Poly students and test if the average GPA differs between males and females.
    \item
    We take a random sample of 100 Cal Poly students and test if the average GPA changes before and after attending a workshop on study habits.
    \item
    We take a random sample of 100 Cal Poly students and test if the average GPA differs by political affiliation (democrat, republican, independent).
\end{enumerate}
\end{clicker}
\end{frame}



\begin{frame}[fragile]
\underline{Parameters of interest}: \\
$\mu_{yes}= $population mean baseline BDI score among patients who are on antidepressants \\
$\mu_{no}= $population mean baseline BDI score among patients who are not on antidepressants \\
\vskip20pt
\begin{columns}
\column{0.40\textwidth}
\underline{First six observations:}
\begin{lcverbatim}
  drug bdi.pre
1   No      29
2  Yes      32
3  Yes      25
4   No      21
5  Yes      26
6  Yes       7
\end{lcverbatim}
\column{0.60\textwidth}
Three possible scenarios:
\begin{itemize}
\item
No difference:
\item[]
\item[]
\item
Difference:
\item[]
\item[]
\item
Difference:
\item[]
\item[]
\end{itemize}
\end{columns}
\end{frame}

\begin{frame}
\frametitle{Answering the research question}
We can answer the research question of interest with:
\begin{itemize}
    \item
    A confidence interval for $\mu_{yes}-\mu_{no}$
    \item
    A hypothesis test of $H_0$: $\mu_{yes}=\mu_{no}$
    \item[]
    (this is the two sample t-test)
\end{itemize}
\vskip10pt
Conditions required for \emph{both} the CI and HT:
\begin{enumerate}
    \item
    independent observations (in each of the two groups)
    \item
    normal underlying population distribution OR
    \item[]
    $n \geq 30$ in \emph{each} group
\end{enumerate}
\end{frame}



\begin{frame}
\frametitle{Possible outcomes}
\begin{columns}
\column{0.5\textwidth}
If we fail to reject $H_0$: $\mu_1=\mu_2$ or if the 95\% CI for $\mu_1-\mu_2$ includes 0, then
\begin{itemize}
    \item
   What is the relationship between the population mean depression scores?
   \item[]
   \item[]
    \item
    Is there evidence of an association between pre-treatment BDI score and antidepressant drug use?
    \item[]
    \item[]
\end{itemize}
\column{0.5\textwidth}
If we reject $H_0$: $\mu_1=\mu_2$ or if the 95\% CI for $\mu_1-\mu_2$ does not include 0, then
\begin{itemize}
    \item
   What is the relationship between the population mean depression scores?
   \item[]
   \item[]
    \item
    Is there evidence of an association between pre-treatment BDI score and antidepressant drug use?
    \item[]
    \item[]
\end{itemize}
\end{columns}
\end{frame}

\begin{frame}[fragile]
\frametitle{The data}
\begin{lcverbatim}
> favstats(BtheB$bdi.pre~BtheB$drug)
  BtheB$drug min Q1 median    Q3 max     mean        sd  n missing
         No   7 14   20.5 27.50  40 21.55357  8.974549 56       0
        Yes   2 17   24.0 34.25  49 25.59091 12.577792 44       0
\end{lcverbatim}
\vskip20pt
Summary data needed for both CI and HT:
\begin{columns}
\column{0.3\textwidth}
\begin{itemize}
    \item[]
    \underline{Yes group}
    \item[]
    $n_1=$
    \item[]
    $\bar{x}_1=$
    \item[]
    $s_1=$
\end{itemize}
\column{0.3\textwidth}
\begin{itemize}
    \item[]
    \underline{No group}
    \item[]
    $n_2=$
    \item[]
    $\bar{x}_2=$
    \item[]
    $s_2=$
\end{itemize}
\column{0.3\textwidth}
\includegraphics[width=1.0\textwidth]{Figures/histbdipre.pdf}
\end{columns}
\end{frame}



\begin{frame}
\begin{clicker}{Which of the following data would provide the \emph{most} evidence against $H_0$: $\mu_{1}=\mu_{2}$?}
\begin{enumerate}
    \item
    $\bar{x}_1=25$, $\bar{x}_2=19$
    \item
    $\bar{x}_1=25$, $\bar{x}_2=21$
    \item
    $\bar{x}_1=25$, $\bar{x}_2=23$
    \item
    $\bar{x}_1=25$, $\bar{x}_2=27$
    \item
    $\bar{x}_1=25$, $\bar{x}_2=29$
\end{enumerate}
\end{clicker}
\end{frame}


\begin{frame}
\frametitle{95\% confidence interval for $\mu_{1} - \mu_{2}$}
\begin{center}
$\displaystyle (\bar{x}_1 - \bar{x}_2) \pm t^*\times se$ where $\displaystyle se =\sqrt{\frac{s_1^2}{n_1}+\frac{s_2^2}{n_2}}$
\end{center}
\vskip90pt
\begin{itemize}
\item
The df for $t^*$ is a complicated formula that you do not need to know.  The $t^*$ for a 95\% CI with $df=74.9$ is 1.99.
\item
We are calculating this CI under the assumption of \emph{unequal variances}.
\end{itemize}
\end{frame}

\begin{frame}
\frametitle{Hypothesis test of $H_0$: $\mu_{1}=\mu_{2}$}
\begin{center}
$\displaystyle t=\frac{(\bar{x}_1 - \bar{x}_2)-0}{se}$ where $\displaystyle se=\sqrt{\frac{s_1^2}{n_1}+\frac{s_2^2}{n_2}}$\\
\end{center}
\vskip90pt
We are calculating this test statistic under the assumption of \emph{unequal variances}.
\end{frame}


\begin{frame}
\frametitle{Interpretation}
\begin{itemize}
    \item
    Literal interpretation of the interval:
    \item[]
    \item[]
    \item
    Does it include zero?
    \item[]
    \item[]
    \item
    What does that mean?
    \item[]
    \item[]
    \item
    Answer the research question:
    \item[]
    \item[]
\end{itemize}
\end{frame}


\begin{frame}[fragile]
\frametitle{Two sample t-test assuming unequal variances}
\begin{center}
$H_0$: $\mu_1=\mu_2$ vs $H_a$: $\mu_1\neq\mu_2$
\end{center}
\begin{lcverbatim}
> t.test(BtheB$bdi.pre~BtheB$drug)

	Welch Two Sample t-test

data:  BtheB$bdi.pre by BtheB$drug
t = -1.7995, df = 74.911, p-value = 0.07597
alternative hypothesis: true difference in means is not equal to 0
95 percent confidence interval:
 -8.5069019  0.4322266
sample estimates:
 mean in group No mean in group Yes
         21.55357          25.59091
\end{lcverbatim}
\vskip5pt
\small{$^*$ Note the order of subtraction in the R output:  R analyzes this as $\mu_{no} - \mu_{yes}$.  You know this because the sample mean for group \texttt{No} is presented first, and the sample mean for group \texttt{Yes} is presented second.}
\end{frame}


\begin{frame}
\frametitle{Interpret the p-value}
\includegraphics[width=0.4\textwidth]{Figures/t96V2mod.png}\\
If average depression scores among patients who do and do not take anti-depressents really is the same (ie, if $H_0$ true), then the probability that we would observe a test statistic less than -1.79 or greater than 1.79 is 0.075.  This probability isn't that small, and does not present evidence against $H_0$.
\end{frame}

\begin{frame}
\frametitle{Conclusion in context}
\begin{enumerate}
    \item \textbf{Decision about $H_0$:}
    \item[]
    \item[]
    \item \textbf{Statement about the parameter tested in context of the research question:}
    \item[]
    \item[]
    \item[]
    \item \textbf{Provide a deeper connection of how this relates to the research question:}
    \item[]
    \item[]
    \item[]
\end{enumerate}
\end{frame}


\begin{frame}
\begin{clicker}{Which of the following are \underline{true}?  Mark \underline{all} that apply.}
\begin{enumerate}
    \item
    We could have committed a Type I error.
    \item
    We could have committed a Type II error.
    \item
    There is evidence of an association between antidepressant drug use and depression scores.
    \item
    There is no evidence of an association between antidepressant drug use and depression scores.
    \item
    We have evidence that using antidepressant drugs causes people to have higher depression scores.
\end{enumerate}
\end{clicker}
\end{frame}



%===========================================================================================================================
\section[ANOVA]{ANOVA}
%===========================================================================================================================
\begin{frame}
\tableofcontents[currentsection, hideallsubsections]
\end{frame}
%\subsection{}


\begin{frame}[fragile]
\frametitle{A new variable}
Suppose we create a variable defining patient history before they enrolled in the clinical trial.  The variable is defined as:
\begin{enumerate}
    \item
    \texttt{history} $= 1$: \texttt{drug} $=$ no; \texttt{length} $<$ 6m
    \item
    \texttt{history} $= 2$: \texttt{drug} $=$ no; \texttt{length} $>$ 6m
    \item
    \texttt{history} $= 3$: \texttt{drug} $=$ yes; \texttt{length} $<$ 6m
    \item
    \texttt{history} $= 4$: \texttt{drug} $=$ yes; \texttt{length} $>$ 6m
\end{enumerate}
\begin{columns}
\column{0.5\textwidth}
\begin{lcverbatim}
  drug length bdi.pre history
1   No    >6m      29       2
2  Yes    >6m      32       4
3  Yes    <6m      25       3
4   No    >6m      21       2
5  Yes    >6m      26       4
6  Yes    <6m       7       3
\end{lcverbatim}
\column{0.5\textwidth}
\begin{clicker}{The \texttt{history} variable is...}
\begin{enumerate}
    \item
    quantitative
    \item
    categorical
\end{enumerate}
\end{clicker}
\end{columns}
\end{frame}



\begin{frame}[fragile]
\underline{The research question}: Do patients' baseline depression levels differ by patient history? OR Is there an association between baseline depression score and patient history?
\vskip20pt
\begin{columns}
\column{0.5\textwidth}
\underline{First six observations:}
\begin{lcverbatim}
  drug length bdi.pre history
1   No    >6m      29       2
2  Yes    >6m      32       4
3  Yes    <6m      25       3
4   No    >6m      21       2
5  Yes    >6m      26       4
6  Yes    <6m       7       3
\end{lcverbatim}
\column{0.6\textwidth}
\begin{enumerate}
\item
What types of variables do we have?
\item[]
\item
Are they ``paired''?
\item[]
\item
How many groups are we studying?
\item[]
\item
How can we approach the problem?
\item[]
\end{enumerate}
\end{columns}
\end{frame}

\begin{frame}
\frametitle{Is there an association?}
\begin{center}
\includegraphics[width=0.7\textwidth]{Figures/bpbdihistory.pdf}
\end{center}
\end{frame}

\begin{frame}
\frametitle{Motivation}
\begin{itemize}
    \item
    We have already learned a hypothesis test for comparing two population means (two-sample $t$-test)
    \item[]
    $H_0: \mu_1=\mu_2$
    \item[]
    $H_a: \mu_1 \neq \mu_2$
    \item[]
    \item
    When we are interested in comparing $>$2 groups (and hence, $>$2 means) we can use ANOVA (analysis of variance)
    \item[]
    $H_0: \mu_1=\mu_2=\mu_3=\ldots \mu_g$ for $g$ groups
    \item[]
    $H_a$: at least one mean is different than the others
\end{itemize}
\end{frame}



\begin{frame}
\begin{center}
\includegraphics[width=0.9\textwidth]{Figures/mineboxplot.png}
\end{center}
\begin{clicker}{Which plots show groups with means that are \underline{most} and \underline{least} likely to be significantly different from each other?}
\begin{enumerate}
    \item
    most I; least II
    \item
    most II; least III
    \item
    most I; least III
    \item
    most III; least II
    \item
    most II; least I
\end{enumerate}
\end{clicker}
\end{frame}

\begin{frame}
\frametitle{Conclusion}
$H_0: \mu_1=\mu_2=\mu_3=\ldots \mu_g$ vs \\
$H_a$: at least one mean different than the others
\vskip15pt
\begin{columns}
\column{0.5\textwidth}
\underline{If $p$-value $>\alpha$}
\begin{itemize}
    \item
    fail to reject $H_0$
    \item
    there is not sufficient evidence to suggest that population means differ significantly
    \item
    there is no evidence of an association between the two variables
    \item[]
    \item[]
\end{itemize}
\column{0.5\textwidth}
\underline{If $p$-value $\leq\alpha$}
\begin{itemize}
    \item
    reject $H_0$
    \item
    there is evidence that at least one population means differs from the others
    \item
    there is evidence of an association between the two variables
    \item
    cannot determine which population means differ (yet)
\end{itemize}
\end{columns}
\end{frame}

\begin{frame}
\frametitle{Conditions}
\begin{enumerate}
    \item
    observations are independent (in each of the $g$ groups)
    \item[]
    \textcolor{OrangeRed}{This means that the baseline depression scores are independent from patient to patient within each of the four patient history groups.}
    \item
    normal underlying population distribution OR $n \geq 30$ in each group
    \item[]
    \textcolor{OrangeRed}{This means that the baseline depression scores are approximately normally distributed OR $n \geq 30$ in each of the four patient history groups.}
    \item
    each group has (about) the same variability
    \item[]
    \textcolor{OrangeRed}{This means that the standard deviation of the baseline depression scores is about the same in each of the four patient history groups.}
\end{enumerate}
\end{frame}

\begin{frame}[fragile]
\frametitle{Checking conditions}
\begin{lcverbatim}
> favstats(BtheB$bdi.pre~BtheB$history)
history min    Q1 median    Q3 max     mean        sd  n missing
      1   9 13.75   19.5 27.00  40 20.87500  8.931198 24       0
      2   7 14.75   21.0 29.25  40 22.06250  9.115522 32       0
      3   2 11.00   21.0 33.00  49 22.12000 13.185598 25       0
      4  14 22.50   30.0 37.00  47 30.15789 10.361591 19       0
\end{lcverbatim}
\vskip10pt
\begin{center}
\includegraphics[width=0.35\textwidth]{Figures/histbdipre.pdf}
\end{center}
\end{frame}

\begin{frame}
\frametitle{The Mechanics}
For ANOVA...
\vskip10pt
\begin{columns}
\column{0.5\textwidth}
\underline{you do need to know}
\begin{itemize}
    \item
    how to set up $H_0$ and $H_a$
    \item
    state a conclusion regarding $H_0$ and $H_a$ based on the p-value
\end{itemize}
\column{0.5\textwidth}
\underline{you don't need to know}
\begin{itemize}
    \item
    how to calculate the test statistic
    \item
    how to shade areas for the p-value
\end{itemize}
\end{columns}
\end{frame}

\begin{frame}[fragile]
\frametitle{Results in R}
\small{$H_0: \mu_1=\mu_2=\mu_3=\mu_4$ vs $H_a$: at least one mean differs}
%\vskip15pt
\begin{lcverbatim}
> results <- aov(BtheB$bdi.pre ~ BtheB$history)
> summary(results)
              Df Sum Sq Mean Sq F value Pr(>F)
BtheB$history  3   1118   372.8   3.404 0.0208
Residuals     96  10516   109.5
\end{lcverbatim}
\begin{clicker}{\small{The p-value is 0.0208.  What is your conclusion?}}
\begin{enumerate}
\small{
    \item
    reject $H_0$; we have evidence that the mean BDI differs in all 4 history groups
    \item
    reject $H_0$; we have evidence that at least one mean BDI differs from the others
    \item
    fail to reject $H_0$; we have evidence that the mean BDI is the same in all 4 history groups
    \item
    fail to reject $H_0$; we do not have evidence that the mean BDI differs in the 4 history groups}
\end{enumerate}
\end{clicker}
\end{frame}

\begin{frame}
\frametitle{Multiple Comparisons}
\begin{itemize}
    \item
    When you reject $H_0: \mu_1=\mu_2=\mu_3$ in one-way ANOVA, it is of interest to determine exactly which means differ and to what extent they differ
    \item
    This results in \emph{pairwise comparisons}.  If you have 4 groups, there are 6 pairwise comparisons:
    \begin{enumerate}
        \item
        group 1 vs group 2
        \item
        group 1 vs group 3
        \item
        group 1 vs group 4
        \item
        group 2 vs group 3
        \item
        group 2 vs group 4
        \item
        group 3 vs group 4
    \end{enumerate}
    \item
    For each comparison, we can answer this by either testing  $H_0: \mu_1=\mu_2$ or by calculating a confidence interval for $\mu_1-\mu_2$
\end{itemize}
\end{frame}



\begin{frame}
\frametitle{BUT...}
But this means we are doing \emph{multiple} testing, or calculating \emph{multiple} confidence intervals
\begin{itemize}
    \item
    this inflates the \emph{overall} Type I error rate of the multiple tests taken together
    \item
    this deflates the \emph{overall} coverage of the confidence intervals taken together
\end{itemize}
\begin{tabular}{p{3cm} p{7cm}}
    \hline
    Number of tests & Chance of at least one Type I error \\
    \hline
    1   &  5\% \\
    3   & 14.3\% \\
    5   & 22.6\% \\
    10   & 40.1\% \\
    20   & 64.2\% \\
    50   & 92.3\%\\
    100   & 99.4\% \\
\end{tabular}
\end{frame}


\begin{frame}
\frametitle{Multiple testing techniques}
\begin{itemize}
    \item
    The goal of multiple testing/confidence interval techniques is to \textbf{control} the overall Type I error rate in a \emph{set} of tests or the overall confidence level in a \emph{set} of confidence intervals
    \item
    There are many techniques available
    \begin{itemize}
        \item
        Fisher method
        \item
        Bonferroni method
        \item
        Tukey method
    \end{itemize}
    \item
    For STAT 217, we will utilize the \textbf{Tukey method} for confidence intervals for pairwise comparisons.
    \begin{itemize}
        \item
        gives overall confidence level very close to desired level
        \item
        confidence intervals are slightly narrower than other methods
        \item
        the formula is complex and you do not need to know it
        \item
        you can get the results in R
    \end{itemize}
\end{itemize}
\end{frame}


\begin{frame}[fragile]
\frametitle{Multiple Comparisons Results}
\begin{lcverbatim}
> TukeyHSD(results)
> plot(TukeyHSD(results))
        diff        lwr       upr     p adj
2-1 1.187500 -6.2017875  8.576787 0.9749056
3-1 1.245000 -6.5750868  9.065087 0.9755714
4-1 9.282895  0.8797670 17.686022 0.0243165
3-2 0.057500 -7.2468503  7.361850 0.9999968
4-2 8.095395  0.1699716 16.020818 0.0433700
4-3 8.037895 -0.2906418 16.366431 0.0626362
\end{lcverbatim}
\begin{columns}
\column{0.4\textwidth}
%\includegraphics[width=1.0\textwidth]{tukey_CI.png}\\
\includegraphics[width=0.90\textwidth]{Figures/tukey.pdf}
\column{0.6\textwidth}
\begin{clicker}{How many pairwise comparisons indicate differences between the population means?}
\hspace{0.1in} 0  \hspace{0.1in} 1 \hspace{0.1in}  2 \hspace{0.1in}  3  \hspace{0.1in} 4 \hspace{0.1in}  5 \hspace{0.1in}  6
\end{clicker}
\end{columns}
\end{frame}

\begin{frame}[fragile]
\frametitle{4-1 Results}
\begin{lcverbatim}
        diff        lwr       upr     p adj
4-1 9.282895  0.8797670 17.686022 0.0243165
\end{lcverbatim}
We have evidence of a difference in mean BDI score among patients in history group 4 versus 1.  But is this meaningful?
\vskip200pt
\end{frame}

%===========================================================================================================================
\section[Summary]{Summary}
%===========================================================================================================================
\begin{frame}
\tableofcontents[currentsection, hideallsubsections]
\end{frame}
%\subsection{}


\begin{frame}
\frametitle{Interpreting the $p$-value}
The $p$-value represents the \textbf{strength} of the \textbf{evidence}:\\
\begin{itemize}
    \item
    small $p$-values mean you have strong evidence of an association between two variables
    \item
    small $p$-values do not mean you have evidence of a strong association between two variables
    \item
    large $p$-values mean there is no evidence of an association
    \item[]
\end{itemize}
Other measures represent the \textbf{strength} of the \textbf{association}:\\
\begin{itemize}
    \item
    difference of means: ($\bar{x}_1-\bar{x}_2$)
\end{itemize}
The \textbf{strength} of the \textbf{association} can help you assess if an observed difference is meaningful.
\end{frame}

\begin{frame}
\small{A researcher investigated if eye color of the mother (brown vs blue) is associated with birth weight of a baby (in pounds).}
\begin{center}
\includegraphics[width=0.80\textwidth]{Figures/ex_ttest.png}
\end{center}
\begin{clicker}{\small{What is the best conclusion from this analysis?  We \underline{(do/do not)} have strong evidence that weight of babies differs by mother's eye color, and the effect of eye color is \underline{(strong/weak)}.}}
\small{
\begin{enumerate}
    \item
    do; strong
    \item
    do not; strong
    \item
    do; weak
    \item
    do not; weak
\end{enumerate}}
\end{clicker}
\end{frame}


\begin{frame}[label=methods]
\frametitle{Different methods}
\resizebox{1.0\textwidth}{!}{
\begin{tabular}{|lllll|}
    \hline
    \textbf{Method} & \textbf{Use} & \textbf{Variables} & \textbf{Estimation} & \textbf{Testing}\\
     \hline
    Single mean &  quantitative response & one quantitative variable  & CI for $\mu$ & $H_0$: $\mu=\mu_0$\\
    \emph{(one-sample $t$-test)}                  & in single group & & &\\
    \hline
    $^*$Two means & quantitative response &  one quantitative variable and & CI for $\mu_1-\mu_2$ & $H_0$: $\mu_1=\mu_2$\\
    \emph{(two-sample $t$-test)}                   &  in two groups & one categorical variable & &\\
    \hline
    Dependent means & quantitative response & two paired & CI for $\mu_d$ & $H_0$: $\mu_d=0$\\
    \emph{(paired $t$-test)} & measured on same observation & quantitative variables & & \\
    \hline
    $^*$ANOVA &  quantitative response & one quantitative variable and & Tukey pairwise & $H_0$: $\mu_1=\mu_2=\cdots=\mu_g$\\
                       & in $>2$ groups & one categorical variable & intervals &\\
    \hline
\end{tabular}}
\vskip10pt
\footnotesize{$^*$The starred methods can answer the question ``Is there an association?''  If we reject $H_0$, then we conclude that some sort of association is present in the two variables.}
\end{frame}


\end{document} 