

\PassOptionsToPackage{subsection=false}{beamerouterthememiniframes}
\PassOptionsToPackage{dvipsnames,table}{xcolor}
\documentclass[fleqn]{beamer}
\usepackage{graphicx}
\usepackage{multirow}
\usepackage{multicol}
\usepackage{amsmath,amsfonts,amsthm,amsopn}
\usepackage{color, colortbl}
\usepackage{subfig}
\usepackage{wrapfig}
\usepackage{fancybox}
\usepackage{tikz}
\usepackage{fancyhdr}
\usepackage{setspace}
\usepackage{xcolor}
\usepackage{movie15}
\usepackage{pifont}
\usepackage{soul}
\usepackage{booktabs}
\usepackage{fancyvrb,newverbs}
\fvset{fontsize=\footnotesize}
\RecustomVerbatimEnvironment{verbatim}{Verbatim}{}

%\usepackage{fancybox}

\usetheme{Szeged}
\usecolortheme{default}

%\definecolor{links}{HTML}{2A1B81}
%\definecolor{links}{blue!20}
\hypersetup{colorlinks,linkcolor=,urlcolor=blue!80}

\setbeamertemplate{blocks}[rounded]
\setbeamercolor{block title}{bg=blue!40,fg=black}
\setbeamercolor{block body}{bg=blue!10}

%\definecolor{myblue1}{blue!10}

%\colorlet{breaks}{myblue1}

\newenvironment<>{clicker}[1]{%
  \begin{actionenv}#2%
      \def\insertblocktitle{#1}%
      \par%
      \mode<presentation>{%
        \setbeamercolor{block title}{fg=white,bg=magenta}
       \setbeamercolor{block body}{fg=black,bg=magenta!10}
       \setbeamercolor{itemize item}{fg=magenta}
       \setbeamertemplate{itemize item}[triangle]
       \setbeamercolor{enumerate item}{fg=magenta}
     }%
      \usebeamertemplate{block begin}}
    {\par\usebeamertemplate{block end}\end{actionenv}}




\defbeamertemplate*{footline}{infolines theme}
{
  \leavevmode%
  \hbox{%
  \begin{beamercolorbox}[wd=.333333\paperwidth,ht=2.25ex,dp=1ex,left]{author in head/foot}%
    \usebeamerfont{author in head/foot}~~\insertshortinstitute: \insertshorttitle
  \end{beamercolorbox}%
  \begin{beamercolorbox}[wd=.67\paperwidth,ht=2.25ex,dp=1ex,right]{date in head/foot}%
    \usebeamerfont{date in head/foot}%\insertshortdate{}\hspace*{2em}
    \insertframenumber{} / \inserttotalframenumber\hspace*{2ex}
  \end{beamercolorbox}
  }%
  \vskip0pt%
}

\newcommand{\cmark}{\ding{51}}%
\newcommand{\xmark}{\ding{55}}%
\newcommand{\grp}{\textcolor{magenta}{Group Exercise}}
\newcommand{\bsans}[1]{\underline{\hspace{0.2in}\color{blue!80}{#1}\hspace{0.2in}}}
\newcommand{\bs}{\underline{\hspace{0.3in}}}


\definecolor{cverbbg}{gray}{0.93}
\newenvironment{cverbatim}
 {\SaveVerbatim{cverb}}
 {\endSaveVerbatim
  \flushleft\fboxrule=0pt\fboxsep=.5em
  \colorbox{cverbbg}{\BUseVerbatim{cverb}}%
  \endflushleft
}
\newenvironment{lcverbatim}
 {\SaveVerbatim{cverb}}
 {\endSaveVerbatim
  \flushleft\fboxrule=0pt\fboxsep=.5em
  \colorbox{cverbbg}{%
    \makebox[\dimexpr\linewidth-2\fboxsep][l]{\BUseVerbatim{cverb}}%
  }
  \endflushleft
}



\title[Unit 2 Deck 4]{Type I/II errors and Exam Practice}
\author[Pileggi]{Shannon Pileggi}

\institute[STAT 217]{STAT 217}

\date{}


\begin{document}

\begin{frame}
\titlepage
\end{frame}

\begin{frame}
\frametitle{OUTLINE\qquad\qquad\qquad} \tableofcontents[hideallsubsections]
\end{frame}


%===========================================================================================================================
\section[Type I/II Errors]{Errors in Hypothesis Testing}
%===========================================================================================================================
%\subsection{}

\begin{frame}
%\frametitle{Clicker}
\begin{clicker}
{When we draw a conclusion about a hypothesis test (ie, reject or fail to reject $H_0$), do we always make the correct decision?}
\begin{enumerate}
    \item
    Yes
    \item
    No
\end{enumerate}
\end{clicker}
\end{frame}

\begin{frame}
\frametitle{Possible outcomes of a hypothesis test}
\begin{tabular}{l|cc|}
\cline{2-3}
\textbf{Decision based} & \multicolumn{2}{|c|}{\textbf{Unknown Truth}}\\
\textbf{on observed data}                                                         & $H_0$ true                   & $H_0$ false \\
\cline{2-3}
  Fail to reject $H_0$ & \visible<2->{\textcolor{blue}{Correct Decision}}  & \visible<5->{\textcolor{red}{Type II Error}}\\
                                        Reject $H_0$ & \visible<4->{\textcolor{red}{Type I Error}} & \visible<3->{\textcolor{blue}{Correct Decision}}\\
  \cline{2-3}
\end{tabular}
\vskip20pt
\begin{itemize}
    \visible<4->{
    \item
    A \textbf{Type I error} occurs when $H_0$ is true in reality but is rejected based on evidence from the test.}
    \visible<5->{
    \item
    A \textbf{Type II error} occurs when $H_0$ is false in reality (or $H_a$ true) but you fail to reject $H_0$ based on evidence from the test.}
\end{itemize}
\end{frame}

\begin{frame}
\frametitle{Type I/Type II error example}
\framesubtitle{Jury trial}
\begin{tabular}{l|cc|}
    \cline{2-3}
     & \multicolumn{2}{|c|}{\textbf{Unknown Truth}}\\
    \textbf{Decision by}    & Defendant is        & Defendant is  \\
    \textbf{the jury}       & innocent ($H_0$)    & guilty ($H_a$) \\
    \cline{2-3}
  Innocent & \visible<2->{\textcolor{blue}{Correct Decision}}  & \visible<4->{\textcolor{red}{Type II Error}}\\
                                        Guilty & \visible<3->{\textcolor{red}{Type I Error}} & \visible<2->{\textcolor{blue}{Correct Decision}}\\
  \cline{2-3}
\end{tabular}
\vskip20pt
\begin{itemize}
    \visible<3->{
    \item
    A \textbf{Type I error} occurs when the jury finds a truly innocent man to be guilty.}
    \visible<4->{
    \item
    A \textbf{Type II error} occurs when the jury finds a truly guilty man to be innocent.}
\end{itemize}
\visible<5->{
\begin{clicker}{Which error do you think is worse?}
\begin{enumerate}
    \item
    Type I
    \item
    Type II
\end{enumerate}
\end{clicker}
}
\end{frame}

\begin{frame}
\frametitle{Type I/Type II error example}
%\framesubtitle{Screening for breast cancer by mammogram}
\resizebox{1.0\textwidth}{!}{
\begin{tabular}{l|cc|}
    \cline{2-3}
    & \multicolumn{2}{|c|}{\textbf{Unknown Truth}}\\
    \textbf{Decision based}     & Woman does not have  & Woman has breast  \\
    \textbf{on screening test} &  breast cancer ($H_0$) &  cancer ($H_a$) \\
    \cline{2-3}
    No suspicion of breast cancer & \textcolor{blue}{Correct Decision}  & \textcolor{red}{Type II Error}\\
    Suspicion of breast cancer & \textcolor{red}{Type I Error} & \textcolor{blue}{Correct Decision}\\
  \cline{2-3}
\end{tabular}}
\begin{itemize}
    \pause
    \item
    A \textbf{Type I error} occurs when the woman does not have breast cancer, but the mammogram indicates that she may.  This is a false positive.
    \pause
    \item
    A \textbf{Type II error} occurs when the woman does have breast cancer, but the mammogram indicates that does not.  This is a false negative.
\end{itemize}
\pause
\begin{clicker}{Which error do you think is worse?}
\begin{enumerate}
    \item
    Type I
    \item
    Type II
\end{enumerate}
\end{clicker}
\end{frame}

\begin{frame}
\begin{clicker}{If I conduct a hypothesis test and I reject the null hypothesis based on evidence from my data, this could have been the result of... \emph{(mark all that apply)}}
\begin{enumerate}
    \item
    a Type I error
    \item
    a Type II error
    \item
    a correct decision
\end{enumerate}
\end{clicker}
\end{frame}

\begin{frame}
\frametitle{Type I error}
\begin{itemize}
    \item
    A \textbf{Type I error} occurs when $H_0$ is true but is rejected.
    \item
    $\alpha$ = Pr(Type I error) = Pr(reject $H_0$ when  $H_0$ true)
    \item
    The choice of significance level controls the probability of a Type I error.
    \item
    The more serious the consequences of a Type I error, the smaller $\alpha$ should be.
\end{itemize}
\end{frame}


\begin{frame}
\frametitle{Type II error}
\begin{itemize}
    \item
    A \textbf{Type II error} occurs when $H_0$ is false (or $H_a$ true) but $H_0$ is \emph{not} rejected.
    \item[]
    Pr(Type II error)=Pr(fail to reject $H_0$ when $H_0$ false)
    %\item
    %The Type II error is defined when the null hypothesis is false, and correspondingly when the alternative hypothesis is true.  The alternative hypothesis is comprised of a range of values (which includes infinitely many values).  Therefore, the Pr(Type II error) can only be calculated by hand for a specific value under $H_a$.
    \item
    The Pr(Type II error) is a function of many things, including: $\alpha$, $n$, $s$, and $\mu_0$ - you won't need to calculate this.
\end{itemize}
\end{frame}

\begin{frame}
Suppose we interested in determining if the population mean length of the longest serious relationship among Cal Poly students differs from 9 months.  That is, we test $H_0$: $\mu=9$ vs $H_a$: $\mu\neq 9$.
\begin{clicker}{What would be an example of a Type I error? Finding that we \underline{(do/do not)} have evidence that the population mean length of longest serious relationship differs from 9 when in reality the population mean length of longest serious relationship \underline{(equals/does not equal)} 9 months.}
\begin{enumerate}
    \item
    do; equals 
    \item
    do not; does not equal 
    \item
    do; does not equal
    \item
    do not; equals 
\end{enumerate}
\end{clicker}
\end{frame}


\begin{frame}
\frametitle{We can't have it all}
\begin{itemize}
    \item
    We cannot simultaneously minimize Pr(Type I error) and Pr(Type II error).
    \item
    Pr(Type I error) and Pr(Type II error) are inversely related - as one goes down the other must go up.
    \item
    If we minimize the chance of a Type I error by making $\alpha$ smaller, we increase our chances of committing a Type II error.
%    This is because:
%    \begin{itemize}
%    \item
%    With smaller $\alpha$, we need a smaller $p$-value to reject $H_0$.
%    \item
%    So it is harder to reject $H_0$.
%    \item
%    Which means it is easier to fail to reject $H_0$, and thus easier to commit a Type II error.
%    \end{itemize}
\end{itemize}
%\vskip5pt
%\footnotesize{$^*$Note that we generally fix the Pr(Type I error) as this is established by our level of significance $\alpha$.}
\end{frame}

%\begin{frame}
%\frametitle{Power}
%\begin{itemize}
%    \item
%    The \textbf{power} of a test is the Pr(reject $H_0|H_0$ false).
%    \item
%    Power$=1-$Pr(Type II error)
%    \item
%    You can think of this as the ability of the test to detect an \emph{effect} when it is truly there.
%    \item
%    So, for example, if the true mean GPA is 3.35 and we are testing that it equals 3.5, we want sufficient power to detect the 0.15 difference in average GPA.
%    \item
%    We want the power of a test to be high in order to be able to reject $H_0$ when it really is false, and detect the effect.
%    \item
%    It is typical for a study to have a power of 80\% or 90\%.
%\end{itemize}
%\end{frame}


\begin{frame}
\frametitle{Importance of conditions}
We always have some conditions we need for \emph{valid} inference.  When these conditions are violated, but we use the statistical method anyway, this results in \emph{invalid} inference.  Invalid inference can mean:
\begin{itemize}
    \item
    your confidence interval isn't capturing the parameter as often as it should (ie, lower than 95\%)
    \item
    you commit Type I errors more than you should (ie, more than 5\%)
    \item
    basically, can't trust your CI or your $p$-value
\end{itemize}
\end{frame}

%===========================================================================================================================
\section[Practice Problems]{Practice Problems}
%===========================================================================================================================


\begin{frame}
\tableofcontents[currentsection, hideallsubsections]
\end{frame}
%\subsection{}

\begin{frame}
\begin{clicker}{For which of the following $p$-values would I reject the null hypothesis at the $\alpha = 0.10$ level of significance?  Mark all that apply.}
\begin{enumerate}
    \item
    0.01
    \item
    0.08
    \item
    0.15
    \item
    0.36
    \item
    0.89
\end{enumerate}
\end{clicker}
\end{frame}



\begin{frame}
\frametitle{Clicker}
We are interested in describing the math SAT scores of Cal Poly students.  Assume that math SAT scores of Cal Poly students are right-skewed with $\mu=600$ and $\sigma=75$.  Consider the sampling distribution of the sample mean when we take a sample of size $n=100$.
\begin{clicker}
{Can we assume that the sampling distribution of the sample mean is approximately normally distributed?}
    \begin{enumerate}
        \item
        Yes, because $n$ is bigger than 30.
        \item
        Yes, because $n$ is bigger than 10.
        \item
        Yes, because $np>10$ and $n(1-p)>10$.
        \item
        No, because $np$ or $n(1-p)$ is not greater than 10.
        \item
        No, because the population is right-skewed.
    \end{enumerate}
\end{clicker}
\end{frame}

\begin{frame}[fragile]
\small{Below are the summary statistics of the data and output from the analysis testing if the population average birth weight of the monkeys is 0.4kg.}
\begin{verbatim}
  min   Q1 median  Q3  max mean   sd  n missing
 0.27 0.37   0.39 0.5 0.68 0.44 0.12 10       0

t = 1.0853, df = 9, p-value = 0.306
alternative hypothesis: true mean is not equal to 0.4
95 percent confidence interval:
 XXXXXXX XXXXXXX
\end{verbatim}
\begin{clicker}{Which is the correct calculation to estimate the population average birth weight of rhesus monkeys with a 95\% CI?}
\begin{enumerate}
\item	$0.44 \pm 1.0853 \times 0.12 / \sqrt{10} $
\item	$0.44 \pm 1.0853 \times 0.12  $
\item	$0.44 \pm 2.26 \times 0.12 / \sqrt{10} $
\item	$0.39 \pm 1.96 \times 0.12 $
\item  $0.39 \pm 2.26 \times 0.12 / \sqrt{9} $
\end{enumerate}
\end{clicker}
\end{frame}

\begin{frame}
Suppose I test if the average age of retirement for Americans differs from 65 ($H_0$: $\mu=65$ vs $H_a$: $\mu \neq 65$), and I get a test statistic of 1.5.  Which figure corresponds to the $p$-value?\\
\vskip20pt
\includegraphics[width=1.0\textwidth,trim=0mm 50mm 0mm 50mm]{Figures/pmatrix.pdf}
\end{frame}

\begin{frame}
\begin{clicker}{The level of significance $\alpha$ at which the test is performed (and hence, the corresponding confidence level of a CI) affects...}
\begin{enumerate}
        \item
        both the value of the test statistic and the width of a confidence interval.
        \item
        only the value of the test statistic and not the width of a confidence interval.
        \item
        only the width of a confidence interval and not the value of the test statistic.
        \item
        neither the value of the test statistic nor the width of a confidence interval.
\end{enumerate}
\end{clicker}
\end{frame}



\begin{frame}[fragile]
\frametitle{Number of states visited}
\begin{verbatim}
> favstats(survey$States)
 min Q1 median Q3 max     mean       sd  n missing
   1  5      8 13  35 9.820896 6.797783 67       0
 \end{verbatim}
\begin{columns}
\column{0.5\textwidth}
\includegraphics[width=1.0\textwidth]{Figures/states.pdf}
\column{0.5\textwidth}
\begin{clicker}{Are conditions satisfied for the following one-sample t-test?  $H_0$: $\mu=8$ vs $H_a$: $\mu\neq8$, where $\mu$ = population mean number of states visited}
\begin{enumerate}
    \item
    yes, $n \geq 30$
    \item
    no, distribution is right skewed
    \item
    yes, $n\hat{p} \geq 10$ and $n(1-\hat{p}) \geq 10$
    \item
    no, $n\hat{p} < 10$ OR $n(1-\hat{p}) < 10$
\end{enumerate}
\end{clicker}
\end{columns}
\end{frame}

\begin{frame}[fragile]
\begin{verbatim}
	One Sample t-test

data:  survey$States
t = 2.1926, df = 66, p-value = 0.03187
alternative hypothesis: true mean is not equal to 8
95 percent confidence interval:
  8.162786 11.479005
sample estimates:
mean of x
 9.820896
\end{verbatim}
\begin{clicker}{Which of the following statements are \underline{true} at $\alpha=0.05$?}
\begin{enumerate}
    \item
    67 individuals were included in this sample
    \item
    the null hypothesis is $\mu\neq 8$
    \item
    the $z$ distribution was used to calculate the $p$-value
    \item
    there is evidence that the null hypothesis is true
    \item
    all are true
    \item
    none are true
\end{enumerate}
\end{clicker}
\end{frame}

\begin{frame}
\begin{clicker}{Suppose Cal Poly administrators want to determine the proportion of students that graduate within 4 years, and they calculate a 95\% confidence interval to be (0.4, 0.8).  They aren't happy with the interval because it is so wide, and hence not very informative.  What would you recommend that they do?}
\begin{enumerate}
    \item
    increase their confidence level
    \item
    take a larger sample
    \item
    perform a hypothesis test
    \item
    reduce the standard deviation of whether or not students graduate
    \item
    more than one of these items
\end{enumerate}
\end{clicker}
\end{frame}

\begin{frame}
The percentage of Cal Poly students that participate in Greek life is 20\%.  Suppose I were to repeatedly sample groups of $n=100$ Cal Poly students.  Describe the sampling distribution of the sample proportion in terms of the
\begin{itemize}
    \item
    shape
    \item
    mean
    \item
    standard deviation
\end{itemize}
of the distribution.
\end{frame}




\begin{frame}
\frametitle{Clicker}
We tested if the average amount of hours worked per week differs from 38 at the $\alpha=0.01$ level of significance. ($H_0$: $\mu=38$ vs $H_a$: $\mu \neq 38$) In the data, $\bar{x}=37.1$, and we get a $p$-value equal to 0.0299.
\begin{clicker}{What can be said about a 99\% confidence interval for the population mean?}
\begin{enumerate}
  \item
  A 99\% confidence interval would contain 38.
  \item
  A 99\% confidence interval would not contain 38.
  \item
  A 99\% confidence interval would not contain 37.1.
  \item
  A 99\% confidence interval would contain 0.0299.
  \item
  A 99\% confidence interval would contain 0.
  \item
  Not enough information to determine.
\end{enumerate}
\end{clicker}
\end{frame}

\begin{frame}
\frametitle{Clicker}
Supposed I used a one-sample t-test to determine if average income of US citizens is less than \$45,000. \\
$H_0$: $\mu=45000$ vs $H_a$: $\mu < 45000$
\begin{clicker}{What would a Type II error be in the context of this research question?  Finding that we \underline{(do/do not)} have evidence that the population average income is less than \$45,000, when in reality the population average income \underline{(equals \$45,000/is less than \$45,000)}.}
\begin{enumerate}
    \item
    do not; is less than \$45,000
    \item
    do not; equals \$45,000
    \item
    do; equals \$45,000
    \item
    do; is less than \$45,000
\end{enumerate}
\end{clicker}
\end{frame}

\end{document} 