

\PassOptionsToPackage{subsection=false}{beamerouterthememiniframes}
\PassOptionsToPackage{dvipsnames,table}{xcolor}
\documentclass[fleqn]{beamer}
\usepackage{graphicx}
\usepackage{multirow}
\usepackage{multicol}
\usepackage{amsmath,amsfonts,amsthm,amsopn}
\usepackage{color, colortbl}
\usepackage{subfig}
\usepackage{wrapfig}
\usepackage{fancybox}
\usepackage{tikz}
\usepackage{fancyhdr}
\usepackage{setspace}
\usepackage{xcolor}
\usepackage{movie15}
\usepackage{pifont}
\usepackage{soul}
\usepackage{booktabs}
\usepackage{fancyvrb,newverbs}
\fvset{fontsize=\footnotesize}
\RecustomVerbatimEnvironment{verbatim}{Verbatim}{}

%\usepackage{fancybox}

\usetheme{Szeged}
\usecolortheme{default}

%\definecolor{links}{HTML}{2A1B81}
%\definecolor{links}{blue!20}
\hypersetup{colorlinks,linkcolor=,urlcolor=blue!80}

\setbeamertemplate{blocks}[rounded]
\setbeamercolor{block title}{bg=blue!40,fg=black}
\setbeamercolor{block body}{bg=blue!10}

%\definecolor{myblue1}{blue!10}

%\colorlet{breaks}{myblue1}

\newenvironment<>{clicker}[1]{%
  \begin{actionenv}#2%
      \def\insertblocktitle{#1}%
      \par%
      \mode<presentation>{%
        \setbeamercolor{block title}{fg=white,bg=magenta}
       \setbeamercolor{block body}{fg=black,bg=magenta!10}
       \setbeamercolor{itemize item}{fg=magenta}
       \setbeamertemplate{itemize item}[triangle]
       \setbeamercolor{enumerate item}{fg=magenta}
     }%
      \usebeamertemplate{block begin}}
    {\par\usebeamertemplate{block end}\end{actionenv}}




\defbeamertemplate*{footline}{infolines theme}
{
  \leavevmode%
  \hbox{%
  \begin{beamercolorbox}[wd=.333333\paperwidth,ht=2.25ex,dp=1ex,left]{author in head/foot}%
    \usebeamerfont{author in head/foot}~~\insertshortinstitute: \insertshorttitle
  \end{beamercolorbox}%
  \begin{beamercolorbox}[wd=.67\paperwidth,ht=2.25ex,dp=1ex,right]{date in head/foot}%
    \usebeamerfont{date in head/foot}%\insertshortdate{}\hspace*{2em}
    \insertframenumber{} / \inserttotalframenumber\hspace*{2ex}
  \end{beamercolorbox}
  }%
  \vskip0pt%
}

\newcommand{\cmark}{\ding{51}}%
\newcommand{\xmark}{\ding{55}}%
\newcommand{\grp}{\textcolor{magenta}{Group Exercise}}
\newcommand{\bsans}[1]{\underline{\hspace{0.2in}\color{blue!80}{#1}\hspace{0.2in}}}
\newcommand{\bs}{\underline{\hspace{0.3in}}}


\definecolor{cverbbg}{gray}{0.93}
\newenvironment{cverbatim}
 {\SaveVerbatim{cverb}}
 {\endSaveVerbatim
  \flushleft\fboxrule=0pt\fboxsep=.5em
  \colorbox{cverbbg}{\BUseVerbatim{cverb}}%
  \endflushleft
}
\newenvironment{lcverbatim}
 {\SaveVerbatim{cverb}}
 {\endSaveVerbatim
  \flushleft\fboxrule=0pt\fboxsep=.5em
  \colorbox{cverbbg}{%
    \makebox[\dimexpr\linewidth-2\fboxsep][l]{\BUseVerbatim{cverb}}%
  }
  \endflushleft
}




\title[Unit 2 Deck 1]{Distribution of Sample Proportions and Confidence Interval for a Population Proportion}
\author[Pileggi]{Shannon Pileggi}

\institute[STAT 217]{STAT 217}

\date{}


\begin{document}

\begin{frame}
\titlepage
\end{frame}

\begin{frame}
\frametitle{OUTLINE\qquad\qquad\qquad} \tableofcontents[hideallsubsections]
\end{frame}



%===========================================================================================================================
\section[Overview]{Overview}
%===========================================================================================================================

%\subsection{}
\begin{frame}[fragile]
\frametitle{The Data}
From the CDC's 2013 Youth Risk Behavior Surveillance System \\
\vskip10pt
 \resizebox{1.0\textwidth}{!}{
\begin{tabular}{llllllll}
\hline
	&	gender	&	height\textunderscore m	&	weight\textunderscore kg	&	bmi	&	carried\textunderscore weapon	&	bullied	&	days\textunderscore drink	\\
\hline
1	&	female	&	1.73	&	84.37	&	28.2	&	yes	&	no	&	30	\\
2	&	female	&	1.6	&	55.79	&	21.8	&	no	&	yes	&	1	\\
3	&	female	&	1.5	&	46.72	&	20.8	&	no	&	yes	&	0	\\
4	&	female	&	1.57	&	67.13	&	27.2	&	no	&	yes	&	0	\\
5	&	female	&	1.68	&	69.85	&	24.7	&	no	&	no	&	0	\\
6	&	female	&	1.65	&	66.68	&	24.5	&	no	&	no	&	1	\\
7	&	male	&	1.85	&	74.39	&	21.7	&	no	&	no	&	0	\\
8	&	male	&	1.78	&	70.31	&	22.2	&	yes	&	no	&	0	\\
9	&	male	&	1.73	&	73.48	&	24.6	&	no	&	yes	&	0	\\
10	&	male	&	1.83	&	67.59	&	20.2	&	no	&	no	&	0	\\
$\vdots$	&	$\vdots$	&$\vdots$		&$\vdots$		&	$\vdots$	&$\vdots$		&	$\vdots$	&$\vdots$		\\
8482	&	male	&	1.73	&	68.95	&	23	&	no	&	no	&	0	\\
\hline
\end{tabular}}
\end{frame}


\begin{frame}
\frametitle{The idea}
\vskip10pt
\begin{columns}
\column{0.27\textwidth}
\includegraphics[width=1.0\textwidth]{Figures/barplot_all.pdf}\\
\vskip20pt
\includegraphics[width=1.0\textwidth]{Figures/barplot_samp3.pdf}
\column{0.27\textwidth}
\includegraphics[width=1.0\textwidth]{Figures/barplot_samp1.pdf}\\
\vskip20pt
\includegraphics[width=1.0\textwidth]{Figures/barplot_samp4.pdf}
\column{0.27\textwidth}
\includegraphics[width=1.0\textwidth]{Figures/barplot_samp2.pdf}\\
\vskip20pt
\includegraphics[width=1.0\textwidth]{Figures/barplot_samp5.pdf}
\end{columns}
\end{frame}



\begin{frame}
\frametitle{Parameters vs statistics}
\begin{columns}
\column{0.5\textwidth}
A \textbf{parameter} is a numerical summary of a population.  It is usually \emph{unknown}, although we can make \emph{assumptions} about parameter values for population distributions.  We generally use Greek letters (without bars or hats) to denote population parameters: \\
\vskip20pt
\resizebox{0.95\textwidth}{!}{
\begin{tabular}{|ll|}
    \hline
    population mean & $\mu$\\
    population standard deviation & $\sigma$ \\
    population proportion & $p$ \\
    \hline
\end{tabular}}
\column{0.5\textwidth}
A \textbf{statistic} is a numerical summary of the sample.  It is estimated from observed data.  We generally use lower case letters (with bars or hats) to denote sample statistics: \\
\vskip60pt
\resizebox{0.85\textwidth}{!}{
\begin{tabular}{|ll|}
    \hline
    sample mean & $\bar{x}$\\
    sample standard deviation & $s$ \\
    sample proportion & $\hat{p}$ \\
    \hline
\end{tabular}}
\end{columns}
\end{frame}


\begin{frame}
\frametitle{Describing distributions}
\grp
\begin{clicker}{When describing distributions, what are the three features you should address?}
\begin{enumerate}
\item
\item[]
\item
\item[]
\item
\item[]
\end{enumerate}
\end{clicker}
\end{frame}

\begin{frame}
\frametitle{Three distributions to keep in mind}
\begin{enumerate}
 \item   The \textbf{population distribution} refers to the actual distribution of a variable in a population.  \\ \vskip5pt
 \item   The \textbf{data distribution} refers to the distribution of observed values from a \emph{single} sample.  \\ \vskip5pt
 \item     The \textbf{sampling distribution} refers to the distribution of a statistic from \emph{many} samples.\\ \vskip5pt
\item[] \begin{tabular}{p{4.2cm} p{0.2cm} p{4.8cm}}
\small{sampling distribution of the sample proportion} & = & \small{the distribution of sample proportions ($\hat{p}$) from many samples} \\
 & & \\
 \small{sampling distribution of the sample means}  & = &  \small{the distribution of sample means ($\bar{x}$) from many samples}
\end{tabular}
%    \begin{itemize}
%        \item
%        sampling distribution of the sample proportion = the distribution of sample proportions ($\hat{p}$ from many samples
%        \item
%        sampling distribution of the sample means  the distribution of sample means ($\bar{x}$ from many samples
%    \end{itemize}
\end{enumerate}
\end{frame}


%===========================================================================================================================
\section[Simulation]{Simulation}
%===========================================================================================================================
\begin{frame}
\tableofcontents[currentsection, hideallsubsections]
\end{frame}

%\subsection{}
\begin{frame}
\frametitle{Population and data distribution of \texttt{bullied}}
Consider the 8,482 observations from the YRBSS data set to be the \emph{entire} population of interest.  Now let's describe the \textbf{population distribution} of \texttt{bullied}.
\begin{itemize}
    \item
    True population proportion:
    \item[]
\end{itemize}
\end{frame}

\begin{frame}
\frametitle{Example data distributions from \texttt{bullied}}
 Now let's take three random samples of size $n=10$ from the population distribution of \texttt{bullied}.  Each random sample represents a \textbf{data distribution}.\\
 \vskip10pt
 %\hspace*{-20pt}
 \resizebox{1.0\textwidth}{!}{
 \begin{tabular}{|l|cccccccccc|c|}
 \hline
         & $x_1$ & $x_2$ & $x_3$ & $x_4$ & $x_5$ & $x_6$ & $x_7$ & $x_8$ & $x_9$ & $x_{10}$ &  $\hat{p}$ \\
         \hline
Sample 1 &       &       &       &       &       &       &       &       &       &          &   \textcolor{white}{1234}    \\
[2.0ex]
Sample 2 &       &       &       &       &       &       &       &       &       &          &      \\
[2.0ex]
Sample 3 &       &       &       &       &       &       &       &       &       &          &      \\
[2.0ex]
\hline
\end{tabular}}
\end{frame}




\begin{frame}
\frametitle{Many samples from \texttt{bullied}}
 Let's repeat the process and take 1000 random samples of size $n=10$ from the population distribution of \texttt{bullied}.
  \vskip5pt
   \resizebox{1.0\textwidth}{!}{
 \begin{tabular}{|l|cccccccccc|c|}
    \hline
 	& $x_1$ & $x_2$ & $x_3$ & $x_4$ & $x_5$ & $x_6$ & $x_7$ & $x_8$ & $x_9$ & $x_{10}$  & $\hat{p}$ \\
    \hline
Sample1	&	no	&	yes	&	no	&	no	&	no	&	no	&	no	&	no	&	no	&	yes	&	0.2	\\
Sample2	&	no	&	no	&	no	&	no	&	no	&	no	&	no	&	no	&	no	&	yes	&	0.1	\\
Sample3	&	no	&	no	&	yes	&	no	&	no	&	yes	&	yes	&	no	&	no	&	no	&	0.3	\\
Sample4	&	no	&	no	&	no	&	yes	&	yes	&	no	&	no	&	no	&	yes	&	no	&	0.3	\\
Sample5	&	no	&	yes	&	no	&	no	&	no	&	no	&	no	&	yes	&	no	&	no	&	0.2	\\
Sample6	&	no	&	n	&	no	&	yes	&	yes	&	yes	&	yes	&	no	&	no	&	no	&	0.4	\\
Sample7	&	yes	&	no	&	no	&	no	&	no	&	no	&	no	&	no	&	no	&	yes	&	0.2	\\
Sample8	&	no	&	yes	&	yes	&	yes	&	no	&	no	&	no	&	no	&	no	&	yes	&	0.4	\\
Sample9	&	no	&	yes	&	no	&	no	&	no	&	no	&	no	&	yes	&	no	&	no	&	0.2	\\
Sample10	&	no	&	yes	&	no	&	no	&	no	&	no	&	no	&	no	&	no	&	no	&	0.1	\\
Sample11	&	yes	&	no	&	no	&	no	&	no	&	no	&	no	&	no	&	no	&	no	&	0.1	\\
$\vdots$	&	$\vdots$	&	$\vdots$	&	$\vdots$	&	$\vdots$	&	$\vdots$	&	$\vdots$	&	$\vdots$	&	$\vdots$	&	$\vdots$	&	$\vdots$	&	$\vdots$	\\
Sample1000	&	no	&	no	&	no	&	no	&	no	&	no	&	no	&	no	&	no	&	no	&	0	\\
\hline
 \end{tabular}}
  \end{frame}


\begin{frame}
\frametitle{\grp}
\begin{clicker}{What do you think will be the shape of the distribution of the 1000 sample proportions?}
\begin{enumerate}
    \item
    bell-shaped
    \item
    left-skewed
    \item
    right-skewed
    \item
    uniform
\end{enumerate}
\end{clicker}
\end{frame}



\begin{frame}
\frametitle{Simulated sampling distribution, example 1}
The collection of the sample proportions from the 1000 samples of size $n=10$ represents a simulated \textbf{sampling distribution} of the sample proportion.
\begin{itemize}
    \item[]
    \item
    Shape of the sampling distribution:
    \item[]
    \item
    Mean of the sampling distribution:
    \item[]
    \item
    Standard deviation of the sampling distribution:
    \item[]
\end{itemize}
\end{frame}

\begin{frame}
\frametitle{Re-cap, example 1}
\begin{columns}
\column{0.33\textwidth}
\textbf{Population distribution}\\
\vskip10pt
$p=0.194$
\vskip110pt
\textcolor{white}{nothing}
\column{0.33\textwidth}
\textbf{Single data distribution ($n=10$)}\\
\vskip10pt
\texttt{no yes	no	no	no	no	no	no	no	yes}\\
\vskip10pt
$\hat{p}=0.2$	
\vskip80pt
\textcolor{white}{nothing}
\column{0.33\textwidth}
\textbf{Distribution of 1000 sample proportions from samples of size $n=10$}\\
\includegraphics[width=1.0\textwidth]{Figures/Sampling1Bullied.pdf}
\begin{center}
    mean = 0.202\\
     sd  = 0.127
\end{center}
\end{columns}
\end{frame}

%\begin{frame}
%\frametitle{Re-cap, example 1}
%\begin{minipage}{0.3\textwidth} \textcolor{white}{CHECK} \end{minipage}
%\begin{minipage}{0.64\textwidth}
%\begin{itemize}
%\item[\textbf{Population distribution:}] In the population of all individuals, the proportion bullied is $p=0.194$
%\item[]
%\item[\textbf{Data distribution:}] In one sample of 10 individuals, we observed 1 person that was bullied ($\hat{p}=0.10$)
%\item[]
%\item[\textbf{Sampling distribution:}] From 1000 samples of size $n=10$ individuals, we calculated 1000 sample proportions.  These 1000 sample proportions had:
%\begin{itemize}
%    \item Shape: right-skewed
%    \item Mean: 0.202
%    \item Standard deviation: 0.127
%\end{itemize}
%\end{itemize}
% \end{minipage}
%\end{frame}

\begin{frame}
\frametitle{\grp}
What do you think will happen to the distribution of sample proportions if we increase the sample size for each individual sample from $n=10$ to $n=100$? (The number of samples will stay the same at 1000.)
\begin{clicker}{The shape will be \underline{\hspace{1in}}, the mean will \underline{\hspace{1in}}, the standard deviation will \underline{\hspace{1in}}.}
\begin{enumerate}
    \item
    shape: right-skewed, left-skewed, approximately normal
    \item
    mean: increase, decrease, remain the same
    \item
    standard deviation: increase, decrease, remain the same
\end{enumerate}
\end{clicker}
\end{frame}



\begin{frame}
\frametitle{Simulated sampling distribution, example 2}
The collection of the sample proportions from the 1000 samples of size $n=100$ represents a simulated \textbf{sampling distribution} of the sample proportion.
\begin{itemize}
    \item[]
    \item
    Shape of the sampling distribution:
    \item[]
    \item
    Mean of the sampling distribution:
    \item[]
    \item
    Standard deviation of the sampling distribution:
    \item[]
\end{itemize}
\end{frame}

\begin{frame}
\frametitle{Re-cap, example 2}
\begin{columns}
\column{0.33\textwidth}
\textbf{Population distribution}\\
\vskip10pt
$p=0.194$
\vskip110pt
\textcolor{white}{nothing}
\column{0.33\textwidth}
\textbf{Single data distribution ($n=100$)}\\
\vskip10pt
\texttt{no no no no no no no yes no  no yes  no  no  no  no  no  no  no  no  no  no  no  no  no  no  no  yes no  no  no  no  no  no  no  no  no  no  no  no  no yes  ...}\\
\vskip10pt
$\hat{p}=0.173$	
%\vskip80pt
\textcolor{white}{nothing}
\column{0.33\textwidth}
\textbf{Distribution of 1000 sample proportions from samples of size $n=100$}\\
\includegraphics[width=1.0\textwidth]{Figures/Sampling2Bullied.pdf}
\begin{center}
    mean = 0.193\\
     sd  = 0.04
\end{center}
\end{columns}
\end{frame}


\begin{frame}
\frametitle{Summary}
{\renewcommand{\arraystretch}{1.5}
\begin{tabular}{p{0.1cm} p{2cm} p{3.7cm} p{3.7cm}}
\toprule
& Feature & Example 1 ($n=10$) & Example 2 ($n=100$) \\
\midrule
\multicolumn{3}{l}{\emph{Observed in simulation}}  \\
& Shape  & & \\
& Mean   & & \\
& Std Dev & & \\
\midrule
\multicolumn{3}{l}{\emph{According to theory}}  \\
& Shape  & & \\
& Mean   & & \\
& Std Dev & & \\
\bottomrule
\end{tabular}}
\end{frame}

%===========================================================================================================================
\section[Distribution of $\hat{p}$]{Distribution of a Sample Proportions}
%===========================================================================================================================
\begin{frame}
\tableofcontents[currentsection, hideallsubsections]
\end{frame}

%\subsection{}

\begin{frame}
\frametitle{Distribution of Sample Proportions}
\framesubtitle{OR: the sampling distribution of the sample proportion}
For a random sample of size $n$ from a population with population proportion $p$, the distribution of sample proportions has
\begin{center}
mean = $p$, standard deviation$=\displaystyle\sqrt{\frac{p(1-p)}{n}}$
\vskip50pt
Saying the same thing, but with more notation:
\vskip10pt
mean($\hat{p}$) = $p$, sd($\hat{p}$)$=\displaystyle\sqrt{\frac{p(1-p)}{n}}$
\end{center}
\end{frame}


\begin{frame}
\frametitle{Distribution of Sample Proportions}
\framesubtitle{OR: the sampling distribution of the sample proportion}
When $np\ge10$ and $n(1-p)\ge10$, then the the distribution of sample proportions has an \textbf{approximately normal} shape.  That is, when this condition is satisfied, the distribution of sample proportions has:
\begin{itemize}
\item[]
\item shape = normal
\item mean = $p$
\item standard deviation = $\displaystyle\sqrt{\frac{p(1-p)}{n}}$
\item[]
\end{itemize}
The condition $np\ge10$ and $n(1-p)\ge 10$ means that you have at least 10 expected `successes' and 10 expected `failures.'
\end{frame}

\begin{frame}
\frametitle{\grp}
\begin{clicker}{Suppose that 80\% of Americans prefer milk chocolate to dark chocolate. For which of the following sample sizes would the distribution of the sample proportions of Americans that prefers milk chocolate be approximately normally distributed? Mark \underline{all} that apply.}
\begin{enumerate}
    \item
    $n=20$
    \item
    $n=40$
    \item
    $n=60$
    \item
    $n=80$
\end{enumerate}
\end{clicker}
\end{frame}


\begin{frame}
\frametitle{\grp}
\begin{clicker}{Which of the following affects the variability (or spread) in the distribution of the sample proportions?  Select all that apply}
\begin{enumerate}
    \item
    the population mean
    \item
    the population standard deviation
    \item
    the sample size
    \item
    the number of samples collected
\end{enumerate}
\end{clicker}
\end{frame}

\begin{frame}
\frametitle{\grp}
\begin{clicker}{When discussing sampling distributions, what started off as KNOWN and UNKNOWN?}
\begin{enumerate}[A.]
    \item[]
    \item
    population proportion: (1) known OR (2) unknown
    \item[]
    \item
    sample proportion: (1) known OR (2) unknown
    \item[]
\end{enumerate}
\end{clicker}
\end{frame}


\begin{frame}
\frametitle{Example}
Suppose that 80\% of Cal Poly students own a Mac laptop, and that we take samples of size $n=100$ from the population of all Cal Poly students.
Specify (with justification) the following features of the distribution of sample proportions and sketch the distribution of sample proportions.
\vskip10pt
\begin{enumerate}
\item shape
\item[]
\item mean
\item[]
\item std dev
\item[]
\end{enumerate}
\vskip50pt
\end{frame}

%===========================================================================================================================
\section[CI general]{Confidence Interval in General}
%===========================================================================================================================
\begin{frame}
\tableofcontents[currentsection, hideallsubsections]
\end{frame}

%\subsection{}
\begin{frame}
\frametitle{Estimation}
Given that we generally collect data from a sample (and not a population), how do we \emph{estimate} population parameter values reliably?  Examples of estimation:
    \begin{itemize}
        \item
        A \href{http://www.npr.org/blogs/thesalt/2013/01/31/170748045/why-mixing-alcohol-with-diet-soda-may-make-you-drunker?sc=tw}{study} found that the average breath alcohol concentration (BrAC) was .091 when subjects drank alcohol mixed with a diet drink. By comparison, BrAC was .077 when the same subjects consumed the same amount of alcohol but with a sugary soda.
        \item
        \href{http://www.bbc.co.uk/news/science-environment-21236690}{Researchers} estimate that domestic cats are responsible for the deaths of between 1.4 and 3.7 billion birds and 6.9-20.7 billion mammals annually.
    \end{itemize}
\end{frame}



\begin{frame}
\frametitle{Point estimate vs interval estimate}
A \textbf{point estimate} is a \textbf{single number} that is our `best guess' for the  population parameter.  Point estimates are given by sample statistics.  \\
\begin{itemize}
    \item
    the average number of hours of sleep on a typical night is 7.024 ($\bar{x}=7.024$)
    \item
    100/208 students indicated the Emory was their $1^{st}$ choice ($\hat{p}=0.48$)
\end{itemize}
\vskip10pt
%\pause
An \textbf{interval estimate} is an \textbf{interval of numbers} within which the \underline{population parameter} value is believed to fall.\\
\begin{itemize}
    \item
    What is a plausible range for the \textbf{population mean} ($\mu$) number of hours of sleep of Emory students in a typical night?
    \item
    What is a plausible range for the \textbf{population proportion} ($p$) of students for whom Emory was their $1^{st}$ choice?
\end{itemize}
\end{frame}

\begin{frame}
\frametitle{Confidence Interval}
A \textbf{confidence interval} is an interval containing the most believable values for a \underline{population parameter}.
\begin{itemize}
    \item
    The probability that this method produces an interval that captures the true parameter value is called the \textbf{confidence level}.
    \item
    The confidence level is a number close to 1, and is most commonly 0.95.
    \item
    A confidence interval with a confidence level of 0.95 is called a 95\% confidence interval.
  \end{itemize}
\end{frame}

\begin{frame}
\frametitle{General Form of  Confidence Interval:}
\begin{center}
    point estimate $\pm$ margin of error
\end{center}
\begin{itemize}
    \item
    the \textbf{point estimate} is your best guess of a population parameter, like $\bar{x}$ or $\hat{p}$
    \item
    the \textbf{margin of error} measures how accurate the point estimate is likely to be in estimated a parameter
\end{itemize}
\end{frame}

\begin{frame}
\frametitle{Relationship between sampling distribution and confidence intervals}
{\renewcommand{\arraystretch}{1.3}
\begin{tabular}{lcc}
\hline
                      & \textbf{Sampling} & \textbf{Confidence} \\
                      & \textbf{Distribution} & \textbf{Interval} \\
\hline
Population proportion & known                 & unknown \\
Sample proportion     & unknown               & known \\
\hline
\end{tabular}}
\end{frame}

%===========================================================================================================================
\section[CI for $p$]{Confidence Interval for a Population Proportion}
%===========================================================================================================================
\begin{frame}
\tableofcontents[currentsection, hideallsubsections]
\end{frame}

%\subsection{}


\begin{frame}
\frametitle{The research question}
How can we estimate the population proportion of Cal Poly students that own a Mac laptop?\\
\vskip10pt
In random sample of Cal Poly students,  50 out of 67 students reported that they own a Mac laptop.
\vskip100pt
\end{frame}



%\begin{frame}
%\frametitle{Unknown values and how they are estimated}
%\begin{columns}
%\column{0.5\textwidth}
%Values that are generally \textbf{unknown}:
%\begin{itemize}
%    \item  the population proportion $p$
%    \item  the standard deviation of the distribution of sample proportions, $\sqrt{p(1-p)/n}$
%\end{itemize}
%\column{0.5\textwidth}
%Values that are \emph{estimated} from a sample of data:
%\begin{itemize}
%    \item  the sample proportion $\hat{p}$
%    \item  the \emph{standard error} of the sample proportion, $\sqrt{\hat{p}(1-\hat{p})/n}$
%\end{itemize}
%\end{columns}
%\end{frame}

\begin{frame}
\frametitle{CI for a population proportion}
\begin{align*}
\hat{p} & \pm z^{*} \times \sqrt{\frac{\hat{p}(1-\hat{p})}{n}}\\
\mbox{\emph{point estimate}} & \pm \mbox{\emph{critical value}} \times \mbox{\emph{standard error}}  \\
\mbox{\emph{point estimate}} & \pm \mbox{\emph{margin of error}}
\end{align*}
\vskip5pt
\begin{itemize}
    \item
    the \textbf{point estimate} is your best guess of a population parameter  $\rightarrow$ $\hat{p}$
    \item
    the \textbf{critical value} establishes your degree of confidence for that interval  $\rightarrow$ use $z$
    \item
    the \textbf{standard error} allows for uncertainty in that point estimate  $\rightarrow$ $\sqrt{\hat{p}(1-\hat{p})/n}$
    \item
   the \textbf{margin of error} is the (critical value $\times$ standard error), and is everything after the $\pm$  $\rightarrow$  $z\times\sqrt{\hat{p}(1-\hat{p})/n}$
\end{itemize}
\end{frame}

\begin{frame}
\frametitle{Choice of \emph{approximate} $z$}
\begin{itemize}
\item[]
$ \mbox{point estimate} \pm \textcolor{OrangeRed}{z^{*}} \times se \rightarrow$ general form of CI\\
\item[]
\item[]
$\mbox{point estimate}  \pm \textcolor{OrangeRed}{1} \times se \rightarrow$ approximate \underline{\hspace{0.3in}} CI \\
\item[]
\item[]
$\mbox{point estimate}  \pm \textcolor{OrangeRed}{2} \times se \rightarrow$ approximate \underline{95\%}  CI \\
\item[]
\item[]
$\mbox{point estimate}  \pm \textcolor{OrangeRed}{3} \times se \rightarrow$ approximate \underline{\hspace{0.3in}} CI  \\
\end{itemize}
\end{frame}


\begin{frame}
\frametitle{Elements of an interpretation of a confidence interval}

\begin{enumerate}
    \item
    State the confidence level
    \item
    Refer to the population
    \item
    State the parameter being estimated
    \item
    Utilize context
    \item
    Include a range of values
    \item[]
\end{enumerate}

At the \bsans{1} \% confidence level, we estimate that the \bsans{2} \bsans{3} of \bsans{4} is in the interval \bsans{5}.
\end{frame}

\begin{frame}
\frametitle{Evaluating claims}
\begin{clicker}{Different faculty members have different guesses on the percent of all Cal Poly students that own laptop.  Based on the interval calculated, which of these claims are plausible?  Mark all that apply.}
\begin{enumerate}
    \item
    Professor A claims that 60\% of students own a Mac laptop.
    \item
    Professor B claims that 70\% of students own a Mac laptop.
    \item
    Professor C claims that 80\% of students own a Mac laptop.
    \item
    Professor D claims that 90\% of students own a Mac laptop.
\end{enumerate}
\end{clicker}
\end{frame}



\begin{frame}[label=assumptions]
\frametitle{Conditions required for a CI for $p$}
    \begin{enumerate}
        \item
        The observations are independent.
        \hyperlink{independentobs}{\beamerreturnbutton{Discussion}}
        %\item[]
        %\textcolor{OrangeRed}{If not, the $se$ tends to be \emph{underestimated}, resulting in CIs that are too narrow (invalid CIs).}
        \item
        $n\hat{p} \geq 10 $ and $n(1-\hat{p}) \geq 10$
        \item[]
        \emph{(at least 10 observed ``successes'' and 10 observed ``failures'')}
        %\item[]
        %\textcolor{OrangeRed}{If not, the sampling distribution of $\hat{p}$ is not approximately normal, and CIs produced by this method are not \emph{valid}.}
    \end{enumerate}
\end{frame}




\begin{frame}
\frametitle{Steps to constructing a confidence interval for a population proportion}
\begin{enumerate}
    \item
    Check your conditions.
    \item[]
    \item
    Identify $z^{*}$ for your specified level of confidence.
    \item[]
    \begin{tabular}{|lcccc|}
\hline
Confidence level & 80\% & 90\% & 95\% & 99\% \\
$z^{*}$  & 1.28 & 1.65 & 1.96 & 2.58 \\
\hline
\end{tabular}
    \item[]
    \item
    Calculate the interval: $\hat{p} \pm z^{*} \times \sqrt{\frac{\hat{p}(1-\hat{p})}{n}}$
    \item[]
\end{enumerate}
\end{frame}

\begin{frame}
\frametitle{Example}
In random sample of Cal Poly students,  50 out of 67 students reported that they own a Mac laptop.  Compute and interpret a 95\% CI for the population proportion of Cal Poly students that own a Mac laptop.
\vskip200pt
\end{frame}


%===========================================================================================================================
\section[Understanding the CI]{Understanding the CI}
%===========================================================================================================================
\begin{frame}
\tableofcontents[currentsection, hideallsubsections]
\end{frame}

%\subsection{}


\begin{frame}
\frametitle{\grp}
A 95\% confidence interval for the true proportion of US citizens who are opposed to issuing traffic tickets from traffic cameras is (0.57, 0.63) based on a sample of 1000 individuals.
\begin{clicker}
{What is the point estimate for the proportion of sampled individuals who are opposed to issuing traffic tickets from traffic cameras?}
\begin{enumerate}
    \item
    0.57
    \item
    0.63
    \item
    0.60
    \item
    0.95
    \item
    not enough information to determine
\end{enumerate}
\end{clicker}
\end{frame}





\begin{frame}
\frametitle{\grp}
\begin{center}
$\hat{p} \pm z^{*} \times \sqrt{\frac{\hat{p}(1-\hat{p})}{n}}$, 90\% $z^{*} = $\underline{\hspace{0.5in}},  95\% $z^{*} = $\underline{\hspace{0.5in}}
\end{center}
\begin{clicker}{The value of $z$ for a  95\% CI is \underline{\hspace{1in}} than the value of $z^{*}$ for a 90\% CI.  This means that higher confidence levels correspond to \underline{\hspace{1in}} confidence intervals.}
\begin{enumerate}
    \item
    greater; wider
    \item
    greater; narrower
    \item
    less; narrower
    \item
    less; wider
\end{enumerate}
\end{clicker}
\vskip10pt
What factors affect the width of the CI?
Constructing a confidence interval is a \emph{compromise} between an acceptable width of your confidence interval and the desired level of confidence in correct inference.
\end{frame}


\begin{frame}
\frametitle{\grp}
\begin{clicker}{Will a 95\% confidence interval always contain the \textbf{estimate} of the population proportion ($\hat{p}$)?}
\begin{enumerate}
    \item
    Yes
    \item
    No
\end{enumerate}
\end{clicker}
\vskip20pt
\begin{clicker}{Will a 95\% confidence interval always contain the \textbf{population} proportion ($p$)?}
\begin{enumerate}
    \item
    Yes
    \item
    No
\end{enumerate}
\end{clicker}
\end{frame}

%\begin{frame}
%\frametitle{Class exercise with partners}
%\begin{enumerate}
%    \item
%    Find the sample proportion.
%    \item
%    Construct a 95\% CI for $p$.
%    \item
%    Plot your sample proportion and CI on the board.
%\end{enumerate}
%\end{frame}

\begin{frame}
\frametitle{The meaning of a 95\% CI}
If we were to repeatedly sample over and over again, \emph{in the long run} 95\% of our confidence intervals would make correct inference.
That is, if we took 100 random samples and calculated 100 confidence intervals, we would expect 95 of those confidence intervals to capture the true parameter value.
\begin{columns}
\column{0.7\textwidth}
\includegraphics[width=1.0\textwidth]{Figures/CIp.png}
\column{0.3\textwidth}
This figures shows 100 samples of size $n=50$ where the true $p=0.50$.  Note that 5 of the 100 CI's don't actually capture the true $p$.
\end{columns}
\end{frame}


\begin{frame}
\frametitle{The meaning of a 95\% CI, continued}
%\begin{center}
%\includegraphics[width=0.5\textwidth]{CIp.png}
%\end{center}
\begin{itemize}
\item
95\% of samples of this size will produce confidence intervals that capture the true proportion.
\item
This can be longwinded, so we say ``We are 95\% confident that the true proportion lies in this interval.''
\item
In reality, we cannot know whether or our sample is one of the 95\% that captured $p$, or one of the unlucky 5\% that did not catch $p$.
\item
95\% confident means that we arrived at this interval by a method that gives us correct results 95\% of the time.
\end{itemize}
\end{frame}

\begin{frame}
\frametitle{\grp}
Based on data from the Winter 2016 STAT 217 class, a 95\% confidence interval for the proportion of Cal Poly students who own a Mac laptop is 0.64 to 0.85.
\begin{clicker}{Which of the following is a correct interpretation?}
\begin{enumerate}
    \item
    We are 95\% confident that the proportion of Cal Poly students from this sample who own a Mac laptop is between 0.64 and 0.85.
    \item
    We are 95\% confident that the population proportion of Cal Poly students who own a Mac laptop is between 0.64 and 0.85.
    \item
    95\% of the time the proportion of Cal Poly students who own a Mac laptop is between 0.64 and 0.85.
    \item
    More than one statement is correct.
\end{enumerate}
\end{clicker}
\end{frame}

\begin{frame}
\frametitle{Interpreting the CI}
\begin{itemize}
    \item
    Interpreting a CI can be challenging
    \item
    Students often try to use their own words, and get the interpretation incorrect (just use my words).
    \item
    Confidence intervals are about values of \underline{population} \underline{parameters}, so \emph{both} pieces of this information \textbf{must} be included in the interpretation.
\end{itemize}
\end{frame}

\begin{frame}
\frametitle{\grp}
\begin{center}
\begin{tabular}{cc}
\hline
 Sample 1 & Sample 2\\
\hline
 $\hat{p}=0.35$        &   $\hat{p}=0.35$ \\
 $n=50$         & $n=100$ \\
 \hline
\end{tabular}
\end{center}
\begin{clicker}{Suppose we construct a 95\% confidence interval for $p$ for both samples.  How will the confidence intervals compare?}
\begin{enumerate}
    \item
    The width of the intervals from the two samples will be the same.
    \item
    The CI for sample 1 will be wider than the CI for sample 2.
    \item
    The CI for sample 2 will be wider than the CI for sample 1.
    \item
    There is not enough information to determine.
 \end{enumerate}
\end{clicker}
\end{frame}

\begin{frame}
\frametitle{\grp}
\begin{clicker}{For each of the following scenarios, indicate if the research question should be answered with a confidence interval for a \textbf{mean} or \textbf{proportion}.  We want to know something about...}{
\begin{enumerate}
\item how many times a day a baby laughs
\item whether or not babies are born with blue eyes
\item the number of months until the baby first sleeps through the night
\item if babies can sit up independently by age 6 months
\item the heart rate of baby immediately after birth
\end{enumerate}
}
\end{clicker}
\end{frame}


%===========================================================================================================================
% Extra
%===========================================================================================================================


\appendix
\newcounter{finalframe}
\setcounter{finalframe}{\value{framenumber}}


\section[Extra Material]{Extra Material}

\subsection{}

\begin{frame}[label=independentobs]
\frametitle{When are observations not independent?}
Observations are not independent when they are correlated with each other.  This occurs when observations are related and are more similar to each other than other observations in the data set.
\begin{itemize}
    \item
    Measurements made on the same subject are typically not independent.
    \begin{itemize}
        \item
        circumference of right and left thigh
        \item
        blood pressure before and after a treatment
    \end{itemize}
    \item
    Observations from different subjects may not always be independent.
    \begin{itemize}
        \item
        siblings, twin pairs, husband-wife
    \end{itemize}
    \item
    Observations from groups of subjects may not always be independent.
    \begin{itemize}
        \item
         children that attend the same elementary school, patients that attend the same in-town clinic
    \end{itemize}
\end{itemize}
\begin{flushright}
\hyperlink{assumptions}{\beamerreturnbutton{Back}}
\end{flushright}
\end{frame}

\end{document} 