

\PassOptionsToPackage{subsection=false}{beamerouterthememiniframes}
\PassOptionsToPackage{dvipsnames,table}{xcolor}
\documentclass[fleqn]{beamer}
\usepackage{graphicx}
\usepackage{multirow}
\usepackage{multicol}
\usepackage{amsmath,amsfonts,amsthm,amsopn}
\usepackage{color, colortbl}
\usepackage{subfig}
\usepackage{wrapfig}
\usepackage{fancybox}
\usepackage{tikz}
\usepackage{fancyhdr}
\usepackage{setspace}
\usepackage{xcolor}
\usepackage{movie15}
\usepackage{pifont}
\usepackage{soul}
\usepackage{booktabs}
\usepackage{fancyvrb,newverbs}
\fvset{fontsize=\footnotesize}
\RecustomVerbatimEnvironment{verbatim}{Verbatim}{}

%\usepackage{fancybox}

\usetheme{Szeged}
\usecolortheme{default}

%\definecolor{links}{HTML}{2A1B81}
%\definecolor{links}{blue!20}
\hypersetup{colorlinks,linkcolor=,urlcolor=blue!80}

\setbeamertemplate{blocks}[rounded]
\setbeamercolor{block title}{bg=blue!40,fg=black}
\setbeamercolor{block body}{bg=blue!10}

%\definecolor{myblue1}{blue!10}

%\colorlet{breaks}{myblue1}

\newenvironment<>{clicker}[1]{%
  \begin{actionenv}#2%
      \def\insertblocktitle{#1}%
      \par%
      \mode<presentation>{%
        \setbeamercolor{block title}{fg=white,bg=magenta}
       \setbeamercolor{block body}{fg=black,bg=magenta!10}
       \setbeamercolor{itemize item}{fg=magenta}
       \setbeamertemplate{itemize item}[triangle]
       \setbeamercolor{enumerate item}{fg=magenta}
     }%
      \usebeamertemplate{block begin}}
    {\par\usebeamertemplate{block end}\end{actionenv}}




\defbeamertemplate*{footline}{infolines theme}
{
  \leavevmode%
  \hbox{%
  \begin{beamercolorbox}[wd=.333333\paperwidth,ht=2.25ex,dp=1ex,left]{author in head/foot}%
    \usebeamerfont{author in head/foot}~~\insertshortinstitute: \insertshorttitle
  \end{beamercolorbox}%
  \begin{beamercolorbox}[wd=.67\paperwidth,ht=2.25ex,dp=1ex,right]{date in head/foot}%
    \usebeamerfont{date in head/foot}%\insertshortdate{}\hspace*{2em}
    \insertframenumber{} / \inserttotalframenumber\hspace*{2ex}
  \end{beamercolorbox}
  }%
  \vskip0pt%
}

\newcommand{\cmark}{\ding{51}}%
\newcommand{\xmark}{\ding{55}}%
\newcommand{\grp}{\textcolor{magenta}{Group Exercise}}
\newcommand{\bsans}[1]{\underline{\hspace{0.2in}\color{blue!80}{#1}\hspace{0.2in}}}
\newcommand{\bs}{\underline{\hspace{0.3in}}}


\definecolor{cverbbg}{gray}{0.93}
\newenvironment{cverbatim}
 {\SaveVerbatim{cverb}}
 {\endSaveVerbatim
  \flushleft\fboxrule=0pt\fboxsep=.5em
  \colorbox{cverbbg}{\BUseVerbatim{cverb}}%
  \endflushleft
}
\newenvironment{lcverbatim}
 {\SaveVerbatim{cverb}}
 {\endSaveVerbatim
  \flushleft\fboxrule=0pt\fboxsep=.5em
  \colorbox{cverbbg}{%
    \makebox[\dimexpr\linewidth-2\fboxsep][l]{\BUseVerbatim{cverb}}%
  }
  \endflushleft
}




\title[Unit 1 Deck 2]{Exploring Variables}
\author[Pileggi]{Shannon Pileggi}

\institute[STAT 217]{STAT 217}

\date{}


\begin{document}

\begin{frame}
\titlepage
\end{frame}


%===========================================================================================================================
\section[Working with Variables]{Working with Variables}
%===========================================================================================================================
\begin{frame}
\tableofcontents[currentsection, hideallsubsections]
\end{frame}

%\subsection{}
\begin{frame}
\frametitle{\grp}
\begin{columns}
\column{0.7\textwidth}
An experiment regarding the physiological cost of reproduction on male fruit flies contains the following variables. Male fruit flies were randomly assigned to cohabitate with one of 5 experimental groups of female fruit flies.  \\
\vskip10pt
\resizebox{1.0\textwidth}{!}{
\begin{tabular}{r|l}
\texttt{type} & Type of experimental assignment \\
           & \hspace{0.1in} $1=$ no females  \\
           & \hspace{0.1in} $2=$ 1 newly pregnant female \\
           & \hspace{0.1in} $3=$ 8 newly pregnant females\\
           & \hspace{0.1in} $4=$ 1 virgin female\\
           & \hspace{0.1in} $5=$ 8 virgin females\\
\texttt{lifespan} & lifespan (days)\\
\texttt{thorax} & length of thorax (mm)\\
\end{tabular}}
\column{0.3\textwidth}
\begin{clicker}{How many quantitative variables does this data set contain?}
\begin{enumerate}
    \item[0]
    \item[1]
    \item[2]
    \item[3]
    \item[4]
    \item[5]
    \item[6]
    \item[7]
\end{enumerate}
\end{clicker}
\end{columns}
\end{frame}


\begin{frame}
\frametitle{10 observations from survey results}
\resizebox{1.0\textwidth}{!}{
\begin{tabular}{llllllll}
\hline
 FirstStats &   gpa & target\_grade &  length\_rel & in\_rel & CP1stChoice & num\_coll & num\_text\\
\hline

          No & 2.500   &          B   &   48.00  &   No   &      Yes    &    3   &    10 \\
         Yes & 3.000    &         B   &   36.00  &   No  &       Yes   &     5  &    100 \\
         Yes & 3.389     &        A   &   24.00  &  Yes   &       No   &    18  &    100 \\
          No & 3.298     &        B   &    4.00  &   No   &       No   &    11   &    30 \\
          No & 3.200    &         A   &    0.25  &   No  &        No  &      8  &    100 \\
          No & 2.920   &          B   &   14.00  &   No  &        No  &      7  &    600 \\
          No & 3.500   &          A   &   12.00  &  Yes  &        No  &      6  &     30 \\
         Yes & 2.800    &         A   &   10.00  &   No   &      Yes  &     13   &   100 \\
          No & 3.470    &         A    &  23.00  &   No  &        No  &     13  &     50 \\
          No & 3.050    &         B   &    6.00  &   No  &        No  &     11  &     35 \\
\hline
\end{tabular}}
\vskip10pt
\resizebox{0.9\textwidth}{!}{
\begin{tabular}{r|l}
\texttt{FirstStats} & first stats class?\\
\texttt{gpa} & GPA\\
\texttt{target\_grade\textunderscore rel} & target grade in stat 217\\
\texttt{length\_rel} & length (in months) of longest serious relationship \\
\texttt{in\_rel} & whether or not currently in a serious relationship\\
\texttt{CP1stChoice} & whether or not Cal Poly was your first choice \\
\texttt{num\_coll} &  number of colleges applied to \\
\texttt{num\_text} & number of texts sent in a day \\
\end{tabular}}
\end{frame}

\begin{frame}[fragile]
\frametitle{Summary of data produced by R}
\begin{lcverbatim}
> summary(survey)
 FirstStats      gpa        CP1stChoice    num_coll
 No :34     Min.   :1.700   No :23      Min.   : 0.000
 Yes:33     1st Qu.:3.000   Yes:44      1st Qu.: 5.000
            Median :3.132               Median : 7.000
            Mean   :3.178               Mean   : 7.239
            3rd Qu.:3.493               3rd Qu.: 9.000
            Max.   :4.000               Max.   :18.000
            NA's   :1
\end{lcverbatim}
\begin{clicker}{}
\begin{enumerate}
\item How are the quantitative and categorical variables summarized differently?
\item[]
\item What else do you notice?
\item[]
\end{enumerate}
\end{clicker}
\end{frame}

\begin{frame}[fragile]
\frametitle{Categorical variable}
\begin{lcverbatim}
> addmargins(table(survey$CP1stChoice))
 No Yes Sum
 23  44  67
\end{lcverbatim}
\begin{clicker}{}
\begin{enumerate}
\item Identify a \emph{statistic} that summarizes this variable.
\item[]
\item[]
\item Produce a visualization of this variable.
\item[]
\item[]
\item[]
\end{enumerate}
\end{clicker}
\vskip100pt
\end{frame}


\begin{frame}
\frametitle{Pie charts}
\begin{columns}
\column{0.5\textwidth}
\includegraphics[trim=10mm 10mm 10mm 0mm,clip,width=1.0\textwidth]{Figures/pie_opinion.pdf}
\column{0.5\textwidth}
\begin{clicker}{Approximately what percent of students rated statistics as a 5?}
\begin{enumerate}
\item 4\%
\item 7\%
\item 10\%
\item 13\%
\item 16\%
\end{enumerate}
\end{clicker}
\end{columns}
% 7.5 percent
\end{frame}


\begin{frame}[fragile]
\frametitle{Quantitative variable - center and variability}
\begin{lcverbatim}
> library(mosaic)
> favstats(survey$num_coll)
 min Q1 median Q3 max     mean       sd  n missing
   0  5      7  9  18 7.238806 3.737969 67       0
\end{lcverbatim}
\begin{clicker}{}
\begin{enumerate}
\item Identify two measures of center, and interpet.
\begin{itemize}
\item
\item[]
\item
\item[]
\end{itemize}
\item Identify two measures of variability, and interpret.
\begin{itemize}
\item
\item[]
\item
\item[]
\end{itemize}
\end{enumerate}
\end{clicker}
\vskip100pt
\end{frame}


\begin{frame}[fragile]
\frametitle{Quantitative variable - position}
\begin{lcverbatim}
> library(mosaic)
> favstats(survey$num_coll)
 min Q1 median Q3 max     mean       sd  n missing
   0  5      7  9  18 7.238806 3.737969 67       0
\end{lcverbatim}
\begin{clicker}{What is the value and interpretation of
}
\begin{enumerate}
\item Q1
\item[]
\item[]
\item Q3
\item[]
\item[]
\end{enumerate}
\end{clicker}
\vskip100pt
\end{frame}

\begin{frame}[fragile]
\frametitle{Quantitative variable - figures}
\begin{columns}
\column{0.33\textwidth}
\includegraphics[width=1.0\textwidth]{Figures/histnumcoll.pdf}
\column{0.33\textwidth}
\includegraphics[width=1.0\textwidth]{Figures/bpnumcoll.pdf}
\column{0.33\textwidth}
\includegraphics[width=1.0\textwidth]{Figures/dpnumcoll.pdf}
\end{columns}
\vskip100pt
\end{frame}


\begin{frame}[fragile]
\frametitle{\grp}
\begin{columns}
\column{0.33\textwidth}
\includegraphics[width=1.0\textwidth]{Figures/histnumcoll.pdf}
\column{0.33\textwidth}
\includegraphics[width=1.0\textwidth]{Figures/bpnumcoll.pdf}
\column{0.33\textwidth}
\includegraphics[width=1.0\textwidth]{Figures/dpnumcoll.pdf}
\end{columns}
\begin{clicker}{True or False}
\begin{enumerate}
\item 5 students applied to 0 colleges
\item 50\% of students applied to 8 colleges or less
\end{enumerate}
\end{clicker}
\vskip50pt
\end{frame}


\begin{frame}
\frametitle{\grp}
%\href{http://www.youtube.com/watch?v=RUwS1uAdUcI}{Video illustrating Gapminder (start at 2min 40sec)}\\
\begin{clicker}{How many variables does a histogram show the distribution of?}
\begin{enumerate}
\item[0]
\item[1]
\item[2]
\item[3]
\item[5.] it depends
\end{enumerate}
\end{clicker}
\end{frame}


\begin{frame}
\frametitle{}
%l b r t
Suppose I asked three groups of 5 college students how many children they want to have.\\
\begin{columns}
\column{0.3\textwidth}
\includegraphics[trim=0mm 40mm 105mm 0mm,clip,width=1.0\textwidth]{Figures/dpkids.pdf}\\
\column{0.7\textwidth}
\begin{clicker}{Which is true? (Don't use a calculator.)}
\begin{enumerate}
    \item
    Group 1 has the largest mean;\\ Group 1 has largest standard deviation
    \item
    Group 3 has the largest mean;\\ Group 3 has largest standard deviation
    \item
    all three groups have same mean;\\ Group 1 has largest standard deviation
    \item
    all three groups have same mean;\\ Group 2 has largest standard deviation
    \item
    all three groups have same mean;\\ Group 3 has largest standard deviation
    \end{enumerate}
\end{clicker}
\end{columns}
\end{frame}




\begin{frame}[fragile]
\frametitle{\grp}
\begin{lcverbatim}
 min Q1 median   Q3 max mean   sd  n missing
   0  3      9 22.5  50 12.5 12.3 63       4
\end{lcverbatim}
\begin{columns}
\column{0.40\textwidth}
\includegraphics[width=1.0\textwidth]{Figures/bp_rel_length.pdf}
\column{0.60\textwidth}
\begin{clicker}{Which of the following statements are \textbf{true}?}
\begin{enumerate}
    \small{
    \item
    Exactly 50\% of students had 9 months as their longest serious relationship
    \item
    50\% of students had a longest serious relationship of 12.5 months or longer.
    \item
    There are no students who have never been in a serious relationship.
    \item
    75\% of students had serious relationships longer than 22.5 months.
    \item 
    None of these are true.}
\end{enumerate}
\end{clicker}
\end{columns}
\end{frame}


\begin{frame}[label=summarizing]
\frametitle{Summarizing and visualizing quantitative variables}
Statistics:
\begin{itemize}
    \item Position: percentiles ($Q1=25^{th}$ , median$=50^{th}$, $Q3=75^{th}=Q3$)
    \item Center: mean, median
    \item Variability: standard deviation, interquartile range
    \item[] \hyperlink{meansd}{\beamerreturnbutton{formulas for mean and sd}} \hyperlink{percentiles}{\beamerreturnbutton{finding percentiles and IQR}}
    \item[]
\end{itemize}
Figures:
\begin{itemize}
    \item dotplot - displays individual values
    \item histogram - displays values in bins
    \item boxplot - based on percentiles \hyperlink{boxplot}{\beamerreturnbutton{how to make a boxplot}}
\end{itemize}
\end{frame}

%===========================================================================================================================
\section[Shape]{Describing the Shape of a Distribution}
%===========================================================================================================================
\begin{frame}
\tableofcontents[currentsection, hideallsubsections]
\end{frame}

%\subsection{}
\begin{frame}
\frametitle{Describing the Shape of Distribution}
\begin{columns}
\column{0.4\textwidth}
\includegraphics[width=1.0\textwidth]{Figures/hist_heights.pdf}
\column{0.6\textwidth}
\begin{itemize}
    \item
    \textbf{unimodal} - has one peak
    \item
    \textbf{symmetric} - mirror image when folded in half
    \item
    \textbf{bell-shaped} (normal) - data follow a bell-shaped curve
\end{itemize}
\end{columns}
\end{frame}

\begin{frame}
\frametitle{Mean vs Median (in symmetric data)}
\begin{columns}
\column{0.4\textwidth}
\includegraphics[trim=0mm 0mm 0mm 0mm,clip,width=1.0\textwidth]{Figures/hist_heights_mm.pdf}
\column{0.6\textwidth}
\begin{itemize}
    \item
    For symmetric data, the mean and the median are approximately equal.
    \item
    In this case, the mean is an appropriate measure of central tendency.
\end{itemize}
\end{columns}
\begin{clicker}{If the mean and median are equal, this means that the data are bell-shaped.}
\begin{enumerate}
\item True
\item False
\end{enumerate}
\end{clicker}
\end{frame}

\begin{frame}
\frametitle{Describing the Shape of Distribution}
\begin{columns}
\column{0.4\textwidth}
\includegraphics[width=1.0\textwidth]{Figures/hist_lifeexp.pdf}
\column{0.6\textwidth}
\begin{itemize}
    \item
    \textbf{unimodal} - has one peak
    \item
    \textbf{left-skewed} - left tail is longer than the right (skew is in the direction of the tail)
    \item
    not symmetric
    \item
    not bell-shaped
\end{itemize}
\end{columns}
\begin{clicker}{Most countries have a life expectancy between 75 and 80 years.}
\begin{enumerate}
\item True
\item False
\end{enumerate}
\end{clicker}
\end{frame}

\begin{frame}
\frametitle{Mean vs Median (in left-skewed data)}
\begin{columns}
\column{0.4\textwidth}
\includegraphics[trim=0mm 0mm 0mm 0mm,clip,width=1.0\textwidth]{Figures/hist_lifeexp_mm.pdf}
\column{0.6\textwidth}
\begin{itemize}
    \item
    For left-skewed data, the mean is less than the median.
    \item
    The mean is pulled in the direction of the long left tail.
    \item
    In highly skewed distributions, the median is preferred over the mean as a measure of central tendency (it better represents what is typical).
\end{itemize}
\end{columns}
\end{frame}


\begin{frame}
\frametitle{Describing the Shape of Distribution}
\begin{columns}
\column{0.4\textwidth}
\includegraphics[width=1.0\textwidth]{Figures/hist_gdpcap.pdf}
\column{0.6\textwidth}
\begin{itemize}
    \item
    \textbf{unimodal} - has one peak
    \item
    \textbf{right-skewed} - right tail is longer than the left (skew is in the direction of the tail)
    \item
    not symmetric
    \item
    not bell-shaped
\end{itemize}
\end{columns}
\end{frame}

\begin{frame}
\frametitle{Mean vs Median (in right-skewed data)}
\begin{columns}
\column{0.4\textwidth}
\includegraphics[trim=0mm 0mm 0mm 0mm,clip,width=1.0\textwidth]{Figures/hist_gdpcap_mm.pdf}
\column{0.6\textwidth}
\begin{itemize}
    \item
    For right-skewed data, the mean is greater than the median.
    \item
    The mean is pulled in the direction of the long right tail.
    \item
    In highly skewed distributions, the median is preferred over the mean as a measure of central tendency (it better represents what is typical).
\end{itemize}
\end{columns}
\end{frame}

\begin{frame}
\frametitle{Describing the Shape of Distribution}
\begin{columns}
\column{0.4\textwidth}
\includegraphics[width=1.0\textwidth]{Figures/hist_popn.pdf}
\column{0.6\textwidth}
\begin{itemize}
    \item
    \textbf{unimodal} - has one peak
    \item
    \textbf{right-skewed} - right tail is longer than the left (skew is in the direction of the tail)
    \item
    has \textbf{outliers} - notice the gap between most of the observations and China and India
    \item
    not symmetric
    \item
    not bell-shaped
\end{itemize}
\end{columns}
\end{frame}

\begin{frame}
\frametitle{\grp}
\begin{columns}
\column{0.30\textwidth}
208 students reported the typical weekly amount of time they spent on extracurricular activities (in hours).  \\
\includegraphics[width=1.15\textwidth]{Figures/hist_hrs.pdf}
\column{0.70\textwidth}
\begin{clicker}{Which of the following statements is \emph{true}?}
\begin{enumerate}
    \item
    This distribution is left-skewed.
    \item
    The mean is an appropriate measure of central tendency to represent a typical student response.
    \item
    As the semester progresses, students are spending fewer hours on extracurricular activities.
    \item
    The maximum hours spent weekly on extracurricular activities is greater than 100.
    \item
    None of these statements are true.
\end{enumerate}
\end{clicker}
\end{columns}
\end{frame}

\begin{frame}
\frametitle{\grp}
A real estate agent is trying to sell a house in a neighborhood in which most houses are worth \$180,000-\$220,000, but a few houses cost much more than that. The house for sale is listed at \$210,000, and the real estate agent is making the argument to the prospective home buyer that this is a really good deal because a typical house sells for \$250,000.
\begin{clicker}{Which statistic is the real estate agent using to support her argument regarding the price of a `typical' house?}
\begin{enumerate}
    \item
    the mean
    \item
    the median
    \item
    the mode
    \item
    the standard deviation
\end{enumerate}
\end{clicker}
\vspace{10pt}
Is this a fair portrayal of `typical' housing prices?
\end{frame}


%===========================================================================================================================
\section[Histogram vs Boxplot]{Histogram vs Boxplot}
%===========================================================================================================================
\begin{frame}
\tableofcontents[currentsection, hideallsubsections]
\end{frame}

%\subsection{}

\begin{frame}
\frametitle{Histogram vs Boxplot}
\includegraphics[width=0.48\textwidth]{Figures/hist_heights.pdf}
%\pause
\includegraphics[width=0.48\textwidth]{Figures/bp_heights.pdf}
\end{frame}

\begin{frame}
\frametitle{Histogram vs Boxplot}
\includegraphics[width=0.48\textwidth]{Figures/hist_lifeexp.pdf}
%\pause
\includegraphics[width=0.48\textwidth]{Figures/bp_lifeexp.pdf}
\end{frame}

\begin{frame}
\frametitle{Histogram vs Boxplot}
\includegraphics[width=0.48\textwidth]{Figures/hist_gdpcap.pdf}
%\pause
\includegraphics[width=0.48\textwidth]{Figures/bp_gdpcap.pdf}
\end{frame}


\begin{frame}
\frametitle{Histogram vs Boxplot}
\includegraphics[width=0.48\textwidth]{Figures/hist_popn.pdf}
%\pause
\includegraphics[width=0.48\textwidth]{Figures/bp_popn.pdf}
\end{frame}


\begin{frame}
\frametitle{\grp}
This is a summary of the distribution of the number of hours spent weekly on extracurricular activities by 208 students.
\begin{center}
\includegraphics[width=0.6\textwidth]{Figures/extraCurrHrs.png}
\end{center}
\begin{clicker}{What is the most plausible shape of this distribution?}
\begin{enumerate}
    \item
    bell-shaped
    \item
    right-skewed
    \item
    left-skewed
    \item
    none of these
\end{enumerate}
\end{clicker}
\end{frame}


\begin{frame}
\frametitle{\grp}
\begin{columns}
\column{0.7\textwidth}
An experiment regarding the physiological cost of reproduction on male fruit flies contains the following variables. Male fruit flies were randomly assigned to cohabitate with one of 5 experimental groups of female fruit flies.  \\
\vskip10pt
\resizebox{1.0\textwidth}{!}{
\begin{tabular}{r|l}
\texttt{type} & Type of experimental assignment \\
           & \hspace{0.1in} $1=$ no females  \\
           & \hspace{0.1in} $2=$ 1 newly pregnant female \\
           & \hspace{0.1in} $3=$ 8 newly pregnant females\\
           & \hspace{0.1in} $4=$ 1 virgin female\\
           & \hspace{0.1in} $5=$ 8 virgin females\\
\texttt{lifespan} & lifespan (days)\\
\texttt{thorax} & length of thorax (mm)\\
\end{tabular}}
\column{0.3\textwidth}
\begin{clicker}{Which figure would you use to plot \texttt{type}?}
\begin{enumerate}
    \item dotplot
    \item histogram
    \item bar plot
    \item pie chart
    \item boxplot
\end{enumerate}
\end{clicker}
\end{columns}
\end{frame}




%===========================================================================================================================
\section[Normal Distribution]{Normal Distribution}
%===========================================================================================================================
\begin{frame}
\tableofcontents[currentsection, hideallsubsections]
\end{frame}

%\subsection{}
\begin{frame}
\frametitle{Normal distribution}
When a distribution is \emph{unimodal}, approximately \emph{symmetric}, and \emph{bell-shaped}, we describe it as a \textbf{normal} distribution..
\begin{center}
\includegraphics[width=0.6\textwidth]{Figures/OI68.png}
\end{center}
For any variable following a normal distribution
\begin{itemize}
    \small{
    \item
    68\% of observations fall within one standard deviation of the mean
    \item
    95\% of observations fall within two standard deviations of the mean
    \item
    99.7\% of observations fall within three standard deviations of the mean}
\end{itemize}
\end{frame}

\begin{frame}
\frametitle{Two normal distributions}
\begin{center}
\includegraphics[width=0.5\textwidth]{Figures/twonormalv2.png}
\end{center}
\end{frame}


\begin{frame}
\frametitle{Using the normal distribution}
\begin{columns}
\column{0.4\textwidth}
Suppose women on average are 64 inches tall with a standard deviation of 3 inches.  
Sketch the distribution of heights of women.
\vskip100pt
%\includegraphics[width=1.0\textwidth]{Figures/hist_heights.pdf}\\
%\begin{center}
%$\bar{x}=63.8$, $s=2.8$
%\end{center}
\textcolor{white}{w}
\column{0.6\textwidth}
\begin{center}
\begin{tabular}{|ccc|}
    \hline
    68\% & 95\% & 99.7\% \\
    $\bar{x}\pm s$ & $\bar{x}\pm 2s$ & $\bar{x}\pm 3s$\\
    \hline
\end{tabular}
\end{center}
\begin{itemize}
    \item
    68\% of women are between \bs and \bs inches tall
    \item
    95\% of women are between \bs and \bs inches tall
    \item
    Nearly all (99.7\%) women are between \bs and \bs inches tall
    \item
    About what percent of women are taller than 73 inches?\\
    %\pause
    %\textcolor{OrangeRed}{About 2.5\%}
\end{itemize}
\vskip100pt
\textcolor{white}{w}
\end{columns}
\end{frame}

\begin{frame}
\frametitle{\grp}
A doctor collects a large set of heart rate measurements that approximately follow a normal distribution.  The doctor reports the the average heart rate is 110 beats per minute, the lowest is 65, and the highest is 155.
\begin{clicker}{Which of the following is most likely to be the standard deviation of this distribution?}
\begin{enumerate}
    \item
    5
    \item
    15
    \item
    35
    \item
    90
\end{enumerate}
\end{clicker}
\end{frame}


\begin{frame}
\frametitle{\grp}
Here we have 9 data sets from samples of size $n=100$.  Which of these 9 data sets come from a normal distribution?
\includegraphics[width=0.55\textwidth]{Figures/normal_100.pdf}
\end{frame}

\begin{frame}
\frametitle{\grp}
Here we have 9 data sets from samples of size $n=30$.  Which of these 9 data sets come from a normal distribution?
\includegraphics[width=0.55\textwidth]{Figures/normal_30.pdf}
\end{frame}

\begin{frame}
\frametitle{$z$-score}
\begin{itemize}
    \item
    Based on the normal distribution, we know it is unusual for an observation to fall more than three standard deviations away from the mean
    \item
    Therefore, one way we can assess if an observation is a potential outlier is to calculate \textbf{how many standard deviations away from the mean it is}.
    \item
    If an observation falls more than three standard deviations away from the mean, it can be regarded as a \textbf{potential outlier}.
\end{itemize}
\begin{center}
$z=\frac{\mbox{value}-\mbox{mean}}{\mbox{standard deviation}}$
\end{center}
\end{frame}

\begin{frame}
\frametitle{$z$-score example}
\begin{columns}
\column{0.4\textwidth}
\includegraphics[width=1.0\textwidth]{Figures/heights_normal.pdf}\\
\begin{center}
mean$=64$, sd$=3$
\end{center}
\column{0.6\textwidth}
\begin{clicker}{Suppose Mary is 67 inches tall.}
\begin{enumerate}
    \item
    What is the z-score for Mary's height?
    \item[]
    \item[]
    \item[]
    \item
    What is the interpretation of this z-score?
    \item[]
    \item[]
    \item[]
\end{enumerate}
\end{clicker}
\end{columns}
\end{frame}

\begin{frame}
\frametitle{the distribution of $z$-scores}
When a $z$-score is calculated from a normal distribution, the $z$-scores themselves follow a normal distribution with a mean of zero and a standard deviation of 1.  We call this the standard normal distribution, and it is often referred as the $z$ distribution.
\begin{center}
\includegraphics[width=0.4\textwidth]{Figures/heights_normal2.pdf}
\includegraphics[width=0.4\textwidth]{Figures/z_scores.pdf}
\end{center}
\end{frame}


\begin{frame}
\frametitle{\grp}
Suppose marketing and accounting majors have their own distribution of starting salaries (that is, each field has its own mean and standard deviation of salaries).  Tom gets a job in marketing and Anna gets a job in accounting.  Tom's z-score for his salary offer is 1.5, and Anna's is 0.67.
\begin{clicker}{Which of the following is \emph{true}?}
\begin{enumerate}
    \item
    Tom's salary offer was higher than Anna's.
    \item
    Since Anna's z-score is less than 1, her salary offer was below the mean.
    \item
    Anna's salary offer is relatively closer to the mean starting salary for her field than Tom's.
    \item
    Tom's salary offer is 150\% better than the mean starting salary for his field.
    \item
    More than one statement is true.
\end{enumerate}
\end{clicker}
\end{frame}

\begin{frame}
\frametitle{\grp}
\begin{columns}
\column{0.35\textwidth}
\begin{center}
\includegraphics[width=1.0\textwidth,trim={0 0 0 2cm},clip]{Figures/hist_heights.pdf}\\
\begin{align*}
n_1&=100\\
\bar{x}_1&=63.8\\
s_1&=?
\end{align*}
\end{center}
\column{0.35\textwidth}
\begin{center}
\includegraphics[width=1.0\textwidth,trim={0 0 0 2cm},clip]{Figures/hist_heights.pdf}\\
\begin{align*}
n_2&=1,000\\
\bar{x}_2&=63.8\\
s_2&=?
\end{align*}
\end{center}
\column{0.3\textwidth}
\begin{clicker}{What is the relationship between $s_1$ and $s_2$?}
\begin{enumerate}
\item $s_1 > s_2$
\item $s_1 < s_2$
\item $s_1 = s_2$
\end{enumerate}
\end{clicker}
\end{columns}
\end{frame}


%===========================================================================================================================
\section[Extra]{Extra}
%===========================================================================================================================
\begin{frame}
\tableofcontents[currentsection, hideallsubsections]
\end{frame}

%\subsection{}
\begin{frame}[label=meansd]
\frametitle{Mean and Standard Deviation}
 Mean (or average): the sum of the observations divided by the number of observations
\begin{equation*}
   \bar{x} = \frac{\sum{x}}{n}
\end{equation*}
\vskip10pt
The standard deviation represents a type of average distance of an observation from the mean.
\begin{equation*}
   s = \sqrt{\frac{\sum(x-\bar{x})^2}{n-1}}
\end{equation*}
\begin{flushright}
\hyperlink{summarizing}{\beamerreturnbutton{Back}}
\end{flushright}
\end{frame}

\begin{frame}[label=percentiles]
\frametitle{Percentiles}
\begin{enumerate}
    \item
    Order your data.
    \item
    Identify the middle of the data.  If $n$ odd, the $50^{th}$ percentile (median) is the value in the middle.  If $n$ is even the $50^{th}$ percentile (median) is the average of the two middle values.
    \item
    Examine the lower half of the data defined by the median (if $n$ odd exclude median).  The median of the lower half is the $25^{th}$ percentile (first quartile).
    \item
    Examine the upper half of the data defined by the median (if $n$ odd exclude median).  The median of the upper half is the $75^{th}$ percentile (third quartile).
    \item[]
\end{enumerate}
The interquartile range (IQR) of the data is the distance between the third and first quartiles: $IQR = Q3 - Q1$
\begin{flushright}
\hyperlink{summarizing}{\beamerreturnbutton{Back}}
\end{flushright}
\end{frame}



\begin{frame}[label=boxplot]
\frametitle{Boxplot}
A \textbf{five-number summary} of data includes the minimum value, Q1, median, Q3, and maximum value.  A five-number summary can be displayed in a \textbf{boxplot}.
\begin{center}
\includegraphics[width=0.6\textwidth]{Figures/boxplot.jpg}
\end{center}
The whiskers extend out to the smallest and largest observations that are \textbf{not} potential outliers.  Potential outliers are indicated with circles.  An observation is a \textbf{potential} outlier if
\begin{itemize}
    \item
    it falls below $Q1-1.5\times IQR$
    \item
    it falls above  $Q3+1.5\times IQR$
\end{itemize}
\begin{flushright}
\hyperlink{summarizing}{\beamerreturnbutton{Back}}
\end{flushright}
\end{frame}






\end{document} 