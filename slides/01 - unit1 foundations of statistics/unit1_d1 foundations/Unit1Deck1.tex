

\PassOptionsToPackage{subsection=false}{beamerouterthememiniframes}
\PassOptionsToPackage{dvipsnames,table}{xcolor}
\documentclass[fleqn]{beamer}
\usepackage{graphicx}
\usepackage{multirow}
\usepackage{multicol}
\usepackage{amsmath,amsfonts,amsthm,amsopn}
\usepackage{color, colortbl}
\usepackage{subfig}
\usepackage{wrapfig}
\usepackage{fancybox}
\usepackage{tikz}
\usepackage{fancyhdr}
\usepackage{setspace}
\usepackage{xcolor}
\usepackage{movie15}
\usepackage{pifont}
\usepackage{soul}
\usepackage{booktabs}
\usepackage{fancyvrb,newverbs}
\fvset{fontsize=\footnotesize}
\RecustomVerbatimEnvironment{verbatim}{Verbatim}{}

%\usepackage{fancybox}

\usetheme{Szeged}
\usecolortheme{default}

%\definecolor{links}{HTML}{2A1B81}
%\definecolor{links}{blue!20}
\hypersetup{colorlinks,linkcolor=,urlcolor=blue!80}

\setbeamertemplate{blocks}[rounded]
\setbeamercolor{block title}{bg=blue!40,fg=black}
\setbeamercolor{block body}{bg=blue!10}

%\definecolor{myblue1}{blue!10}

%\colorlet{breaks}{myblue1}

\newenvironment<>{clicker}[1]{%
  \begin{actionenv}#2%
      \def\insertblocktitle{#1}%
      \par%
      \mode<presentation>{%
        \setbeamercolor{block title}{fg=white,bg=magenta}
       \setbeamercolor{block body}{fg=black,bg=magenta!10}
       \setbeamercolor{itemize item}{fg=magenta}
       \setbeamertemplate{itemize item}[triangle]
       \setbeamercolor{enumerate item}{fg=magenta}
     }%
      \usebeamertemplate{block begin}}
    {\par\usebeamertemplate{block end}\end{actionenv}}




\defbeamertemplate*{footline}{infolines theme}
{
  \leavevmode%
  \hbox{%
  \begin{beamercolorbox}[wd=.333333\paperwidth,ht=2.25ex,dp=1ex,left]{author in head/foot}%
    \usebeamerfont{author in head/foot}~~\insertshortinstitute: \insertshorttitle
  \end{beamercolorbox}%
  \begin{beamercolorbox}[wd=.67\paperwidth,ht=2.25ex,dp=1ex,right]{date in head/foot}%
    \usebeamerfont{date in head/foot}%\insertshortdate{}\hspace*{2em}
    \insertframenumber{} / \inserttotalframenumber\hspace*{2ex}
  \end{beamercolorbox}
  }%
  \vskip0pt%
}

\newcommand{\cmark}{\ding{51}}%
\newcommand{\xmark}{\ding{55}}%
\newcommand{\grp}{\textcolor{magenta}{Group Exercise}}
\newcommand{\bsans}[1]{\underline{\hspace{0.2in}\color{blue!80}{#1}\hspace{0.2in}}}
\newcommand{\bs}{\underline{\hspace{0.3in}}}


\definecolor{cverbbg}{gray}{0.93}
\newenvironment{cverbatim}
 {\SaveVerbatim{cverb}}
 {\endSaveVerbatim
  \flushleft\fboxrule=0pt\fboxsep=.5em
  \colorbox{cverbbg}{\BUseVerbatim{cverb}}%
  \endflushleft
}
\newenvironment{lcverbatim}
 {\SaveVerbatim{cverb}}
 {\endSaveVerbatim
  \flushleft\fboxrule=0pt\fboxsep=.5em
  \colorbox{cverbbg}{%
    \makebox[\dimexpr\linewidth-2\fboxsep][l]{\BUseVerbatim{cverb}}%
  }
  \endflushleft
}




\title[Unit 1 Deck 1]{Introduction}
\author[Pileggi]{Shannon Pileggi}

\institute[STAT 217]{STAT 217}

\date{}


\begin{document}

\begin{frame}
\titlepage
\end{frame}

\begin{frame}
\frametitle{OUTLINE\qquad\qquad\qquad} \tableofcontents[hideallsubsections]
\end{frame}


%===========================================================================================================================
\section[Getting Started]{Getting Started}
%===========================================================================================================================

\subsection{}
\begin{frame}
\begin{clicker}{A study of 120 articles from 1998 to 2006 in 10 leading international psychology journals found that statistical inference was used in about \underline{\hspace{0.5in}} of articles.}
\begin{enumerate}
    \item
    5\%
    \item
    30\%
    \item
    50\%
    \item
    70\%
    \item
    95\%
\end{enumerate}
\end{clicker}
\end{frame}

\begin{frame}
\frametitle{About Dr. Pileggi}
Degrees
\begin{itemize}
\item BS Mathematics and Hispanic Studies
\item MS Biostatistics
\item PhD Biostatistics
\item[]
\end{itemize}
Personal
\begin{itemize}
\item Married
\item Have a 2 year old daughter
\item Have 2 dogs
\item Enjoy: bike commuting, soccer, disc golf, hiking, board games
\item[]
\end{itemize}
\end{frame}

%\begin{frame}
%\frametitle{\grp}
%Use PolyLearn and/or the syllabus to find the following:
%\begin{enumerate}
%    \item
%    When is your final exam?
%    \item
%    What is the best way to communicate with Dr. Pileggi?
%    \item
%    What are some ways you can get help outside of class?
%    \item
%    On the list of skills valued by employers published by Forbes in 2006, what is skill \#6?
%    \item
%    In R, what command is used to create a scatter plot?
%    \item
%    How many project deadlines are there?
%    \item
%    Does attendance count as part of your grade?
%\end{enumerate}
%\end{frame}


%\begin{frame}
%\frametitle{Course Objectives}
%\begin{enumerate}
%\item Differentiate strengths and weaknesses of studies.
%\item Identify appropriate statistical methods when presented with data.
%\item Conduct, explain, and defend your conclusion from a statistical analysis.
%\item Read and interpret basic statistical literature of various sources, such as newspaper articles and academic journals.
%\item Use R as a tool to perform statistical analysis.
%\end{enumerate}
%\end{frame}

\begin{frame}
\frametitle{Course Philosophy}
Most assignments are due \textbf{before} we cover a topic.
\begin{itemize}
\item Pre-lab assignemnts
\item Topic readiness
\item[]
\end{itemize}
Then in class we will discuss elements of your assignment and go deeper into the material.
\begin{itemize}
\item This gives you \emph{repetition}.
\item This gives you an opportunity to ask more and better questions in class.
\item This allows for you get a deeper understanding of the material during your face to face time with the instructor.
\end{itemize}
\end{frame}

\begin{frame}
\frametitle{Lab Groups vs Project Teams}
\begin{center}
Statistics is a team sport!
\end{center}
\vskip10pt
\begin{columns}
\column{0.5\textwidth}
\underline{Lab Groups}
\begin{itemize}
\item
Size: 3-4
\item
Lab groups will be randomly assigned by the instructor
\item
Each of the three course units will have a different group assignment
\item
In class, sit with your lab group to discuss the group exercises
\end{itemize}
\column{0.5\textwidth}
\underline{Project Team}
\begin{itemize}
\item
Size: 3-4
\item
You may select your project team
\item
Project team stays constant throughout the quarter
\item[]
\item[]
\item[]
\end{itemize}
\end{columns}
\end{frame}

\begin{frame}
\frametitle{Statistics in the News}
\begin{itemize}
    \item
    Psychology - \href{http://www.bbc.co.uk/news/technology-23709009}{Facebook use `undermines well-being'}
    \item
    Linguistics - \href{http://www.bbc.co.uk/news/science-environment-19368988}{English language `originated in Turkey'}
    \item
    Education - \href{http://www.bbc.co.uk/news/education-20958928}{US college degree still worth it, says study}
    \item
    Sociology - \href{http://well.blogs.nytimes.com/2012/08/13/ct-scans-more-likely-for-white-children/?ref=health}{Disparities: CT Scans More Likely for White Children}
    \item
    Public Health - \href{http://www.bbc.co.uk/news/health-19372456}{Young cannabis smokers run risk of lower IQ, report claims}
    \item
    Medicine - \href{http://www.bbc.co.uk/news/world-us-canada-19159167}{Alzheimer's disease drug shelved after trial failure}
    \item
    Neuroscience - \href{http://www.bbc.co.uk/news/health-19323061}{Obesity hastens cognitive decline}
    \item
    Anthropology - \href{http://www.bbc.co.uk/news/health-24016988}{Testicle size 'link to father role'}
    \item
    Sports - \href{http://www.bbc.co.uk/news/magazine-23724517}{Does it make statistical sense to sack a football manager?}
    \item
    Gambling - \href{http://www.forbes.com/sites/kiriblakeley/2011/07/21/meet-the-luckiest-woman-in-the-world/}{PhD wins Lottery 4 Times}
\end{itemize}
\end{frame}

\begin{frame}
\frametitle{What is statistics?}
  Statistics is way to make sense of data.  It deals with
\begin{itemize}
    \item
    data collection
    \item
    data analysis
    \item
    interpretation of results
    \item[]
\end{itemize}
Generally, we use statistics to make decisions about a population based on information obtained from a randomly selected \emph{sample}.
\end{frame}

%===========================================================================================================================
\section[Foundations]{Foundations}
%===========================================================================================================================
\subsection{}
\begin{frame}
\tableofcontents[currentsection, hideallsubsections]
\end{frame}


%\begin{frame}
%\frametitle{Research}
%Research often involves the following components...
%\vskip10pt
%\begin{enumerate}
%    \item
%    Formulate a question of interest
%    \item
%    Collect data
%    \item
%    Analyze data
%    \item
%    Interpret results and draw conclusions
%\end{enumerate}
%\vskip10pt
%regardless of your discipline.
%\vskip20pt
%This is statistics!
%\end{frame}

\begin{frame}
\frametitle{Example research question}
What percent of US households earn over \$200,000 annually?
%https://factfinder.census.gov/faces/tableservices/jsf/pages/productview.xhtml?src=CF
\end{frame}

\begin{frame}
\frametitle{Anecdotal evidence vs statistical evidence}
\emph{Informal observations} constitute \textbf{anecdotal} evidence.\\
%\vskip10pt
%\underline{Example}: A young man who committed a robbery and murder at a convenience store was also known to play Grand Theft Auto.\\
\vskip10pt
\begin{itemize}
    \item
    Anecdotal evidence may be true, but is often based on small samples that are not representative of an entire population of interest.
    \item
    We would need a more formal study to make strong conclusions with statistical evidence about relationships observed or findings presented.
\end{itemize}
\end{frame}


\begin{frame}
\frametitle{Population vs Sample}
To answer a research question, you identify the \textbf{population} of interest from which you will collect your \textbf{sample} data.
\begin{columns}
\column{0.6\textwidth}
\begin{itemize}
    \item
    A \textbf{population} is the set of all subjects of interest.
    \item
    A \textbf{sample} is the subset of the population of interest on which you collect data.
\end{itemize}
\column{0.4\textwidth}
\includegraphics[width=1.1\textwidth]{Figures/popsamp.pdf}
\end{columns}
\end{frame}


\begin{frame}
\frametitle{\grp}
Collecting data on everyone is called a \textbf{census}.
\vskip20pt
Suppose I wanted to know about the income level of United States citizens.  Why don't we just collect data on everyone in the population, ie, perform a census?
\end{frame}

\begin{frame}
\frametitle{Example research - helper vs hinder}
\begin{itemize}
    \item
    \emph{Nature} 2007 (Hamlin, Wynn, Bloom) Social evaluation by preverbal infants
    \item
    Research question: Is a moral compass innate or is it learned?
    \item
    16 10-month-old infants from New Haven, CT  witness a ``climber'' trying to make it up a hill
    \item
    the climber then faces either a ``helper'' (climber pushed to the top) or a ``hinderer'' (climber pushed to the bottom)
    \item
    the child was encourage to select either the helper or the hinderer to play with
\end{itemize}
See videos from the \emph{Nature} article.
\end{frame}

\begin{frame}
\frametitle{\grp}
\begin{clicker}{What is the:}
\begin{enumerate}
\item sample?
\item[]
\item[]
\item population?
\item[]
\item[]
\end{enumerate}
\end{clicker}
\end{frame}


%\begin{frame}
%\begin{clicker}{What is the sample and the population?}
%\begin{enumerate}
%    \item
%    \emph{Sample} = the 16 infants;\\ \emph{population} = all humans in the world
%    \item
%    \emph{Sample} = all humans in the world;\\ \emph{population} = the 16 infants
%    \item
%    \emph{Sample} = the 16 infants;\\ \emph{population} = all infants in the US
%    \item
%    \emph{Sample} = the 16 infants;\\ \emph{population} = all infants in the world
%    \item
%    \emph{Sample} = all infants in the US;\\ \emph{population} = the 16 infants
%\end{enumerate}
%\end{clicker}
%\end{frame}

\begin{frame}
\frametitle{Description vs Inference}
After you have collected data, you \textbf{describe} the characteristics of your sample.  You use the data collected from the sample to make \textbf{inference} on the population of interest.
\begin{itemize}
    \item
    \textbf{Descriptive statistics} are used to summarize the collected data through numbers such as averages and percentages.
    \item
    \textbf{Inferential statistics} are used to draw conclusions about a \emph{population}, based on the data obtained from a \emph{sample} of the population.
\end{itemize}
\end{frame}

\begin{frame}
\frametitle{\grp}
The researchers find that 14 out of 16 (87.5\%) infants prefer the helper toy, and conclude that 63-98\% of infants in general would select the helper toy.
\begin{clicker}{Which parts refer to descriptive vs inferential statistics?}
\begin{enumerate}
    \item
    both statements are descriptive statistics
    \item
    both statements are inferential statistics
    \item
    \emph{descriptive} = 87.5\% of infants prefer the helper toy;\\ \emph{inferential} = 63-98\% of infants in general would select the helper toy
    \item
    \emph{descriptive} = 63-98\% of infants in general would select the helper toy;\\ \emph{inferential} = 87.5\% of infants prefer the helper toy
\end{enumerate}
\end{clicker}
\end{frame}


%\begin{frame}
%\frametitle{Statistical Inference}
%Major considerations in statistical inference include:
%\begin{itemize}
%    \item
%    \textbf{Significance}: How \textbf{strong} is the evidence of an effect?
%    \item
%    \textbf{Estimation}: What is the \textbf{size} of the effect?
%    \item
%    \textbf{Generalization}: How \textbf{broadly} do the conclusions apply?
%    \item
%    \textbf{Causation}:  Can we say what \textbf{caused} the effect?
%\end{itemize}
%\end{frame}


\begin{frame}
\frametitle{Parameters vs statistics}
\begin{columns}
\column{0.5\textwidth}
A \textbf{parameter} is a numerical summary of the \emph{population}.
    \begin{itemize}
        \item
        We want to make inference on parameters
        \item
        The true value of a parameter is unknown
        \item
        We denote parameters with Greek letters
        \item[]
    \end{itemize}
\vskip10pt
\textbf{Parameters:}\\
\resizebox{0.95\textwidth}{!}{
\begin{tabular}{|ll|}
    \hline
    population mean & $\mu$\\
    population standard deviation & $\sigma$ \\
    population proportion & $p$ \\
    \hline
\end{tabular}}
\column{0.5\textwidth}
 A \textbf{statistic} is a numerical summary of the \emph{sample}.
    \begin{itemize}
        \item
        We calculate statistics from our \emph{sample} data.
        \item
        Statistics are our best estimate of parameters.
        \item
        We denote statistics with lower case letters, bars, and hats
    \end{itemize}
\vskip10pt
\textbf{Statistics:}\\
\resizebox{0.85\textwidth}{!}{
\begin{tabular}{|ll|}
    \hline
    sample mean & $\bar{x}$\\
    sample standard deviation & $s$ \\
    sample proportion & $\hat{p}$ \\
    \hline
\end{tabular}}
\end{columns}
\end{frame}

\begin{frame}
\frametitle{\grp}
The researchers find that 14 out of 16 (87.5\%) infants prefer the helper toy.
\begin{clicker}{What is the parameter and the statistic?}
\begin{enumerate}
    \item
    Parameter = 87.5\%, Statistic = 87.5\%
    \item
    Parameter = unknown, Statistic = unknown
    \item
    Parameter = 87.5\%, Statistic = unknown
    \item
    Parameter = unknown, Statistic = 87.5\%
    \end{enumerate}
\end{clicker}
\end{frame}

\begin{frame}
\frametitle{Summary}
\begin{itemize}
    \item
    Sample data are an approximate (imperfect) reflection of the population data.
    \item
    What is see in the data is not exactly as things are in the population.
    \item
    Statistical inference is about describing what you think is likely to be happening in the population, based on the observed sample data.
    \item
    In order to do this, we need to understand \emph{variation} in our data.
\end{itemize}
\end{frame}

%===========================================================================================================================
\section[Evidence]{Evidence in Statistics}
%===========================================================================================================================
\subsection{}
\begin{frame}
\tableofcontents[currentsection, hideallsubsections]
\end{frame}


\begin{frame}
\frametitle{\grp}
Think about the Helper vs Hinderer study...  What are two possible explanations for why 14 out of 16 infants selected the helper toy?
\begin{enumerate}
    \item
    \item[]
    \item[]
    \item
    \item[]
    \item[]
\end{enumerate}
Which explanation do you think is more plausible?
\end{frame}

\begin{frame}
\frametitle{Statistical logic}
With statistical reasoning, we generally consider two possible models:
\begin{enumerate}
    \item
    The data arose from random chance
    \item
    The data didn't arise from random chance - something is really going on here
\end{enumerate}
Then we come up with evidence to differentiate between the two models.
\end{frame}

\begin{frame}
\frametitle{Simulating a chance model}
\begin{itemize}
\item
coin flip $=$ an individual infant's selection \\
\item
heads $=$ represents infant selecting helper\\
\item
tails $=$ infant selects hinderer \\
\item
chance of heads $=$ 0.5, the probability that the infant randomly selects the helper\\
\item
one repetition $=$ one set of 16 coin flips to represent the 16 infants in the study\\
\end{itemize}
\end{frame}


\begin{frame}
\frametitle{Plot of class results for number of heads out of 16 coin flips}

\end{frame}

\begin{frame}
\frametitle{\grp}
Analyzing the evidence:
\begin{itemize}
    \item
    Would it be surprising to have 14 out of 16 infants select the helper just by chance?
    \item[]
     \item[]
    \item
    Does this \emph{prove} or \emph{provide evidence} that infants have a moral compass?
     \item[]
      \item[]
\end{itemize}
\end{frame}

%\begin{frame}
%\frametitle{Consider...}
%\begin{itemize}
%    \item
%    \textbf{Significance}: How \textbf{strong} is the evidence of an effect?
%    \item
%    \textbf{Estimation}: What is the \textbf{size} of the effect?
%    \item
%    \textbf{Generalization}: How \textbf{broadly} do the conclusions apply?
%    \item
%    \textbf{Causation}:  Can we say what \textbf{caused} the effect?
%\end{itemize}
%\end{frame}


%===========================================================================================================================
\section[Describing Variables]{Describing variables}
%===========================================================================================================================
\subsection{}
\begin{frame}
\tableofcontents[currentsection, hideallsubsections]
\end{frame}


\begin{frame}
\frametitle{Data sets: Selling Mario Kart on eBay}
\vskip20pt
\begin{columns}
\column{0.25\textwidth}
\emph{Rows} indicate  \\
\textbf{observations} $\rightarrow$\\
\column{0.75\textwidth}
\hspace{0.5in} \textit{Columns} indicate \textbf{variables} \\
\hspace{1.3in}$\downarrow$ \\
\vskip10pt
\resizebox{1.0\textwidth}{!}{
\begin{tabular}{rrrrrrr}
    \hline
    obs & nBids & cond & startPr & totalPr & shipSp & wheels \\
    \hline\hline
    1  &    20 & new  &  0.99 &  51.55 &  standard  &    1 \\
    2  &    13 & used &  0.99 &  37.04 & firstClass &    1 \\
    3  &    16 & new  &  0.99 &  45.50 & firstClass &    1 \\
    4  &    18 & new  &  0.99 &  44.00 &  standard  &    1 \\
    5  &    20 & new  &  0.01 &  71.00 &     media  &    2 \\
    6  &    19 & new  &  0.99 &  45.00 &  standard  &    0 \\
    7  &    13 & used &  0.01 &  37.02 &  standard  &    0 \\
    8  &    15 & new  &  1.00 &  53.99 & upsGround  &    2 \\
    9  &    29 & used &  0.99 &  47.00 &  priority  &    1 \\
    10 &     8 & used & 19.99 &  50.00 & firstClass &    1 \\
    $\vdots$ & $\vdots$ & $\vdots$ & $\vdots$ & $\vdots$ & $\vdots$ & $\vdots$\\
    143 &   13 & new &   1.00 &  54.51 &  upsGround    &  2 \\
    \hline
\end{tabular}}
\end{columns}
\end{frame}

\begin{frame}
\frametitle{Variables}
A \textbf{variable} is any characteristic observed in a study that is measured on a \textbf{unit of observation}.
\begin{itemize}
    \item
    A variable is called \textbf{categorical} if each observation belongs to one of a set of categories.  Categorical variables typically contain descriptive words or phrases.
    \item
    A variable is called \textbf{quantitative} if observations take on numeric values.
\end{itemize}
\end{frame}

\begin{frame}
\frametitle{Which variables are categorical and which are quantitative?}
%\resizebox{1.0\textwidth}{!}{
\begin{tabular}{rrrrrrr}
    \hline
    obs & nBids & cond & startPr & totalPr & shipSp & wheels \\
    \hline\hline
    1  &    20 & new  &  0.99 &  51.55 &  standard  &    1 \\
    2  &    13 & used &  0.99 &  37.04 & firstClass &    1 \\
    3  &    16 & new  &  0.99 &  45.50 & firstClass &    1 \\
    4  &    18 & new  &  0.99 &  44.00 &  standard  &    1 \\
    5  &    20 & new  &  0.01 &  71.00 &     media  &    2 \\
    6  &    19 & new  &  0.99 &  45.00 &  standard  &    0 \\
    7  &    13 & used &  0.01 &  37.02 &  standard  &    0 \\
    8  &    15 & new  &  1.00 &  53.99 & upsGround  &    2 \\
    9  &    29 & used &  0.99 &  47.00 &  priority  &    1 \\
    10 &     8 & used & 19.99 &  50.00 & firstClass &    1 \\
    $\vdots$ & $\vdots$ & $\vdots$ & $\vdots$ & $\vdots$ & $\vdots$ & $\vdots$\\
    143 &   13 & new &   1.00 &  54.51 &  upsGround    &  2 \\
    \hline
\end{tabular}%}
\end{frame}

\begin{frame}
\frametitle{\grp}
In an article published in the \emph{British Medical Journal} (2004), researchers reported that heart transplantations at St. George’s Hospital in London had been suspended in September 2000 after a sudden spike in mortality rate. Of the last 10 heart transplants, 80\% had resulted in deaths within 30 days of the transplant.
\begin{clicker}{What is the unit of observation?}
\begin{enumerate}
    \item
    St. George's Hospital
    \item
    London
    \item
    a heart transplant
    \item
    80\% resulted in deaths
    \item
    the 30 days studied
    \item 
    whether or not the patient died
\end{enumerate}
\end{clicker}

\end{frame}

\begin{frame}
\frametitle{Categorical variables}
There are also more ways to describe categorical variables.
\begin{itemize}
    \item
    A categorical variable that only has two categories is said to be \textbf{dichotomous} (e.g., students that live on or off campus).
    \item
    There is not a special term for categorical variables with more than two categories (e.g., religious beliefs can be classified as Christian, Muslim, Hindu, etc.)
    \item
    A categorical variable is said to be \textbf{ordinal} if its categories have a natural ordering. (e.g., year in college can be classified as freshman, sophomore, junior, senior)
\end{itemize}
\end{frame}


\begin{frame}
\frametitle{\grp}
\begin{clicker}{In the helper vs hinder study, where we found that that 14 out of 16 (87.5\%) infants prefer the helper toy, what was...}
\begin{enumerate}
    \item
    the unit of observation?
    \item[]
    %\item[]
    \item
    the variable studied?
    \item[]
    %\item[]
    \item
    the parameter?
    \item[]
    %\item[]
    \item
    the statistic?
    \item[]
    %\item[]
    \end{enumerate}
\end{clicker}
\end{frame}


%\begin{frame}
%\frametitle{\grp}
%\begin{clicker}{In the helper vs hinder study, where we found that that 14 out of 16 (87.5\%) infants prefer the helper toy, what type of variable was studied?}
%\begin{enumerate}
%    \item
%    quantitative
%    \item
%    categorical
%    \item
%    categorical, dichotomous
%    \item
%    categorical, ordinal
%    \item
%    more than one variable was studied
%\end{enumerate}
%\end{clicker}
%\end{frame}




\end{document} 